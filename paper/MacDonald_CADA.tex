% Template for APA submission with R Markdown

% Stuff changed from PLOS Template
\documentclass[a4paper,man,apacite,floatsintext]{apa6}
\usepackage{apacite}

% amsmath package, useful for mathematical formulas
\usepackage{amsmath}
% amssymb package, useful for mathematical symbols
\usepackage{amssymb}

% hyperref package, useful for hyperlinks
\usepackage{hyperref}

% graphicx package, useful for including eps and pdf graphics
% include graphics with the command \includegraphics
\usepackage{graphicx}

% Sweave(-like)
\usepackage{fancyvrb}
\DefineVerbatimEnvironment{Sinput}{Verbatim}{fontshape=sl}
\DefineVerbatimEnvironment{Soutput}{Verbatim}{}
\DefineVerbatimEnvironment{Scode}{Verbatim}{fontshape=sl}
\newenvironment{Schunk}{}{}
\DefineVerbatimEnvironment{Code}{Verbatim}{}
\DefineVerbatimEnvironment{CodeInput}{Verbatim}{fontshape=sl}
\DefineVerbatimEnvironment{CodeOutput}{Verbatim}{}
\newenvironment{CodeChunk}{}{}

% cite package, to clean up citations in the main text. Do not remove.
\usepackage{cite}

\usepackage{color}

% Use doublespacing - comment out for single spacing
%\usepackage{setspace}
%\doublespacing


% Text layout
\topmargin 0.0cm
\oddsidemargin 0.5cm
\evensidemargin 0.5cm
\textwidth 16cm
\textheight 21cm

% Bold the 'Figure #' in the caption and separate it with a period
% Captions will be left justified
\usepackage[labelfont=bf,labelsep=period,justification=raggedright]{caption}


% Remove brackets from numbering in List of References
\makeatletter
\renewcommand{\@biblabel}[1]{\quad#1.}
\makeatother


% Leave date blank
\date{}

%\pagestyle{myheadings}
%% ** EDIT HERE **


%% ** EDIT HERE **
%% PLEASE INCLUDE ALL MACROS BELOW

%% END MACROS SECTION


% ALL OF THE TITLE PAGE INFORMATION IS SPECIFIED IN THE YAML
\title{\textbf{Language learning from conversations: the influence of social contexts
on active learning behaviors}}
\shorttitle{Language learning is active and social}

\author{Kyle MacDonald}

\affiliation{Department of Psychology, Stanford University}

\authornote{Conceptual Analysis of Dissertation Area. Readers: Michael C. Frank,
Hyowon Gweon, and Anne Fernald}
\abstract{Human language learning is a striking accomplishment. How do children
progress from being unable to understand or produce words to using
language to communicate their goals, to ask questions, and to coordinate
others' behaviors? In this paper, I propose that conversations -- the
fundamental language learning context -- can be productively construed
as opportunities for constrained self-directed learning, allowing
children to shape the input to language learning mechansims. I also
argue that integrating formal models of social learning with models of
decision making provides a productive framework to better understand the
mutual influence of these two factors on language learning. Finally, I
present a conceptual analysis of children's decisions about where to
allocate visual attention during language comprehension to illustrate
the utility of this framework.}
\keywords{language acquisition, active learning, social learning, theory}

\begin{document}
\maketitle

\begin{quote}
\emph{Conversation provides a forum for using language. It displays
language embedded in larger systems for communication and so should
present children with critical material for making sense of language as
they try to understand others and make themselves understood.} (Clark
(2009), p.~21)
\end{quote}

\section{Introduction}\label{introduction}

Human language learning is a remarkable feat. Consider that children,
despite possesing limited processing capabilities, develop language
skills quite rapidly, building an adult vocabulary ranging betewen
50,000 to 100,000 distinct words (P. Bloom, 2002). However, even to
learn the meaning of a concrete noun, children must solve a variety of
complex learning problems, such as segmenting discrete words from
continuous speech stream (Saffran, Aslin, \& Newport, 1996), reducing
referential uncertainty to link words to concepts (Quine, 1960), and
making the appropriate generalizations to new contexts (Xu \& Tenenbaum,
2007). What makes this process even more impressive is that children
must accomplish all of this using input that can be noisy, transient,
and unfolds rapidly in time. How do children do it?

Fortunately, young language learners do not have to solve these problems
on their own since the majority of input to language learning mechansims
comes from \emph{conversations} with more knowledable others.
Social-pragmatic theories of language acquisition have emphasized the
importance of these social interactions for reducing the difficulty of
various language learning challenges (P. Bloom, 2002; Clark, 2009;
Hollich et al., 2000). Moreover, recent modeling work has formalized the
role of social reasoning processes (e.g., thinking about others' goals)
in word/concept learning (Frank \& Goodman, 2014; Shafto, Goodman, \&
Frank, 2012) and infant-directed speech (Eaves Jr, Feldman, Griffiths,
\& Shafto, 2016).

However, social partners are just one of the forces that can shape the
input to children's language learning mechansims. Another important
factor is the choices that children make. That is, children are not just
passive recipients of language; instead, they can select behaviors
(e.g., ask questions) that modulate the content, pacing, and sequence of
information during conversation. Recent empirical work across a variety
of domains has begun to explore the effects of self-directed choice on
learning outcomes, showing that additional control can lead to speedier
and more robust learning compared to more passive contexts (Castro et
al., 2009; Markant \& Gureckis, 2014; Settles, 2012).

One important challenge for the field of language aquisition is to
precisely characterize the interactions between fundamentally social
learning contexts, social reasoning, and children's developing ability
to exert control over their environments. In this paper, I argue that
learning from conversation can be productively construed as providing
rich opportunities for \emph{constrained active learning}. The key
insight is that the presence of other people qualitatively changes the
cost/benefit calculus for learners' choices, which in turn shapes
subsequent input to language leanring mechanisms. The plan for the paper
is as follows. I first review evidence that supports the importance of
social contexts for language learning, integrating ideas from formal
models of social learning. In part II, I connect the social learning
ideas with research on formal models of decision making that emphasize
the utility structure of different actions available to the learner. I
present a conceptual utility analysis of a wide variety of behaviors
available to the developing language learner. I end by introducing a
case study -- the allocation of visual attention to gather information
for language comprehension -- to illustrate the usefulness of this
framework for understanding the joint effects of social and
self-directed learning processes on language acquisition.

\section{Part 1: Language learning is a social
process}\label{part-1-language-learning-is-a-social-process}

\subsection{Language learning as social
interaction}\label{language-learning-as-social-interaction}

\subsection{Formal models of learning from
others}\label{formal-models-of-learning-from-others}

Social learning is the accumulation of knowledge based on the sampling
decisions of other agents (e.g., via the framework in Shafto et al.,
2012). Requires reasoning about why the other agent made the choices
they did.

\subsection{Why is social learning so
powerful?}\label{why-is-social-learning-so-powerful}

\subsection{Different models of social
learning}\label{different-models-of-social-learning}

Examples: * Sobel and Kirkham * observational learning * pedagogical
inference * social as attention vs.~social as changing underlying
inferences because of reasoning about others minds

\subsection{What is missing from the social learning
account?}\label{what-is-missing-from-the-social-learning-account}

Models of seeking information from social targets: * Baldwin \& Moses
(1998): The Ontogeny of Social Information gathering * Chouinard (2007):
Children's questions as learning mechanism

\section{Part 2: Active learning as rational decision
making}\label{part-2-active-learning-as-rational-decision-making}

Classic theories of development have shared the intuition that knowledge
acquisition is a fundamentally active process, with the learner playing
an important role in shaping the learning environment (Bruner, Piaget,
Vygotsky). And recent theoretical and empirical work has formalized
these ideas by characterizing development as a process of active
hypothesis testing and theory revision that can be described by
principles of Bayesian reasoning (Gopnik; Schulz, 2007).

Moreover, the potential benefits of active learning have been the focus
of empirical work from a broad set of research areas, including fields
such as education (Grabinger \& Dunlap, 1995), machine learning
(Settles, 2012), and cognitive science (Castro et al., 2009). Across
these different literatures, the term ``active learning'' has been used
to mean a variety of behaviors such as question asking (cite), increased
physical activity (cite), or active memory retrieval (cite).

In this paper, I focus on a specific subset of active learning
behaviors: the \emph{decisions} that people make, or could make, during
learning. That is, the capacity to exert control over the learning
experience, including the selection, sequencing, or pacing of new
information.

Markant et al. (2016) describe the benefits as ``enhanced memory may be
a common outcome of active learning that can arise from a number of
distinct mechanisms, depending on the kinds of control afforded by an
instructional activity''

Examples: * Encoding of Distinctive Sensorimotor Associations *
Elaborative Encoding Through Goal-Directed Search and Planning *
Co-ordination of Selective Attention and Memory Encoding * Adaptive
Selection of Material * Enhanced Memory Due to Metacognitive Monitoring

Focusing of learners' choices is beneficial since there is a rich
literature that has formalized decision making process, which can be
used to describe choices made during learning.

\subsection{Active learning as decision
making}\label{active-learning-as-decision-making}

Active learning takes into account a utility structure that can include
both the costs of data acquisition and the rewards of choosing an
example (e.g., in terms of information acquisition/uncertainty reduction
relative to some longer term learning goal).

Process: 1) analyze costs and benefits of behavior 2) planning models
that take into account long-term value 3) decisions in the brain and in
non-human primates

\subsection{What is missing from the active learning
account?}\label{what-is-missing-from-the-active-learning-account}

The presence of another agent can change the cost/benefit structure of
choices made for learning and therefore we must include this information
in our models of self-directed learning, which often view the learner as
moving back and forth between active exploration and passive reception.
This type of active learning account does not leave room for social
reasoning processes (i.e., native utility calculus stuff) to change the
value of an active learning behavior.

\section{Part 3: Active information gathering via visual fixations to
understand
language}\label{part-3-active-information-gathering-via-visual-fixations-to-understand-language}

\section{Conclusion}\label{conclusion}

Models of self-directed learning cannot continue to ignore the
social-communicative context in which learning often occurs.
Reasoning/inferences about other people should modulate the choices that
learners make: whether it's who to talk to, what to attend to, or what
questions to ask.

\newpage

\section{References}\label{references}

\setlength{\parindent}{-0.4in} \setlength{\leftskip}{0.125in} \noindent

\hypertarget{refs}{}
\hypertarget{ref-bloom2002children}{}
Bloom, P. (2002). \emph{How children learn the meaning of words}. The
MIT Press.

\hypertarget{ref-castro2009human}{}
Castro, R. M., Kalish, C., Nowak, R., Qian, R., Rogers, T., \& Zhu, X.
(2009). Human active learning. In \emph{Advances in neural information
processing systems} (pp. 241--248).

\hypertarget{ref-clark2009first}{}
Clark, E. V. (2009). \emph{First language acquisition}. Cambridge
University Press.

\hypertarget{ref-eaves2016infant}{}
Eaves Jr, B. S., Feldman, N. H., Griffiths, T. L., \& Shafto, P. (2016).
Infant-directed speech is consistent with teaching. \emph{Psychological
Review}, \emph{123}(6), 758.

\hypertarget{ref-frank2014inferring}{}
Frank, M. C., \& Goodman, N. D. (2014). Inferring word meanings by
assuming that speakers are informative. \emph{Cognitive Psychology},
\emph{75}, 80--96.

\hypertarget{ref-grabinger1995rich}{}
Grabinger, R. S., \& Dunlap, J. C. (1995). Rich environments for active
learning: A definition. \emph{Research in Learning Technology},
\emph{3}(2).

\hypertarget{ref-hollich2000breaking}{}
Hollich, G. J., Hirsh-Pasek, K., Golinkoff, R. M., Brand, R. J., Brown,
E., Chung, H. L., \ldots{} Bloom, L. (2000). Breaking the language
barrier: An emergentist coalition model for the origins of word
learning. \emph{Monographs of the Society for Research in Child
Development}, i--135.

\hypertarget{ref-markant2014better}{}
Markant, D. B., \& Gureckis, T. M. (2014). Is it better to select or to
receive? Learning via active and passive hypothesis testing.
\emph{Journal of Experimental Psychology: General}, \emph{143}(1), 94.

\hypertarget{ref-quine19600}{}
Quine, W. V. (1960). 0. word and object. \emph{111e MIT Press}.

\hypertarget{ref-saffran1996statistical}{}
Saffran, J. R., Aslin, R. N., \& Newport, E. L. (1996). Statistical
learning by 8-month-old infants. \emph{Science}, \emph{274}(5294),
1926--1928.

\hypertarget{ref-settles2012active}{}
Settles, B. (2012). Active learning. \emph{Synthesis Lectures on
Artificial Intelligence and Machine Learning}, \emph{6}(1), 1--114.

\hypertarget{ref-shafto2012learning}{}
Shafto, P., Goodman, N. D., \& Frank, M. C. (2012). Learning from others
the consequences of psychological reasoning for human learning.
\emph{Perspectives on Psychological Science}, \emph{7}(4), 341--351.

\hypertarget{ref-xu2007word}{}
Xu, F., \& Tenenbaum, J. B. (2007). Word learning as bayesian inference.
\emph{Psychological Review}, \emph{114}(2), 245.

\bibliography{}

\end{document}
