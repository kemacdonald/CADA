% Template for APA submission with R Markdown

% Stuff changed from PLOS Template
\documentclass[a4paper,man,apacite,floatsintext]{apa6}
\usepackage{apacite}

% amsmath package, useful for mathematical formulas
\usepackage{amsmath}
% amssymb package, useful for mathematical symbols
\usepackage{amssymb}

% hyperref package, useful for hyperlinks
\usepackage{hyperref}

% graphicx package, useful for including eps and pdf graphics
% include graphics with the command \includegraphics
\usepackage{graphicx}

% Sweave(-like)
\usepackage{fancyvrb}
\DefineVerbatimEnvironment{Sinput}{Verbatim}{fontshape=sl}
\DefineVerbatimEnvironment{Soutput}{Verbatim}{}
\DefineVerbatimEnvironment{Scode}{Verbatim}{fontshape=sl}
\newenvironment{Schunk}{}{}
\DefineVerbatimEnvironment{Code}{Verbatim}{}
\DefineVerbatimEnvironment{CodeInput}{Verbatim}{fontshape=sl}
\DefineVerbatimEnvironment{CodeOutput}{Verbatim}{}
\newenvironment{CodeChunk}{}{}

% cite package, to clean up citations in the main text. Do not remove.
\usepackage{cite}

\usepackage{color}

% Use doublespacing - comment out for single spacing
%\usepackage{setspace}
%\doublespacing


% Text layout
\topmargin 0.0cm
\oddsidemargin 0.5cm
\evensidemargin 0.5cm
\textwidth 16cm
\textheight 21cm

% Bold the 'Figure #' in the caption and separate it with a period
% Captions will be left justified
\usepackage[labelfont=bf,labelsep=period,justification=raggedright]{caption}


% Remove brackets from numbering in List of References
\makeatletter
\renewcommand{\@biblabel}[1]{\quad#1.}
\makeatother


% Leave date blank
\date{}

%\pagestyle{myheadings}
%% ** EDIT HERE **


%% ** EDIT HERE **
%% PLEASE INCLUDE ALL MACROS BELOW

%% END MACROS SECTION


% ALL OF THE TITLE PAGE INFORMATION IS SPECIFIED IN THE YAML
\title{\textbf{Children's active learning is social learning}}
\shorttitle{Active learning is social}

\author{Kyle MacDonald}

\affiliation{Department of Psychology, Stanford University}

\authornote{Conceptual Analysis of Dissertation Area. Readers: Michael C. Frank,
Hyowon Gweon, and Anne Fernald}
\abstract{Children's rapid conceptual development is one of the more remarkable
features of human cognition. How do they learn so much so quickly?
Social learning theories argue for the importance of learning from more
knowledgeable others. In contrast, active learning accounts focus on
children's knowledge acquisition via self-directed exploration. In this
paper, I argue that an important step towards a more complete theory of
early learning is to understand how active learning behaviors unfold in
social learning contexts. To integrate the two theoretical accounts, I
use ideas from theories of rational decision making that emphasize the
expected utility and cost of different actions in order to explain
choice behavior. The key insight is that the costs and benefits of
active learning behaviors (e.g., metacognitive monitoring, information
seeking, and question asking) are fundamentally shaped by interactions
with other people.}
\keywords{human learning, active learning, social learning, decision making,
theory}

\begin{document}
\maketitle

\section{Introduction}\label{introduction}

Human learning is remarkable. Consider that children, despite
limitations on their general processing capabilities, are able to
acquire new concepts at a high rate, eventually reaching an adult
vocabulary ranging between 50,000 to 100,000 words (P. Bloom, 2002). And
they accomplish this while also developing motor skills, learning social
norms, and building causal knowledge. What sorts of processes can
account for children's prodigious learning abilities?

Social learning theories offer a solution by pointing out that children
do not solve these problems on their own. And although children learn a
great deal from observation, they are typically surrounded by parents,
other knowledgeable adults, or older peers -- all of whom likely know
more than they do. These social contexts can bootstrap learning via
several mechanisms. For example, work on early language acquisition
shows that social partners provide input that is tuned to children's
cognitive abilities (Eaves Jr, Feldman, Griffiths, \& Shafto, 2016;
Fernald \& Kuhl, 1987), that guides children's attention to important
features in the world (Yu \& Ballard, 2007), and that increases levels
of arousal and sustained attention, which lead to better learning (P. K.
Kuhl, 2007; Yu \& Smith, 2016).

Social contexts can also change the computations that support children'
learning from evidence. Recent work on both concept learning and causal
intervention suggests that the presence of another person leads the
learner to reason about \emph{why} people perform certain actions. The
key insight is that knowledge of the underlying process that generates
examples allows learners to make more appropriate inferences that speed
learning (Bonawitz \& Shafto, 2016; Frank, Goodman, \& Tenenbaum, 2009;
Shafto, Goodman, \& Griffiths, 2014). For example, people will draw
different inferences from observing the same actions depending on
whether they think that the behavior was accidental or intentional.
Moreover, adults and children will make even stronger inferences if they
think an action was selected with the goal to help them learn (i.e.,
teaching) (Shafto, Goodman, \& Frank, 2012).

However, children are not merely passive recipients of information --
from people or from the world. Instead, children actively select
behaviors (e.g., asking questions or choosing where to allocate
attention) that change the content, pacing, and sequence of the
information that they receive. In fact, recent theorizing and empirical
work characterizes early learning as a process of exploration and
hypothesis testing similar to the scientific method (Gopnik, Meltzoff,
\& Kuhl, 1999; Schulz, 2012). Moreover, recent empirical work across a
variety of domains (education, machine learning, and cognitive science)
has begun to explore the benefits of self-directed choice for speeding
learning outcomes by increasing learners' attention/arousal or by
providing learners with better information that is more tightly linked
to their current cognitive state and interests (Castro et al., 2009; D.
B. Markant \& Gureckis, 2014; Settles, 2012).

Thus, both social and active contexts can facilitate cognitive
development by activating distinct learning processes and by providing
the learner with better information. However, real-world learning
involves a mixture of these processes where young learners must
consistently integrate information that they generate with information
provided by other peopole. Thus, one of the fundamental challenges for
increasing our understanding of human learning is to precisely
characterize the interplay between social contexts and self-directed
learning behaviors.

In this paper, I use the framework of Optimal Experiment Design (CITE)
to integrate ideas from the active and social learning accounts. The key
insight is that social learning contexts can \emph{constrain} the
decision processes that the self-directed learner makes. I argue that
taking into account the effects of social context provides a way to
explain a diverse set of findings on children's uncertainty monitoring
(markman; kim et al.), information seeking (begus), and question asking
(katz et al. (2010). Before presenting the integrative framework of
active learning within social contexts, it is useful to review evidence
showing how both social contexts and self-directed behaviors can
modulate learning outcomes.

\section{Part I: Learning from other
people}\label{part-i-learning-from-other-people}

Social learning theories argue that children's rapid conceptual
development is facilitated by humans' unique capacity to transmit and
acquire information from other people.\footnote{In this paper, I define
  ``social'' contexts as learning environments where another agent is
  present. This definition includes a broad range of social learning
  behaviors: e.g., observation, imitation, and learning from direct
  pedagogy.} One of the primary benefits of cultural learning is that
children gain access to knowledge that has accumulated over many
generations; information that would be far too complex for any
individual person to figure out on their own (Boyd, Richerson, \&
Henrich, 2011). In addition to these cumulative effects, social contexts
facilitate learning because more knowledgeable others select the input
that could best support children's learning (Kline, 2015; Shafto et al.,
2012), providing learning opportunities for generalizable information
(Csibra \& Gergely, 2009).

There is now a large body of empirical work on children's learning that
show the effect of social contexts across a variety of domains. These
learning effects manifest via different pathways such as guiding
attention, increasing arousal, providing better information, and
changing the strength of children's inferences. In this section, I
briefly review the evidence for the role of each of these social
learning processes, with the goal of providing a high-level taxonomy of
social learning effects. Outlining these social learning effects will
set the stage for the discussion of how they shape self-directed
learning behaviors in Part III.

\subsubsection{Social interactions enhance
attention}\label{social-interactions-enhance-attention}

From infancy humans preferentially attend to social information. For
example, newborn infants will choose to look at face-like patterns
compared to other abstract configurations (Johnson, Dziurawiec, Ellis,
\& Morton, 1991) and will even show a preference for faces that make
direct eye contact compared to faces with averted gaze (Farroni, Csibra,
Simion, \& Johnson, 2002). In the auditory domain, newborns prefer to
listen to speech over non-speech (Vouloumanos \& Werker, 2007), their
mother's voice over other voices (DeCasper, Fifer, Oates, \& Sheldon,
1987), and infant-directed speech over adult-directed speech (Cooper \&
Aslin, 1990; Fernald \& Kuhl, 1987; Pegg, Werker, \& McLeod, 1992). And
recent work by Yu \& Smith (2016) using head-mounted eye trackers to
record parent-child interactions found that one-year-olds will sustain
visual attention to an object longer when their parents' had previously
looked at that object.

These early attentional biases can lead to differential outcomes when
learning occurs with another person present. For example, 4-month-olds'
show better memory for faces if that face gazed directly at them as
compared to memory for a face with averted gaze (Farroni, Massaccesi,
Menon, \& Johnson, 2007) and for objects if an adult gazed at that
object during learning (Cleveland, Schug, \& Striano, 2007; Reid \&
Striano, 2005). Moreover, 7-month-olds perform better at word
segmentation if the words are presented in infant-directed speech
compared to adult-directed speech (Thiessen, Hill, \& Saffran, 2005).

P. K. Kuhl (2007) refer to these effects as ``social gating'' phenomena
since the presence of another person activates or enhances children's
underlying computational learning mechanisms such as attention. One
particularly striking piece of evidence for the social gating hypothesis
comes from P. K. Kuhl, Tsao, \& Liu (2003)`s study of infants'
foreign-language phonetic learning. In this experiment, 9-10 month-old
English-learning infants listened to Mandarin speakers either via live
interactions or via audiovisual recordings and their ability to
discriminate Mandarin-specific phonemes was assessed two months later.
Only the infants who were exposed to Mandarin within social interactions
were able to succeed on the phonetic discrimination task and infants in
the audiovisual recording condition showed no evidence of learning. P.
K. Kuhl et al. (2003) also provided evidence that infants in the social
interaction condition showed higher rates of visual attention to the
speaker, suggesting that the social contexts enhanced learning by
increasing children's attention to the input.

The common thread across these findings is that the presence of another
person is a particularly good way to increase attention. In this model,
social input becomes more salient and therefore more likely to come into
contact with general learning mechanisms. However, increases in arousal,
attention, and memory are only one way that social contexts can
influence learning. In fact, one of the defining features of early
learning environments is the presence of other people who know more than
the child, creating opportunities for more knowledgeable others to
select learning experiences that are particularly beneficial -- either
because the information is tuned to children's current cognitive
abilities or because the information is likely to be generalizable.

\subsubsection{\texorpdfstring{Social interactions provide ``good''
information}{Social interactions provide good information}}\label{social-interactions-provide-good-information}

The notion that children's input might be shaped to facilitate their
learning is a key tenet of several influential theories of cognitive
development (e.g., Zone of Proximal Development (Vygotsky, 1987), Guided
Participation (Rogoff et al., 1993), and Natural Pedagogy (Csibra \&
Gergely, 2009)). But how do social interactions provide particularly
useful information for children's learning?

A particularly compelling set of evidence comes from studies of how
caregivers alter the way they communicate with young children. That is,
adults do not speak to children in the same way as they speak to other
adults; instead, they exaggerate prosody, reduce speed, shorten
utterances, and elevate both pitch and affect (for a review, see
(Fernald \& Simon, 1984)). And subsequent empirical work has shown that
these features of ``infant-directed speech'' facilitate vowel learning
(Adriaans \& Swingley, 2017; De Boer \& Kuhl, 2003), word segmentation
(Fernald \& Mazzie, 1991; Thiessen et al., 2005), word recognition
(Singh, Nestor, Parikh, \& Yull, 2009), and word learning (Graf Estes \&
Hurley, 2013).

Work on infants' early vocal production also provides evidence for the
importance social feedback, highlighting the feature of
\emph{contingency}. For example, Goldstein \& Schwade (2008) measured
whether infants modified their babbling to produce more speech-like
sounds after interacting with caregivers who either provided contingent
or non-contingent responses to infants' babbling. They found that only
infants in the contingent feedback condition changed their vocalization
behavior to produce more adult-like language forms. Goldstein \& Schwade
(2008) hypothesized that the contingency effect was driven by infants'
receiving input that was particularly useful for solving this learning
problem since the feedback was close in time to infants' vocalizations,
making it easier for them to compare discrepancies between the two.

A third piece of evidence comes from research on children's early word
learning. Social-pragmatic theories of language acquisition have
emphasized the importance of social cues for reducing the (in principle)
unlimited amount of referential uncertainty present when children are
trying to acquire novel words (P. Bloom, 2002; Clark, 2009; Hollich et
al., 2000). Empirical work by Yu \& Smith (2012) shows that young
learners tend to retain words that are accompanied by clear referential
cues, which serve to make a single object dominant in the visual field
(see also (Yu \& Smith, 2013; Yu, Ballard, \& Aslin, 2005). Moreover,
correlational studies show positive links between early vocabulary
development and parents' tendency to refer to objects that children are
already attending to (i.e., ``follow-in'' labeling) (Tomasello \&
Farrar, 1986).

Thus far, I have reviewed evidence showing that social information can
benefit learning because it enhances attention and it contains features
that make it easier to learn. Learning from other people also changes
learning by engaging distinct social reasoning processes that change how
learners interpret and learn from evidence.

\subsubsection{Social interactions change inferences and
generalization}\label{social-interactions-change-inferences-and-generalization}

Perhaps one of the defining features of human social learning is that
teachers and learners' actions are not random. Instead, people select
behaviors with respect to some goal (e.g., to communicate a concept),
and learners reason about \emph{why} someone chose to perform a
particular action. The key point is that this reciprocal process of
reasoning about others' goal-directed actions can change how people
interpret superficially similar behaviors, altering the learning
process.

In recent empirical and modeling work, Shafto et al. (2012) formalized
this social reasoning process within the framework of Bayesian models of
cognition. In these models, learning is a process of belief updating
that depends on two factors: what the learner believed before seeing the
data and what the learner thinks about the process that generated the
data. The key insight is that if the learner assumes that information is
generated with the intention to communicate/teach, then they can make
``stronger'' inferences.\footnote{Formally, these models change the form
  of the likelihood term in Bayes theorem in order to capture a person's
  theory of how data are generated.}

For example, Goodman, Baker, \& Tenenbaum (2009) presented adults with
causal learning scenarios with the following structure: either the
participant or someone else who knows the causal structure generates an
effect (e.g., growing flowers) by performing two actions at the same
time (e.g., pouring a yellow liquid and a blue liquid). The
participant's task was to identify the correct causal structure. Results
showed that when participants thought the other person was
knowledgeable, they were more likely to infer that performing
\emph{both} actions was necessary. In contrast, when the participant
performed the action on their own (and did not know the causal
structure), adults were less sure that both actions were necessary.
Shafto et al. (2012) interpreted these results as learners going through
a psychological reasoning process such as ``if the other person was
knowledgeable and wanted to generate the effect, he would definitely
perform both actions if that was the correct causal structure.''

Similar psychological reasoning effects have been shown in the domains
of word learning (Frank \& Goodman, 2014; Xu \& Tenenbaum, 2007),
selective trust in testimony (Shafto, Eaves, Navarro, \& Perfors, 2012),
tool use (Sage \& Baldwin, 2011), and concept learning (Shafto et al.,
2014). Moreover, there is evidence that even young learners are
sensitive to the presence of others' goal-directed behaviors. For
example, Yoon, Johnson, \& Csibra (2008) showed that 8-month-olds will
encode an object's identity if their attention was directed by a
communicative point, but they will encode an object's spatial location
if their attention was directed by non-communicative reach. And Senju \&
Csibra (2008) found that infants will follow another person's gaze only
if the gaze event was preceded by the person providing a relevant,
communicative cue (e.g., infant-directed speech or direct eye contact).

In addition to being easier to learn, information acquired in social
contexts is also more likely to generalize and be useful beyond the
current learning context. Csibra \& Gergely (2009) argue that this
assumption of \emph{generalizability} is a fundamental component of
``Natural Pedagogy'' -- a uniquely human communication system that
allows adults to efficiently pass along cultural knowledge to children.
Experimental work testing predictions from this account shows that
children are biased to think that information presented in communicative
contexts is generalizable (Butler \& Markman, 2012; Yoon et al., 2008),
and corpus analyses provide evidence that generic language (e.g.,
``birds fly'') is common in everyday adult-child conversations (Gelman,
Goetz, Sarnecka, \& Flukes, 2008).

Across all of these studies, learners interpreted similar information in
different ways depending on their assumptions about other people's
goals. These effects are different from the attentional and
informational explanations reviewed above in that the inferences based
on social information are part of the underlying computations that
support learning. This account fits well with evolutionary models that
emphasize the importance of pedagogy for the accumulation of human
cultural knowledge (Boyd et al., 2011; Kline, 2015) and theories of
cognitive development that emphasize the adult's role as providing
children with generalizable information (Csibra \& Gergely, 2009).

\section{Part II: Learning on your
own}\label{part-ii-learning-on-your-own}

Another key ingredient for children's rapid conceptual development is
their ability to learn on their own. That is, children are not just
passive recipients of information; instead, they actively seek knowledge
via their own actions. This model of the child as an ``active'' learner
has been an influential idea in many classic theories of cognitive
development (e.g., Bruner (1961); Berlyne (1960); and piaget\_cite). And
recent theorizing has characterized cognitive development as a process
of active hypothesis testing and theory revision following principles
similar to the scientific method (Gopnik et al., 1999; Schulz, 2012).

In addition to playing a prominent role in developmental theory, the
potential benefits of ``active''\footnote{The term ``active learning''
  has been used to describe a wide variety of behaviors such as question
  asking, increased physical activity, or active memory retrieval. In
  this paper, I focus on a specific subset of these behaviors: the
  \emph{decisions} that people make, or could make, during learning.
  This definition captures several ways that people can exert control
  over their learning experiences, including the selection, sequencing,
  and/or pacing of new information.} learning have been the focus of a
great deal of empirical work in education (Grabinger \& Dunlap, 1995;
Prince, 2004), machine learning (Ramirez-Loaiza, Sharma, Kumar, \&
Bilgic, 2017; Settles, 2012), and cognitive psychology (Castro et al.,
2009; Chi, 2009). The common finding across these studies is that active
learning contexts -- where people have control over some aspect of the
learning environment -- lead to better outcomes when compared to passive
contexts where people do not have control over the information that they
receive.

But what makes active control a useful way to learn about the world? In
this section, I present evidence for two mechanisms -- enhanced
attention/memory and higher quality information -- through which active
control can improve learning outcomes. I then review work that
formalizes human inquiry as a process of ``optimal experiment design''
(OED) to ask when and how human self-directed learning deviates from
optimal information gathering principles. I conclude Part II with a
discussion of what makes optimal active learning difficult and why this
is an intresting point of contact with research on children's social
learning.

\subsubsection{Active control enhances attention and
memory}\label{active-control-enhances-attention-and-memory}

A growing body of work has explored the effect of active control on
basic processes underlying learning and memory. In these tasks, outcomes
for active and passive learning experiences are directly compared across
a variety of tasks, such as episodic memory, casual learning, and
concept learning. D. B. Markant, Ruggeri, Gureckis, \& Xu (2016) review
this diverse literature and suggest that the active learning advantage
found across these domains is caused by an increase in attention and
memory with the precise pathway determined by the type of control in the
study. For example, one effect of active control is that is allows
people to coordinate the timing of incoming information with their
current cognitive state, including attention and readiness to learn.

One nice illustration of this effect comes from a study by D. Markant,
DuBrow, Davachi, \& Gureckis (2014). In this task, participants
memorized the identities and locations of objects that were hidden in a
grid (adapted from Voss et al. (2011)). D. Markant et al. (2014) varied
the \emph{level} of control across conditions and compared the
performance of active learners to a group of ``yoked'' participants who
saw training data that was generated by the active group. Across
conditions, participants could either control: (a) the next location in
the grid, (b) the next item to be revealed, (c) the duration of each
learning trial, and/or (d) the time between learning trials (i.e.,
inter-stimulus-interval or ISI). Results showed an active learning
advantage for all levels of control, including the lowest amount of
control in the ISI-only condition. D. Markant et al. (2014) interpreted
these results as providing evidence that active control allowed people
to, ``optimize their experience with respect to short-term fluctuations
in their own motivational or attentional state.''

Developmental studies have extended this work on adults' spatial memory
to 6- to 8-year old children, showing similar advantages for conditions
of active control (Ruggeri, Markant, Gureckis, \& Xu, 2016). Other work
has found similar benefits of active control in word learning
(Partridge, McGovern, Yung, \& Kidd (2015); see also Kachergis, Yu, \&
Shiffrin (2013) for evidence in adults) and understanding causal
structures (Schulz, 2012). Sobel and Kushnir (2006) showed that learners
who designed their own interven- tions on a causal system learned better
than yoked participants who either passively observed the same sequence
of actions or re-created the same choices made by others. Moreover, even
young infants seem to benefit from active engagement with the learning
environment. For example, Begus, Gliga, \& Southgate (2014) showed that
16-month-old infants show evidence of stronger memory for information
that was provided about an object they had previously pointed to as
opposed to information about an object they had previously ignored.

Additional evidence that active control enhances attention and memory
comes from research on children's engagement with educational technology
(for a review, see Hirsh-Pasek et al. (2015)). For example, Calvert,
Strong, \& Gallagher (2005) exposed preschool-aged children to two
sessions of reading a computer storybook with an adult, and manipulated
whether the adult or the child controlled the mouse and could advance
the story. Children in the adult-control condition showed a decrease in
attention to the storybook materials in the second session; in contrast,
children who were given control over the experience maintained similar
levels of attention across both sessions. Other research shows that when
adults interact with an avatar that is controlled by a real person
rather than a computer, people experience higher levels of arousal,
learn more, and pay more attention (Okita, Bailenson, \& Schwartz,
2008). And work by Roseberry et al. (2014) showed children learned
equally well from interactions with a person in a video chat (e.g.,
Skype) when social contingency was established, but they did not learn
from watching a digital interaction between the adult and another child.

These results parallel the literature on attention/memory effects in
social learning reviewed in Part 1. That is, both active and social
processes can modulate attention and memory to facilitate in-the-moment
learning. However, as in social learning, the effects of active control
operate through multiple mechanisms, going beyond changes in lower-level
cognitive processes to changing the quality of \emph{information} that
learners get from the world.

\subsubsection{\texorpdfstring{Active control provides ``good''
information}{Active control provides good information}}\label{active-control-provides-good-information}

Active learning allows people to gather information that is particularly
``useful'' for their own learning. This benefit relies on the fact that
learners have better access to their own prior knowledge, current
hypotheses, and ability level, which they can leverage to create more
helpful learning contexts (e.g., asking a question about something that
is particularly confusing). Research on this compoenent of active
learning focuses on how people take actions to create learning
experiences that are more useful compared to entirely passive contexts
where the learner has little control.

For example, Castro et al. (2009) directly compared human active and
passive category learning to predictions from statistical learning
theory under conditions of varying difficulty. They found that human
active learning was always superior to passive learning, but did not
reach the performance of the optimal model and the advantage for active
control decreased in the more difficult (i.e., noisier) learning tasks.
Using a similar model-based approach, D. B. Markant \& Gureckis (2014)
investigated the effects of active vs.~passive hypothesis testing on the
rate of adults' category learning. They varied the difficulty of the
learning task by testing two different types of category structures: a
rule-based category, which varied along 1 dimension (easier to learn),
and an information-integration category, which varied along 2 dimensions
(harder to learn). In the active condition, the learner chose specific
observations from the category to test his or her beliefs, whereas in
the passive condition, the data were generated randomly by the
experiment. Participants in the active condition learned the category
structure faster and achieved a higher overall accuracy rate compared to
participants in the passive learning condition, but only for the
simpler, rule-based category.

Together, the Castro et al. (2009) and D. B. Markant \& Gureckis (2014)
results illustrate several important points about active learning.
First, the quality of active exploration was fundamentally linked to the
learner's understanding of the task: if the representation was poor,
then self-directed learning was biased and ineffective. Second, the
benefits of active control were tied to the aspects of the individual
learner -- i.e., their prior knowledge and the current hypotheses under
consideration -- such that the same sequence of data did not provide
``good'' information for another learner. And third, the benefits of
active learning diminished with increased task difficulty, perhaps
because learners struggled to generate ``helpful'' examples.

These findings help to illusrate the complexity of human self-directed
learning. That is, it is not the case that active learning functions
similarly across individuals, contexts, and learning domains. However,
the multitude of factors that could influence active learning create a
complicated set of possibilities to explore. One useful approach to
understanding this complexity is to compare human behavior to formal
models of scientific inquiry that were developed by statisticians to
quantify the usefulness of a particular experiment.

\subsubsection{Optimal Experiment Design: A formal account of active
learning}\label{optimal-experiment-design-a-formal-account-of-active-learning}

Optimal Experiment Design (OED) (Emery \& Nenarokomov, 1998; Nelson,
2005) is a statisical framework that attempts to quantify the
``usefulness'' of a set of possible experiments relative to the
experimenter's current goal and understanding of the problem. The
benefit of using an OED approach is that it then allows the scientist to
make design choices that maximize the effectiveness of their experiment,
thus reducing the cost of additional experimentation. For example,
Nelson, McKenzie, Cottrell, \& Sejnowski (2010) used OED principles to
differentiate competing theories of information seeking during adults'
category learning. To do this, they created an OED model of the task and
found the experiments in the space of possible experiments that
maximized disagreement between the competing theories, or the
experiments that were most likely to provide a useful answer.

A growing body of psychological research has used the OED framework as a
metaphor for human active learning. The idea is that when people make
decisions about how to act on the world, they are engaging in a similar
process of evaluating the ``goodness'' of these different actions
relative to some learning goal, and in turn, select behaviors that
maximize the potential for gaining information about the world. One of
the major successes of the OED model is that it can be used to account
for a wide range of information seeking behaviors, including verbal
question asking (CITE), planning interventions in causal learning tasks
(CITE), and decisions about visual fixations during scene understanding
(Najemnik).

What are the key ingredients of an OED model? Coenen, Nelson, \&
Gureckis (2017) provide a thorough review of the OED framework and its
links to research in psychology. In their review, they lay out four key
pieces of an OED model: 1) a set of hypotheses, 2) a set of questions to
learn about the hypotheses, 3) a way to model the types of answers that
each question could elicit, and 4) a way to score each of the possible
answers with respect to change in belief about the hypotheses. They also
highlight the importance of learners' \emph{inquiry goals} (e.g.,
``What's that object called?'') for engaging in OED-like reasonsing. The
key point is that without a clear learning goal, then it becomes
difficult to instantiate the hypotheses, questions, and answers that a
learner will consider when deciding how to act.

Together, these four pieces allow an OED model to quantify the
\emph{expected utility} of each question \(EU(q)\) in the set of
questions that a person could ask \(q_1, q_2,..., q_n = \{Q\}\). This
expected value is a function of two values: 1) the probability of
obtaining a certain answer given a question \(P(a|q)\) and 2) the
usefulness of that answer for achieving the learner's goal \(U(a)\).
Then, we can define the expected utility for a specific question as the
average utility over all the possible answers to that question.

\[EU(q) = \sum_{a\in q}{P(a|q)U(a)}\] There are a variety of ways to
define the usefulness function to score each answer. An exhaustive
review is beyond the scope of this paper, but for a detailed analysis of
different approaches and their match to human behavior, see Nelson
(2005). One common approach is to use the change in uncertainty in a
specific hypothesis after receiving a particular answer or
\emph{information gain}. This can be calculated as the change in entropy
after seeing a particular answer.

\[P(h|a) = ent(h) - ent(h|a)\]

Finally, to select the best question, an OED learner must perform the
expected utility computation for each possible question. This involves
considering all of the possible answers for each question, scoring the
answers using some usefulness function, and then weighting each score by
the probability of getting that answer.

There are several benefits of the OED formalization for understanding
human active learning. First, it makes researchers define the different
components of an active learning problem, thus making their assumptions
about the phenomenon more explicit. Second, if researchers can develop
an OED model, then they can ask whether people's behavior matches or
deviates from the optimal behavior predicted by the model. Finally,
casting information seeking as rational \emph{choice} links psychology
with several rich literatures (economics, statistics, computer science)
that have attempted to formalize the decision-making process as a
process of utility analysis that can include both the costs of
information acquisition and the benefits of choosing a particular
behavior.

One nice demonstration of this approach comes from Nelson (2005) model
of eye movements during novel concept learning. The model combines
Bayesian probabilistic learning, which represents the learner's current
knowledge as a probability distribution over a concept, with an OED
model of the usefulness of a particular eye movement (modeled as a type
of question-asking behavior) for gathering additional information about
the target concept from the visual world. Together, these model
components allowed Nelson (2005) to predict changes in the pattern of
eye movements at different points in the learning task. Specifically,
they found that early in learning, when the concepts were unfamiliar,
the model predicted a wider, less efficient distribution of fixations to
all candidate features that could be used to categorize the stimulus.
However, after the model learned the target concepts, eye movement
patterns shifted, becoming more efficient and focusing on a single
stimulus dimension.

Another promising aspect of the OED models is that recent developmental
work has provided evidence that even young children appear to select
behaviors that efficiently maximize learning goals. For example, there
is evidence that children from a young age use verbal questions to
gather information from other people. In a corpus analysis of four
children's parent-child conversations, Chouinard, Harris, \& Maratsos
(2007) found that children begin asking questions early in development
(18 months) and at an impressive rate, ranging from 70-198 questions per
hour of conversation. Chouinard et al. (2007) also coded the intent of
children's questions, finding that 71\% were for the purpose of
gathering information, as opposed to attention getting or
clarifications. Other corpus analyses provide converging evidence that
question asking is a common behavior in parent-child conversations
(Davis, 1932), that children are seeking knowledge with their questions
(Bova \& Arcidiacono, 2013), and that children will persist in asking
questions if they do not receive a satisfactory explanation (Frazier,
Gelman, \& Wellman, 2009).

Experimental work has investigated the quality of children's question
asking by measuring the quality of questions in constrained
problem-solving tasks. For example, Legare, Mills, Souza, Plummer, \&
Yasskin (2013) used a modified question asking game where 4- to
6-year-old children saw 16 cards with a drawing of an animal on them.
The animals varied along several dimensions, including type, size, and
pattern on the animal. The child's task was to ask the experimenter
yes-no questions in order to figure out which animal card the
experimenter had hidden in a special box. Legare et al. (2013) coded
whether children asked \emph{constraint-seeking} questions that narrowed
the set of possible cards by increasing knowledge of a particular
dimension or dimensions (e.g., ``Is it red?''), \emph{confirmatory}
questions that provided redundant information, or \emph{ineffective}
questions that did not provide any useful information (e.g., ``Does it
have a tail?''). Results showed that all age groups asked a higher
proportion of the effective, constraint-seeking questions relative to
the other question types, and that the number of constraint-seeking
questions was correlated with children's accuracy in guessing the
identity of the card hidden in the special box. Legare et al. (2013)
interpret these results as evidence that children can use questions to
solve problems in a efficient manner. Converging evidence in support of
this interpretation comes from experimental work using this approach
finding that children prefer to direct questions to someone who is
knowledgeable compared to someone who is inaccurate or ignorant (Mills,
Legare, Bills, \& Mejias, 2010; Mills, Legare, Grant, \& Landrum, 2011),

Although the OED approach has provided a formal account of seemingly
uncontstrained information seeking, there are several ways in which it
falls short as an explanation of human self-directed learning. Coenen et
al. (2017) argue that in practice OED models make several critical
assumptions about the learner and the problem, including the
hypotheses/questions/answers under consideration, that people are
actually engaging in some kind of expected utility compuation in order
to maximize the goal of knowledge acquisition, and the cognitive
capacities of the learner to carry out these computations. For the
purpose of our argument, there are two limitations that are most
relevant for connecting social learning contexts with the OED model of
self-directed learning: (1) assumptions about the hypotheses, questions,
and answers that children take into consideration during learning and
(2) the role of resource constraints on children's cognitive capacities
that limit their ability to engage in OED-like reasoning.

In the next section, I intergrate social and active learning acounts
using expected utility and OED framework to structure the argument. I
focus on a subset of the nine open questions about human inquiry put
forth by Coenen et al. (2017). I argue that learning from others
provides one path to setting up a tractable space of hypotheses,
questions, and answers. The crux of the argument is that learning from
more knowledgable others plays an important role in providing the
fundamental building blocks that are required for children to engage in
effective OED reasoning in the moment of learning. \# Part III: How
social contexts can shape active learning

\subsubsection{Hypothesis space}\label{hypothesis-space}

\subsubsection{Generating questions}\label{generating-questions}

\subsubsection{Generating answers}\label{generating-answers}

\subsubsection{Reputation management}\label{reputation-management}

\subsubsection{Stopping rules}\label{stopping-rules}

\subsubsection{Social information
seeking}\label{social-information-seeking}

Active social learning - seek information from social targets. Models of
seeking information from social targets:

\begin{verbatim}
-  Baldwin & Moses (1998): The Ontogeny of Social Information gathering
-  Chouinard (2007): Children's questions as learning mechanism
-  Hyo's and Liz Bonawitz's work
\end{verbatim}

These studies of the benefits of active information selection connect
nicely to developmental work on children's question asking. Verbal
questions are a spontaneous behavior that occurs in everyday
interactions that could allow children to seek information that is
directly relevant to their current interests and misconceptions.
Moreover, asking a good question is complex: the child must know what
they don't know, how to ask about it, who to ask about it, and be able
to assimilate new information. However, other work has focused on the
social context in which question asking occurs.

\section{Part IV: Resource constraints in active-social
learning}\label{part-iv-resource-constraints-in-active-social-learning}

Active learning takes into account a utility structure that can include
both the costs of data acquisition and the rewards of choosing an
example (e.g., in terms of information acquisition/uncertainty reduction
relative to some longer term learning goal).

Focusing on \emph{choices} is useful since there is a rich literature
that has formalized decision-making process, which can be used to
describe behaviors made by both more knowledgeable others and by
learners. The interesting question is how costs/benefits of active
learning behaviors are altered by the social context and how reasoning
about learners as active might change the social context.

Process:

\begin{verbatim}
- analyze costs and benefits of behavior
- planning models that take into account long-term value
- decisions in the brain and in non-human primates 
\end{verbatim}

Active learning in social contexts. The presence of another agent can
change the cost/benefit structure of choices made for learning and
therefore models of self-directed learning should include this
information. In contrast, these models often view the learner as moving
back and forth between active exploration and passive reception. This
type of active learning account does not leave room for social reasoning
processes (i.e., naive utility calculus, goal reasoning) to change the
utility of active learning behaviors.

\subsubsection{Metacognition}\label{metacognition}

\subsubsection{Question asking}\label{question-asking}

\section{Conclusions and a way
forward}\label{conclusions-and-a-way-forward}

Models of self-directed learning should include information the
social-communicative context in which learning often occurs. Reasoning
about other people modulate the choices that learners make: whether it's
who to talk to, what to look at, or what questions to ask.

Models of social learning should take into account the choice behaviors
available to the learner. i.e., think about teaching as reasoning about
another person's active learning or setting up a social learning context
where the learner selects actions

\newpage

\section{References}\label{references}

\setlength{\parindent}{-0.4in} \setlength{\leftskip}{0.125in} \noindent

\hypertarget{refs}{}
\hypertarget{ref-adriaans2017prosodic}{}
Adriaans, F., \& Swingley, D. (2017). Prosodic exaggeration within
infant-directed speech: Consequences for vowel learnability. \emph{The
Journal of the Acoustical Society of America}, \emph{141}(5),
3070--3078.

\hypertarget{ref-begus2014infants}{}
Begus, K., Gliga, T., \& Southgate, V. (2014). Infants learn what they
want to learn: Responding to infant pointing leads to superior learning.

\hypertarget{ref-berlyne1960conflict}{}
Berlyne, D. E. (1960). Conflict, arousal, and curiosity.

\hypertarget{ref-bloom2002children}{}
Bloom, P. (2002). \emph{How children learn the meaning of words}. The
MIT Press.

\hypertarget{ref-bonawitz2016computational}{}
Bonawitz, E., \& Shafto, P. (2016). Computational models of development,
social influences. \emph{Current Opinion in Behavioral Sciences},
\emph{7}, 95--100.

\hypertarget{ref-bova2013investigating}{}
Bova, A., \& Arcidiacono, F. (2013). Investigating children's
why-questions: A study comparing argumentative and explanatory function.
\emph{Discourse Studies}, \emph{15}(6), 713--734.

\hypertarget{ref-boyd2011cultural}{}
Boyd, R., Richerson, P. J., \& Henrich, J. (2011). The cultural niche:
Why social learning is essential for human adaptation. \emph{Proceedings
of the National Academy of Sciences}, \emph{108}(Supplement 2),
10918--10925.

\hypertarget{ref-bruner1961act}{}
Bruner, J. S. (1961). The act of discovery. \emph{Harvard Educational
Review}.

\hypertarget{ref-butler2012preschoolers}{}
Butler, L. P., \& Markman, E. M. (2012). Preschoolers use intentional
and pedagogical cues to guide inductive inferences and exploration.
\emph{Child Development}, \emph{83}(4), 1416--1428.

\hypertarget{ref-calvert2005control}{}
Calvert, S. L., Strong, B. L., \& Gallagher, L. (2005). Control as an
engagement feature for young children's attention to and learning of
computer content. \emph{American Behavioral Scientist}, \emph{48}(5),
578--589.

\hypertarget{ref-castro2009human}{}
Castro, R. M., Kalish, C., Nowak, R., Qian, R., Rogers, T., \& Zhu, X.
(2009). Human active learning. In \emph{Advances in neural information
processing systems} (pp. 241--248).

\hypertarget{ref-chi2009active}{}
Chi, M. T. (2009). Active-constructive-interactive: A conceptual
framework for differentiating learning activities. \emph{Topics in
Cognitive Science}, \emph{1}(1), 73--105.

\hypertarget{ref-chouinard2007children}{}
Chouinard, M. M., Harris, P. L., \& Maratsos, M. P. (2007). Children's
questions: A mechanism for cognitive development. \emph{Monographs of
the Society for Research in Child Development}, i--129.

\hypertarget{ref-clark2009first}{}
Clark, E. V. (2009). \emph{First language acquisition}. Cambridge
University Press.

\hypertarget{ref-cleveland2007joint}{}
Cleveland, A., Schug, M., \& Striano, T. (2007). Joint attention and
object learning in 5-and 7-month-old infants. \emph{Infant and Child
Development}, \emph{16}(3), 295--306.

\hypertarget{ref-coenen2017asking}{}
Coenen, A., Nelson, J. D., \& Gureckis, T. (2017). Asking the right
questions about human inquiry.

\hypertarget{ref-cooper1990preference}{}
Cooper, R. P., \& Aslin, R. N. (1990). Preference for infant-directed
speech in the first month after birth. \emph{Child Development},
\emph{61}(5), 1584--1595.

\hypertarget{ref-csibra2009natural}{}
Csibra, G., \& Gergely, G. (2009). Natural pedagogy. \emph{Trends in
Cognitive Sciences}, \emph{13}(4), 148--153.

\hypertarget{ref-davis1932form}{}
Davis, E. A. (1932). The form and function of children's questions.
\emph{Child Development}, \emph{3}(1), 57--74.

\hypertarget{ref-de2003investigating}{}
De Boer, B., \& Kuhl, P. K. (2003). Investigating the role of
infant-directed speech with a computer model. \emph{Acoustics Research
Letters Online}, \emph{4}(4), 129--134.

\hypertarget{ref-decasper1987human}{}
DeCasper, A. J., Fifer, W. P., Oates, J., \& Sheldon, S. (1987). Of
human bonding: Newborns prefer their mothers' voices. \emph{Cognitive
Development in Infancy}, 111--118.

\hypertarget{ref-eaves2016infant}{}
Eaves Jr, B. S., Feldman, N. H., Griffiths, T. L., \& Shafto, P. (2016).
Infant-directed speech is consistent with teaching. \emph{Psychological
Review}, \emph{123}(6), 758.

\hypertarget{ref-emery1998optimal}{}
Emery, A., \& Nenarokomov, A. V. (1998). Optimal experiment design.
\emph{Measurement Science and Technology}, \emph{9}(6), 864.

\hypertarget{ref-farroni2002eye}{}
Farroni, T., Csibra, G., Simion, F., \& Johnson, M. H. (2002). Eye
contact detection in humans from birth. \emph{Proceedings of the
National Academy of Sciences}, \emph{99}(14), 9602--9605.

\hypertarget{ref-farroni2007direct}{}
Farroni, T., Massaccesi, S., Menon, E., \& Johnson, M. H. (2007). Direct
gaze modulates face recognition in young infants. \emph{Cognition},
\emph{102}(3), 396--404.

\hypertarget{ref-fernald1987acoustic}{}
Fernald, A., \& Kuhl, P. (1987). Acoustic determinants of infant
preference for motherese speech. \emph{Infant Behavior and Development},
\emph{10}(3), 279--293.

\hypertarget{ref-fernald1991prosody}{}
Fernald, A., \& Mazzie, C. (1991). Prosody and focus in speech to
infants and adults. \emph{Developmental Psychology}, \emph{27}(2), 209.

\hypertarget{ref-fernald1984expanded}{}
Fernald, A., \& Simon, T. (1984). Expanded intonation contours in
mothers' speech to newborns. \emph{Developmental Psychology},
\emph{20}(1), 104.

\hypertarget{ref-frank2014inferring}{}
Frank, M. C., \& Goodman, N. D. (2014). Inferring word meanings by
assuming that speakers are informative. \emph{Cognitive Psychology},
\emph{75}, 80--96.

\hypertarget{ref-frank2009using}{}
Frank, M. C., Goodman, N. D., \& Tenenbaum, J. B. (2009). Using
speakers' referential intentions to model early cross-situational word
learning. \emph{Psychological Science}, \emph{20}(5), 578--585.

\hypertarget{ref-frazier2009preschoolers}{}
Frazier, B. N., Gelman, S. A., \& Wellman, H. M. (2009). Preschoolers'
search for explanatory information within adult--child conversation.
\emph{Child Development}, \emph{80}(6), 1592--1611.

\hypertarget{ref-gelman2008generic}{}
Gelman, S. A., Goetz, P. J., Sarnecka, B. W., \& Flukes, J. (2008).
Generic language in parent-child conversations. \emph{Language Learning
and Development}, \emph{4}(1), 1--31.

\hypertarget{ref-goldstein2008social}{}
Goldstein, M. H., \& Schwade, J. A. (2008). Social feedback to infants'
babbling facilitates rapid phonological learning. \emph{Psychological
Science}, \emph{19}(5), 515--523.

\hypertarget{ref-goodman2009cause}{}
Goodman, N. D., Baker, C. L., \& Tenenbaum, J. B. (2009). Cause and
intent: Social reasoning in causal learning. In \emph{Proceedings of the
31st annual conference of the cognitive science society} (pp.
2759--2764).

\hypertarget{ref-gopnik1999scientist}{}
Gopnik, A., Meltzoff, A. N., \& Kuhl, P. K. (1999). \emph{The scientist
in the crib: Minds, brains, and how children learn.} William Morrow \&
Co.

\hypertarget{ref-grabinger1995rich}{}
Grabinger, R. S., \& Dunlap, J. C. (1995). Rich environments for active
learning: A definition. \emph{Research in Learning Technology},
\emph{3}(2).

\hypertarget{ref-graf2013infant}{}
Graf Estes, K., \& Hurley, K. (2013). Infant-directed prosody helps
infants map sounds to meanings. \emph{Infancy}, \emph{18}(5), 797--824.

\hypertarget{ref-hirsh2015putting}{}
Hirsh-Pasek, K., Zosh, J. M., Golinkoff, R. M., Gray, J. H., Robb, M.
B., \& Kaufman, J. (2015). Putting education in ``educational'' apps:
Lessons from the science of learning. \emph{Psychological Science in the
Public Interest}, \emph{16}(1), 3--34.

\hypertarget{ref-hollich2000breaking}{}
Hollich, G. J., Hirsh-Pasek, K., Golinkoff, R. M., Brand, R. J., Brown,
E., Chung, H. L., \ldots{} Bloom, L. (2000). Breaking the language
barrier: An emergentist coalition model for the origins of word
learning. \emph{Monographs of the Society for Research in Child
Development}, i--135.

\hypertarget{ref-johnson1991newborns}{}
Johnson, M. H., Dziurawiec, S., Ellis, H., \& Morton, J. (1991).
Newborns' preferential tracking of face-like stimuli and its subsequent
decline. \emph{Cognition}, \emph{40}(1), 1--19.

\hypertarget{ref-kachergis2013actively}{}
Kachergis, G., Yu, C., \& Shiffrin, R. M. (2013). Actively learning
object names across ambiguous situations. \emph{Topics in Cognitive
Science}, \emph{5}(1), 200--213.

\hypertarget{ref-kline2015learn}{}
Kline, M. A. (2015). How to learn about teaching: An evolutionary
framework for the study of teaching behavior in humans and other
animals. \emph{Behavioral and Brain Sciences}, \emph{38}.

\hypertarget{ref-kuhl2007speech}{}
Kuhl, P. K. (2007). Is speech learning `gated'by the social brain?
\emph{Developmental Science}, \emph{10}(1), 110--120.

\hypertarget{ref-kuhl2003foreign}{}
Kuhl, P. K., Tsao, F.-M., \& Liu, H.-M. (2003). Foreign-language
experience in infancy: Effects of short-term exposure and social
interaction on phonetic learning. \emph{Proceedings of the National
Academy of Sciences}, \emph{100}(15), 9096--9101.

\hypertarget{ref-legare2013use}{}
Legare, C. H., Mills, C. M., Souza, A. L., Plummer, L. E., \& Yasskin,
R. (2013). The use of questions as problem-solving strategies during
early childhood. \emph{Journal of Experimental Child Psychology},
\emph{114}(1), 63--76.

\hypertarget{ref-markant2014better}{}
Markant, D. B., \& Gureckis, T. M. (2014). Is it better to select or to
receive? Learning via active and passive hypothesis testing.
\emph{Journal of Experimental Psychology: General}, \emph{143}(1), 94.

\hypertarget{ref-markant2016enhanced}{}
Markant, D. B., Ruggeri, A., Gureckis, T. M., \& Xu, F. (2016). Enhanced
memory as a common effect of active learning. \emph{Mind, Brain, and
Education}, \emph{10}(3), 142--152.

\hypertarget{ref-markant2014deconstructing}{}
Markant, D., DuBrow, S., Davachi, L., \& Gureckis, T. M. (2014).
Deconstructing the effect of self-directed study on episodic memory.
\emph{Memory \& Cognition}, \emph{42}(8), 1211--1224.

\hypertarget{ref-mills2010preschoolers}{}
Mills, C. M., Legare, C. H., Bills, M., \& Mejias, C. (2010).
Preschoolers use questions as a tool to acquire knowledge from different
sources. \emph{Journal of Cognition and Development}, \emph{11}(4),
533--560.

\hypertarget{ref-mills2011determining}{}
Mills, C. M., Legare, C. H., Grant, M. G., \& Landrum, A. R. (2011).
Determining who to question, what to ask, and how much information to
ask for: The development of inquiry in young children. \emph{Journal of
Experimental Child Psychology}, \emph{110}(4), 539--560.

\hypertarget{ref-nelson2005finding}{}
Nelson, J. D. (2005). Finding useful questions: On bayesian
diagnosticity, probability, impact, and information gain.
\emph{Psychological Review}, \emph{112}(4).

\hypertarget{ref-nelson2010experience}{}
Nelson, J. D., McKenzie, C. R., Cottrell, G. W., \& Sejnowski, T. J.
(2010). Experience matters: Information acquisition optimizes
probability gain. \emph{Psychological Science}, \emph{21}(7), 960--969.

\hypertarget{ref-partridge2015young}{}
Partridge, E., McGovern, M. G., Yung, A., \& Kidd, C. (2015). Young
children's self-directed information gathering on touchscreens. In
\emph{Proceedings of the 37th annual conference of the cognitive science
society, austin, tx. cognitive science society}.

\hypertarget{ref-pegg1992preference}{}
Pegg, J. E., Werker, J. F., \& McLeod, P. J. (1992). Preference for
infant-directed over adult-directed speech: Evidence from 7-week-old
infants. \emph{Infant Behavior and Development}, \emph{15}(3), 325--345.

\hypertarget{ref-prince2004does}{}
Prince, M. (2004). Does active learning work? A review of the research.
\emph{Journal of Engineering Education}, \emph{93}(3), 223--231.

\hypertarget{ref-ramirez2017active}{}
Ramirez-Loaiza, M. E., Sharma, M., Kumar, G., \& Bilgic, M. (2017).
Active learning: An empirical study of common baselines. \emph{Data
Mining and Knowledge Discovery}, \emph{31}(2), 287--313.

\hypertarget{ref-reid2005adult}{}
Reid, V. M., \& Striano, T. (2005). Adult gaze influences infant
attention and object processing: Implications for cognitive
neuroscience. \emph{European Journal of Neuroscience}, \emph{21}(6),
1763--1766.

\hypertarget{ref-rogoff1993guided}{}
Rogoff, B., Mistry, J., Göncü, A., Mosier, C., Chavajay, P., \& Heath,
S. B. (1993). Guided participation in cultural activity by toddlers and
caregivers. \emph{Monographs of the Society for Research in Child
Development}, i--179.

\hypertarget{ref-ruggeri2016active}{}
Ruggeri, A., Markant, D. B., Gureckis, T. M., \& Xu, F. (2016). Active
control of study leads to improved recognition memory in children. In
\emph{Proceedings of the 38th annual conference of the cognitive science
society. austin, tx: Cognitive science society}.

\hypertarget{ref-sage2011disentangling}{}
Sage, K. D., \& Baldwin, D. (2011). Disentangling the social and the
pedagogical in infants' learning about tool-use. \emph{Social
Development}, \emph{20}(4), 825--844.

\hypertarget{ref-schulz2012origins}{}
Schulz, L. (2012). The origins of inquiry: Inductive inference and
exploration in early childhood. \emph{Trends in Cognitive Sciences},
\emph{16}(7), 382--389.

\hypertarget{ref-senju2008gaze}{}
Senju, A., \& Csibra, G. (2008). Gaze following in human infants depends
on communicative signals. \emph{Current Biology}, \emph{18}(9),
668--671.

\hypertarget{ref-settles2012active}{}
Settles, B. (2012). Active learning. \emph{Synthesis Lectures on
Artificial Intelligence and Machine Learning}, \emph{6}(1), 1--114.

\hypertarget{ref-shafto2012epistemic}{}
Shafto, P., Eaves, B., Navarro, D. J., \& Perfors, A. (2012). Epistemic
trust: Modeling children's reasoning about others' knowledge and intent.
\emph{Developmental Science}, \emph{15}(3), 436--447.

\hypertarget{ref-shafto2012learning}{}
Shafto, P., Goodman, N. D., \& Frank, M. C. (2012). Learning from others
the consequences of psychological reasoning for human learning.
\emph{Perspectives on Psychological Science}, \emph{7}(4), 341--351.

\hypertarget{ref-shafto2014rational}{}
Shafto, P., Goodman, N. D., \& Griffiths, T. L. (2014). A rational
account of pedagogical reasoning: Teaching by, and learning from,
examples. \emph{Cognitive Psychology}, \emph{71}, 55--89.

\hypertarget{ref-singh2009influences}{}
Singh, L., Nestor, S., Parikh, C., \& Yull, A. (2009). Influences of
infant-directed speech on early word recognition. \emph{Infancy},
\emph{14}(6), 654--666.

\hypertarget{ref-thiessen2005infant}{}
Thiessen, E. D., Hill, E. A., \& Saffran, J. R. (2005). Infant-directed
speech facilitates word segmentation. \emph{Infancy}, \emph{7}(1),
53--71.

\hypertarget{ref-tomasello1986joint}{}
Tomasello, M., \& Farrar, M. J. (1986). Joint attention and early
language. \emph{Child Development}, 1454--1463.

\hypertarget{ref-voss2011spontaneous}{}
Voss, J. L., Warren, D. E., Gonsalves, B. D., Federmeier, K. D., Tranel,
D., \& Cohen, N. J. (2011). Spontaneous revisitation during visual
exploration as a link among strategic behavior, learning, and the
hippocampus. \emph{Proceedings of the National Academy of Sciences},
\emph{108}(31), E402--E409.

\hypertarget{ref-vouloumanos2007listening}{}
Vouloumanos, A., \& Werker, J. F. (2007). Listening to language at
birth: Evidence for a bias for speech in neonates. \emph{Developmental
Science}, \emph{10}(2), 159--164.

\hypertarget{ref-vygotsky1987zone}{}
Vygotsky, L. (1987). Zone of proximal development. \emph{Mind in
Society: The Development of Higher Psychological Processes},
\emph{5291}, 157.

\hypertarget{ref-xu2007word}{}
Xu, F., \& Tenenbaum, J. B. (2007). Word learning as bayesian inference.
\emph{Psychological Review}, \emph{114}(2), 245.

\hypertarget{ref-yoon2008communication}{}
Yoon, J. M., Johnson, M. H., \& Csibra, G. (2008). Communication-induced
memory biases in preverbal infants. \emph{Proceedings of the National
Academy of Sciences}, \emph{105}(36), 13690--13695.

\hypertarget{ref-yu2007unified}{}
Yu, C., \& Ballard, D. H. (2007). A unified model of early word
learning: Integrating statistical and social cues.
\emph{Neurocomputing}, \emph{70}(13), 2149--2165.

\hypertarget{ref-yu2012embodied}{}
Yu, C., \& Smith, L. B. (2012). Embodied attention and word learning by
toddlers. \emph{Cognition}.

\hypertarget{ref-yu2013joint}{}
Yu, C., \& Smith, L. B. (2013). Joint attention without gaze following:
Human infants and their parents coordinate visual attention to objects
through eye-hand coordination. \emph{PloS One}, \emph{8}(11), e79659.

\hypertarget{ref-yu2016social}{}
Yu, C., \& Smith, L. B. (2016). The social origins of sustained
attention in one-year-old human infants. \emph{Current Biology},
\emph{26}(9), 1235--1240.

\hypertarget{ref-yu2005role}{}
Yu, C., Ballard, D. H., \& Aslin, R. N. (2005). The role of embodied
intention in early lexical acquisition. \emph{Cognitive Science},
\emph{29}(6), 961--1005.

\bibliography{}

\end{document}
