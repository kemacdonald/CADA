% Template for APA submission with R Markdown

% Stuff changed from PLOS Template
\documentclass[a4paper,man,apacite,floatsintext]{apa6}
\usepackage{apacite}

% amsmath package, useful for mathematical formulas
\usepackage{amsmath}
% amssymb package, useful for mathematical symbols
\usepackage{amssymb}

% hyperref package, useful for hyperlinks
\usepackage{hyperref}

% graphicx package, useful for including eps and pdf graphics
% include graphics with the command \includegraphics
\usepackage{graphicx}

% Sweave(-like)
\usepackage{fancyvrb}
\DefineVerbatimEnvironment{Sinput}{Verbatim}{fontshape=sl}
\DefineVerbatimEnvironment{Soutput}{Verbatim}{}
\DefineVerbatimEnvironment{Scode}{Verbatim}{fontshape=sl}
\newenvironment{Schunk}{}{}
\DefineVerbatimEnvironment{Code}{Verbatim}{}
\DefineVerbatimEnvironment{CodeInput}{Verbatim}{fontshape=sl}
\DefineVerbatimEnvironment{CodeOutput}{Verbatim}{}
\newenvironment{CodeChunk}{}{}

% cite package, to clean up citations in the main text. Do not remove.
\usepackage{cite}

\usepackage{color}

% Use doublespacing - comment out for single spacing
%\usepackage{setspace}
%\doublespacing


% Text layout
\topmargin 0.0cm
\oddsidemargin 0.5cm
\evensidemargin 0.5cm
\textwidth 16cm
\textheight 21cm

% Bold the 'Figure #' in the caption and separate it with a period
% Captions will be left justified
\usepackage[labelfont=bf,labelsep=period,justification=raggedright]{caption}


% Remove brackets from numbering in List of References
\makeatletter
\renewcommand{\@biblabel}[1]{\quad#1.}
\makeatother


% Leave date blank
\date{}

%\pagestyle{myheadings}
%% ** EDIT HERE **


%% ** EDIT HERE **
%% PLEASE INCLUDE ALL MACROS BELOW

%% END MACROS SECTION


% ALL OF THE TITLE PAGE INFORMATION IS SPECIFIED IN THE YAML
\title{\textbf{Children's social learning is active: how social contexts shape
learners' choices}}
\shorttitle{}

\author{Kyle MacDonald}

\affiliation{Department of Psychology, Stanford University}

\authornote{Conceptual Analysis of Dissertation Area. Readers: Michael C. Frank,
Hyowon Gweon, and Anne Fernald}
\abstract{Children's rapid conceptual development is one of the more remarkable
features of human cognition. How do they learn so much so quickly?
Social learning theories argue for the importance of learning from more
knowledgeable others. In contrast, active learning accounts focus on
children's learning via their self-directed selection and testing of
hypotheses. In this paper, I argue that an important step towards a
complete theory of human learning is to understand how active learning
behaviors unfold within fundamentally social learning contexts. To
integrate the two accounts, I use the framework of rational decision
making that emphasizes the role of utility (i.e., costs and benefits of
an action) for explaining choice behavior. The key insight is that
social learning is not separate from active learning, and the
costs/benefits of children' decisions about what to do are shaped by
interactions with other people.}
\keywords{human learning, active learning, social learning, decision making,
theory}

\begin{document}
\maketitle

\section{Introduction}\label{introduction}

Human learning is remarkable. Consider that children, despite striking
limitations in their general processing capabilities, are able to
acquire new concepts at a high rate eventually reaching an adult
vocabulary ranging between 50,000 to 100,000 words. And they do this
while also developing motor skills, learning social norms, and building
causal knowledge. What sorts of processes can account for children's
prodigious learning abilities?

Social learning theories offer a solution by pointing out that children
do not solve these problems on their own. And although they learn a
great deal from observation, children are typically surrounded by
parents, other knowledgeable adults, or older peers -- all of whom
likely know more than they do. These ``social'' learning contexts can
bootstrap learning via several mechanisms. For example, work on early
language learning shows that social partners provide input that is tuned
to the child's cognitive abilities (Eaves Jr, Feldman, Griffiths, \&
Shafto, 2016; Fernald \& Kuhl, 1987), guide children's attention to the
important features in the world (Yu \& Ballard, 2007), and increase
levels of arousal and sustained attention (P. K. Kuhl, 2007; Yu \&
Smith, 2016).

Social learning contexts can also change the computations that support
children' learning from evidence. Recent work on both concept learning
and causal learning suggest that the presence of another person engages
social reasoning processes about \emph{why} people perform certain
actions. The key insight is that knowledge of the underlying process
that generates information allows learners to make more appropriate
inferences to facilitate learning (Bonawitz \& Shafto, 2016; Frank,
Goodman, \& Tenenbaum, 2009). For example, people will draw different
inferences from the same observations depending on whether they think
that the cause accidental or intentional behavior. Moreover, adults and
children will make even stronger inferences if they think an action was
intentional and selected to help them learn (i.e., teaching) (Shafto,
Goodman, \& Frank, 2012).

However, children are not just passive recipients of information -- from
people or from the world. Instead, children can actively select
behaviors (e.g., ask questions, choose where to allocate visual
attention) that modulate the content, pacing, and sequence of
information they receive. In fact, recent theorizing and empirical work
in cognitive development conceptualize early learning as an active
process of exploration and hypothesis testing similar to the scientific
method (Gopnik, Meltzoff, \& Kuhl, 1999; Schulz, 2012). Moreover, recent
empirical work across a variety of domains (education, machine learning,
and cognitive science) has begun to explore the benefits of
self-directed choice for speeding learning outcomes compared to passive
learning contexts (Castro et al., 2009; Markant \& Gureckis, 2014;
Settles, 2012).

Thus, both social and active contexts can facilitate learning by
activating distinct learning processes and by providing the learner with
better information. But real-world learning involves a complex mixture
of these processes. Thus, one important challenge for understanding the
power of human learning is to precisely characterize the mutual
influence of social learning contexts and children's developing ability
to exert control over their environment. In this paper, I argue that
learning from social contexts can be productively construed as providing
opportunities for \emph{constrained active learning}. The key insight is
that the presence of other people can qualitatively change the
cost-benefit calculus of learners' choices, which in turn shapes both
children's input and the social context.

The plan for the paper is as follows. I first review evidence of the
effects of social learning across a variety of learning domains. I also
present three different mechanisms through which social contexts
facilitate learning: informational, attentional, and inferential. In
part II, I discuss work on active learning that explores the influence
of giving people control over the learning environment. In part III, I
integrate the social and active learning accounts using ideas from
formal models of decision-making that emphasize the utility structure of
different actions available to the learner. I conclude by presenting a
conceptual analysis that explores the utility of a variety of choice
behaviors available to young learners in real-world social learning
contexts.

\section{Part I: Learning from
others}\label{part-i-learning-from-others}

Social learning is the accumulation of knowledge based on the sampling
decisions of other agents (e.g., via the framework in Shafto et al.,
2012). Requires reasoning about why the other agent made the choices
they did.

\subsection{Why is social learning
beneficial?}\label{why-is-social-learning-beneficial}

Three mechs: 1) Better arousal/attention, 2) Better information, 3) More
appropriate inferences. Some information can only be accessed via
interactions with other people (e.g., language)

\subsection{Different models of social
learning}\label{different-models-of-social-learning}

Examples:

\begin{verbatim}
-  Sobel and Kirkham 
-  observational learning
-  imitation learning
-  pedagogical inference
-  social as attention vs. social as changing underlying inferences because of reasoning about others minds
\end{verbatim}

\subsection{What is missing from the social learning
account?}\label{what-is-missing-from-the-social-learning-account}

TODO: Still need to figure out exactly what I want to say here. But I'm
thinking it will be something like ``emphasize the teacher's model of
the learner's goals and potential decision-making processes, i.e.,
behavior selection.''

Models of seeking information from social targets:

\begin{verbatim}
-  Baldwin & Moses (1998): The Ontogeny of Social Information gathering
-  Chouinard (2007): Children's questions as learning mechanism
-  Hyo's and Liz Bonawitz's work 
\end{verbatim}

\section{Part II: Self-directed
learning}\label{part-ii-self-directed-learning}

Classic theories of development have shared the intuition that knowledge
acquisition is a fundamentally active process, with the learner playing
an important role in shaping the learning environment (Bruner, Piaget,
Vygotsky). And recent theoretical and empirical work has formalized
these ideas by characterizing development as a process of active
hypothesis testing and theory revision that can be described by
principles of Bayesian reasoning (Gopnik; Schulz, 2007).

Moreover, the potential benefits of active learning have been the focus
of empirical work from a broad set of research areas, including fields
such as education (Grabinger \& Dunlap, 1995), machine learning
(Settles, 2012), and cognitive science (Castro et al., 2009). Across
these different literatures, the term ``active learning'' has been used
to mean a variety of behaviors such as question asking (cite), increased
physical activity (cite), or active memory retrieval (cite).

In this paper, I focus on a specific subset of active learning
behaviors: the \emph{decisions} that people make, or could make, during
learning. That is, the capacity to exert control over the learning
experience, including the selection, sequencing, or pacing of new
information.

\subsection{Why is active learning
beneficial?}\label{why-is-active-learning-beneficial}

Markant et al. (2016) describe the benefits as ``enhanced memory may be
a common outcome of active learning that can arise from a number of
distinct mechanisms, depending on the kinds of control afforded by an
instructional activity''

Examples:

\begin{verbatim}
* Encoding of Distinctive Sensorimotor Associations
* Elaborative Encoding Through Goal-Directed Search and Planning
* Co-ordination of Selective Attention and Memory Encoding
* Adaptive Selection of Material
* Enhanced Memory Due to Metacognitive Monitoring 
\end{verbatim}

Focusing on \emph{choices} is useful since there is a rich literature
that has formalized decision-making process, which can be used to
describe behaviors made by both more knowledgeable others and by
learners. The interesting question is how costs/benefits of active
learning behaviors are altered by the social context and how reasoning
about learners as active might change the social context.

\subsection{What is missing from the active learning
account?}\label{what-is-missing-from-the-active-learning-account}

The presence of another agent can change the cost/benefit structure of
choices made for learning and therefore we must include this information
in our models of self-directed learning, which often view the learner as
moving back and forth between active exploration and passive reception.
This type of active learning account does not leave room for social
reasoning processes (i.e., native utility calculus, goal reasoning) to
change the utility of active learning behaviors.

\section{Part III: Integrating social and active learning via rational
decision
making}\label{part-iii-integrating-social-and-active-learning-via-rational-decision-making}

Active learning takes into account a utility structure that can include
both the costs of data acquisition and the rewards of choosing an
example (e.g., in terms of information acquisition/uncertainty reduction
relative to some longer term learning goal).

Process:

\begin{verbatim}
- analyze costs and benefits of behavior
- planning models that take into account long-term value
- decisions in the brain and in non-human primates 
\end{verbatim}

\section{Conclusion}\label{conclusion}

Models of self-directed learning should include information the
social-communicative context in which learning often occurs. Reasoning
about other people modulate the choices that learners make: whether it's
who to talk to, what to look at, or what questions to ask.

Models of social learning should take into account the choice behaviors
available to the learner. i.e., think about teaching as reasoning about
another person's active learning or setting up a social learning context
where the learner selects actions

\newpage

\section{References}\label{references}

\setlength{\parindent}{-0.4in} \setlength{\leftskip}{0.125in} \noindent

\hypertarget{refs}{}
\hypertarget{ref-bonawitz2016computational}{}
Bonawitz, E., \& Shafto, P. (2016). Computational models of development,
social influences. \emph{Current Opinion in Behavioral Sciences},
\emph{7}, 95--100.

\hypertarget{ref-castro2009human}{}
Castro, R. M., Kalish, C., Nowak, R., Qian, R., Rogers, T., \& Zhu, X.
(2009). Human active learning. In \emph{Advances in neural information
processing systems} (pp. 241--248).

\hypertarget{ref-eaves2016infant}{}
Eaves Jr, B. S., Feldman, N. H., Griffiths, T. L., \& Shafto, P. (2016).
Infant-directed speech is consistent with teaching. \emph{Psychological
Review}, \emph{123}(6), 758.

\hypertarget{ref-fernald1987acoustic}{}
Fernald, A., \& Kuhl, P. (1987). Acoustic determinants of infant
preference for motherese speech. \emph{Infant Behavior and Development},
\emph{10}(3), 279--293.

\hypertarget{ref-frank2009using}{}
Frank, M. C., Goodman, N. D., \& Tenenbaum, J. B. (2009). Using
speakers' referential intentions to model early cross-situational word
learning. \emph{Psychological Science}, \emph{20}(5), 578--585.

\hypertarget{ref-gopnik1999scientist}{}
Gopnik, A., Meltzoff, A. N., \& Kuhl, P. K. (1999). \emph{The scientist
in the crib: Minds, brains, and how children learn.} William Morrow \&
Co.

\hypertarget{ref-grabinger1995rich}{}
Grabinger, R. S., \& Dunlap, J. C. (1995). Rich environments for active
learning: A definition. \emph{Research in Learning Technology},
\emph{3}(2).

\hypertarget{ref-kuhl2007speech}{}
Kuhl, P. K. (2007). Is speech learning `gated'by the social brain?
\emph{Developmental Science}, \emph{10}(1), 110--120.

\hypertarget{ref-markant2014better}{}
Markant, D. B., \& Gureckis, T. M. (2014). Is it better to select or to
receive? Learning via active and passive hypothesis testing.
\emph{Journal of Experimental Psychology: General}, \emph{143}(1), 94.

\hypertarget{ref-schulz2012origins}{}
Schulz, L. (2012). The origins of inquiry: Inductive inference and
exploration in early childhood. \emph{Trends in Cognitive Sciences},
\emph{16}(7), 382--389.

\hypertarget{ref-settles2012active}{}
Settles, B. (2012). Active learning. \emph{Synthesis Lectures on
Artificial Intelligence and Machine Learning}, \emph{6}(1), 1--114.

\hypertarget{ref-shafto2012learning}{}
Shafto, P., Goodman, N. D., \& Frank, M. C. (2012). Learning from others
the consequences of psychological reasoning for human learning.
\emph{Perspectives on Psychological Science}, \emph{7}(4), 341--351.

\hypertarget{ref-yu2007unified}{}
Yu, C., \& Ballard, D. H. (2007). A unified model of early word
learning: Integrating statistical and social cues.
\emph{Neurocomputing}, \emph{70}(13), 2149--2165.

\hypertarget{ref-yu2016social}{}
Yu, C., \& Smith, L. B. (2016). The social origins of sustained
attention in one-year-old human infants. \emph{Current Biology},
\emph{26}(9), 1235--1240.

\bibliography{}

\end{document}
