% Template for APA submission with R Markdown

% Stuff changed from PLOS Template
\documentclass[a4paper,man,apacite,floatsintext]{apa6}
\usepackage{apacite}

% amsmath package, useful for mathematical formulas
\usepackage{amsmath}
% amssymb package, useful for mathematical symbols
\usepackage{amssymb}

% hyperref package, useful for hyperlinks
\usepackage{hyperref}

% graphicx package, useful for including eps and pdf graphics
% include graphics with the command \includegraphics
\usepackage{graphicx}

% Sweave(-like)
\usepackage{fancyvrb}
\DefineVerbatimEnvironment{Sinput}{Verbatim}{fontshape=sl}
\DefineVerbatimEnvironment{Soutput}{Verbatim}{}
\DefineVerbatimEnvironment{Scode}{Verbatim}{fontshape=sl}
\newenvironment{Schunk}{}{}
\DefineVerbatimEnvironment{Code}{Verbatim}{}
\DefineVerbatimEnvironment{CodeInput}{Verbatim}{fontshape=sl}
\DefineVerbatimEnvironment{CodeOutput}{Verbatim}{}
\newenvironment{CodeChunk}{}{}

% cite package, to clean up citations in the main text. Do not remove.
\usepackage{cite}

\usepackage{color}

% Use doublespacing - comment out for single spacing
%\usepackage{setspace}
%\doublespacing


% Text layout
\topmargin 0.0cm
\oddsidemargin 0.5cm
\evensidemargin 0.5cm
\textwidth 16cm
\textheight 21cm

% Bold the 'Figure #' in the caption and separate it with a period
% Captions will be left justified
\usepackage[labelfont=bf,labelsep=period,justification=raggedright]{caption}


% Remove brackets from numbering in List of References
\makeatletter
\renewcommand{\@biblabel}[1]{\quad#1.}
\makeatother


% Leave date blank
\date{}

%\pagestyle{myheadings}
%% ** EDIT HERE **


%% ** EDIT HERE **
%% PLEASE INCLUDE ALL MACROS BELOW

%% END MACROS SECTION


% ALL OF THE TITLE PAGE INFORMATION IS SPECIFIED IN THE YAML
\title{\textbf{Active learning is social and social learning is active: a case study of
learning words from conversation}}
\shorttitle{Active learning is social learning}

\author{Kyle MacDonald}

\affiliation{Department of Psychology, Stanford University}

\authornote{Readers: Michael C. Frank, Hyowon Gweon, and Anne Fernald}
\abstract{}
\keywords{}

\begin{document}
\maketitle

\section{Introduction}\label{introduction}

\subsection{Active learning is social}\label{active-learning-is-social}

Hypothesis: early active learning behaviors often occur within
communicative/social/pedagogical contexts.

There's an interesting distinction between active learning behaviors
that directly affect social targets (or are directed towards social
targets) such as questions vs.~active learning behaviors that might be
changed by the presence of a social partner or being in a communicative
context such as gaze patterns to the visual world compared to gaze
patterns directed towards a social partner to gather information.

\subsection{Social learning is active}\label{social-learning-is-active}

Hypothesis: Social interactions that yield information can productively
be construed as active learning.

This analysis explains a range of social behaviors from the micro
(eye-movements) to the macro (decisions about conversational partners).

\subsubsection{{[}Some interesting motivating
example{]}}\label{some-interesting-motivating-example}

\subsubsection{What are the puzzles of self-directed
learning?}\label{what-are-the-puzzles-of-self-directed-learning}

Much of real-world learning is driven by people's choices of what to
learn. in contexts that contain \emph{both} active and passive input.
People rarely learn a new concept entirely from information generated by
themselves (active learning\footnote{Here we focus on deliberate
  decisions about what to learn, as opposed to other uses of the term
  ``active'' learning (e.g., being engaged with learning materials).})
or entirely from information received from the world (passive learning).
And yet we do not have a theory about whether different sequences of
active/passive input are better for different kinds of learning
problems. Consider a teacher introducing a challenging math concept:
should she allow students to explore first and then provide instruction,
or should she teach first and then let students actively explore?

The potential benefits of active learning have been the focus of much
research in education ({\textbf{???}}), machine learning
({\textbf{???}}), and cognitive science ({\textbf{???}}).

In a review of this diverse literature, ({\textbf{???}}) suggest that
active learning can be superior to passive learning because it allows
people to use their prior experience and current hypotheses to select
the most helpful examples (e.g., asking a question about something that
is particularly confusing). But is active learning always better than
passive learning?

Contexts for rationality is important to distinguish: * rational utility
maximizing agents in economics * KandT on higher level cognition are
challenges --\textgreater{} behavioral economics, psychological theories
of uncertainty * perception literature --\textgreater{} rational
models/analysis (perception under noise) --\textgreater{} David Marr:
model of the demands of perception in the ecological context * bayesian
perception: optimal relevant to environmental demands

Jay --\textgreater{} good person to ask about overview of Noah
--\textgreater{} planning model Berkeley folks --\textgreater{} Falk,
Jess, Stephan on resource-rational cognitive models {[}reach out to
Todd{]}

\subsubsection{Resource-constrained active
learning}\label{resource-constrained-active-learning}

\subsubsection{Context-sensitive active
learning}\label{context-sensitive-active-learning}

\subsubsection{Active learning during
conversation}\label{active-learning-during-conversation}

\section{Part 1: Active learning}\label{part-1-active-learning}

\subsection{What is active learning? And why do we
care?}\label{what-is-active-learning-and-why-do-we-care}

\subsection{What's missing from current developmental models of active
learning?}\label{whats-missing-from-current-developmental-models-of-active-learning}

\section{Part 2: Resource-rational models of
cognition}\label{part-2-resource-rational-models-of-cognition}

\subsection{Rational choice theory}\label{rational-choice-theory}

\subsection{Overview of bounded
rationality}\label{overview-of-bounded-rationality}

\subsection{Bridging levels of
analysis}\label{bridging-levels-of-analysis}

\subsection{Cost-sensitive AI}\label{cost-sensitive-ai}

\subsection{Cost-sensitive human
learning}\label{cost-sensitive-human-learning}

\subsection{Overview of ecological
rationality}\label{overview-of-ecological-rationality}

\subsection{Review of ecological rational
models}\label{review-of-ecological-rational-models}

\section{Part 3: Active learning during
conversation}\label{part-3-active-learning-during-conversation}

\subsection{Why should we care about the conversational
context?}\label{why-should-we-care-about-the-conversational-context}

\subsection{What does a converstation look
like?}\label{what-does-a-converstation-look-like}

\subsection{What behaviors are
available?}\label{what-behaviors-are-available}

\subsubsection{Behavior 1: speaker
choice}\label{behavior-1-speaker-choice}

\paragraph{Utility analysis}\label{utility-analysis}

\paragraph{Ecological opportunity}\label{ecological-opportunity}

\subsubsection{Behavior 2: attention (integrating between success and
failures)}\label{behavior-2-attention-integrating-between-success-and-failures}

\paragraph{Utility analysis}\label{utility-analysis-1}

\paragraph{Ecological opportunity}\label{ecological-opportunity-1}

\subsubsection{Behavior 3: questions (why didn't questions get asked in
this setting?) --\textgreater{} Markman et
al.}\label{behavior-3-questions-why-didnt-questions-get-asked-in-this-setting-markman-et-al.}

\paragraph{Utility analysis}\label{utility-analysis-2}

\paragraph{Ecological opportunity}\label{ecological-opportunity-2}

\section{Looking ahead}\label{looking-ahead}

\newpage

\section{References}\label{references}

\setlength{\parindent}{-0.1in} \setlength{\leftskip}{0.125in} \noindent

\bibliography{library.bib}

\end{document}
