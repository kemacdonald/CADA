% Template for APA submission with R Markdown

% Stuff changed from PLOS Template
\documentclass[a4paper,man,apacite,floatsintext]{apa6}
\usepackage{apacite}

% amsmath package, useful for mathematical formulas
\usepackage{amsmath}
% amssymb package, useful for mathematical symbols
\usepackage{amssymb}

% hyperref package, useful for hyperlinks
\usepackage{hyperref}

% graphicx package, useful for including eps and pdf graphics
% include graphics with the command \includegraphics
\usepackage{graphicx}

% Sweave(-like)
\usepackage{fancyvrb}
\DefineVerbatimEnvironment{Sinput}{Verbatim}{fontshape=sl}
\DefineVerbatimEnvironment{Soutput}{Verbatim}{}
\DefineVerbatimEnvironment{Scode}{Verbatim}{fontshape=sl}
\newenvironment{Schunk}{}{}
\DefineVerbatimEnvironment{Code}{Verbatim}{}
\DefineVerbatimEnvironment{CodeInput}{Verbatim}{fontshape=sl}
\DefineVerbatimEnvironment{CodeOutput}{Verbatim}{}
\newenvironment{CodeChunk}{}{}

% cite package, to clean up citations in the main text. Do not remove.
\usepackage{cite}

\usepackage{color}

% Use doublespacing - comment out for single spacing
%\usepackage{setspace}
%\doublespacing


% Text layout
\topmargin 0.0cm
\oddsidemargin 0.5cm
\evensidemargin 0.5cm
\textwidth 16cm
\textheight 21cm

% Bold the 'Figure #' in the caption and separate it with a period
% Captions will be left justified
\usepackage[labelfont=bf,labelsep=period,justification=raggedright]{caption}


% Remove brackets from numbering in List of References
\makeatletter
\renewcommand{\@biblabel}[1]{\quad#1.}
\makeatother


% Leave date blank
\date{}

%\pagestyle{myheadings}
%% ** EDIT HERE **


%% ** EDIT HERE **
%% PLEASE INCLUDE ALL MACROS BELOW

%% END MACROS SECTION


% ALL OF THE TITLE PAGE INFORMATION IS SPECIFIED IN THE YAML
\title{\textbf{Children's active learning is social: how social contexts shape
learners' choices}}
\shorttitle{Active learning is social}

\author{Kyle MacDonald}

\affiliation{Department of Psychology, Stanford University}

\authornote{Conceptual Analysis of Dissertation Area. Readers: Michael C. Frank,
Hyowon Gweon, and Anne Fernald}
\abstract{Children's rapid conceptual development is one of the more remarkable
features of human cognition. How do they learn so much so quickly?
Social learning theories argue for the importance of learning from more
knowledgeable others. In contrast, active learning accounts focus on
children's learning via exploration and the testing of hypotheses. In
this paper, I argue that an important step towards a complete theory of
human learning is to understand how active learning behaviors unfold
within fundamentally social learning contexts. To integrate the two
accounts, I use a framework of rational decision making that emphasizes
the role of utility (i.e., costs and benefits of an action) for
explaining choice behavior. The key insight is that social learning is
not separate from active learning, and the costs/benefits of children'
decisions about what to learn are shaped by interactions with other
people.}
\keywords{human learning, active learning, social learning, decision making,
theory}

\begin{document}
\maketitle

\section{Introduction}\label{introduction}

Human learning is remarkable. Consider that children, despite striking
limitations in their general processing capabilities, are able to
acquire new concepts at a high rate eventually reaching an adult
vocabulary ranging between 50,000 to 100,000 words (P. Bloom, 2002). And
they accmplish this while also developing motor skills, learning social
norms, and building causal knowledge. What sorts of processes can
account for children's prodigious learning abilities?

Social learning theories offer a solution by pointing out that children
do not solve these problems on their own. And although children learn a
great deal from observation, they are typically surrounded by parents,
other knowledgeable adults, or older peers -- all of whom likely know
more than they do. These social contexts can bootstrap learning via
several mechanisms. For example, work on early language acquisition
shows that social partners provide input that is tuned to children's
cognitive abilities (Eaves Jr, Feldman, Griffiths, \& Shafto, 2016;
Fernald \& Kuhl, 1987), that guides children's attention to important
features in the world (Yu \& Ballard, 2007), and that increases levels
of arousal and sustained attention, which lead to better learning (P. K.
Kuhl, 2007; Yu \& Smith, 2016).

Social contexts can also change the computations that support children'
learning from evidence. Recent work on both concept learning and causal
intervention suggests that the presence of another person leads the
learner to reason about \emph{why} people perform certain actions. The
key insight is that knowledge of the underlying process that generates
examples allows learners to make more appropriate inferences that speed
learning (Bonawitz \& Shafto, 2016; Frank, Goodman, \& Tenenbaum, 2009;
Shafto, Goodman, \& Griffiths, 2014). For example, people will draw
different inferences from observing the same actions depending on
whether they think that the behavior was accidental or intentional.
Moreover, adults and children will make even stronger inferences if they
think an action was selected with the goal to help them learn (i.e.,
teaching) (Shafto, Goodman, \& Frank, 2012).

However, children are not passive recipients of information -- from
people or from the world. Instead, children actively select behaviors
(e.g., ask questions, choose where to allocate attention) that modulate
the content, pacing, and sequence of information they receive. In fact,
recent theorizing and empirical work in cognitive development
characterizes early learning as an active process of exploration and
hypothesis testing similar to the scientific method (Gopnik, Meltzoff,
\& Kuhl, 1999; Schulz, 2012). Moreover, recent empirical work across a
variety of domains (education, machine learning, and cognitive science)
has begun to explore the benefits of self-directed choice for speeding
learning outcomes via an increase in attention/arousal or by providing
better information that is more tightly linked to learners' current
cognitive state and interests (Castro et al., 2009; D. B. Markant \&
Gureckis, 2014; Settles, 2012).

Both social and active contexts facilitate cognitive development by
activating distinct learning processes and by providing the learner with
better information. However, real-world learning is not neatly divided
into active and social contexts, but instead involves a mixture of these
processes that unfold in dynamic environments. Thus, one of the
fundamental challenge for understanding human learning is to precisely
characterize the interaction between social learning contexts and
children's developing ability to exert control over their environments.

In this paper, I propose that the framework of Bayesian rational
decision making (CITE) is a productive way to integrate ideas from both
the active and social learning accounts. The key insight is that social
learning contexts can be construed as opportunities for
\emph{constrained active learning} where both learners and teachers make
decisions about what to do whil also reasoning about the costs and
benefits of those actions for themselves and for their social partner. I
argue that considering how the presence of other people changes the
cost-benefit calculus of choosing (costly) active learning processes
provides a way to explain a diverse set of findings on children's
uncertainty monitoring (markman; kim et al.), information seeking
(begus), and question asking (katz et al. (2010). However, before
presenting the integrative framework of active learning within social
contexts, it is useful to review evidence showing how social and active
processes can impact learning outcomes.

\section{Part I: Learning from other
people}\label{part-i-learning-from-other-people}

Social learning theories argue that a key factor in children's rapid
conceptual development is humans' unique capacity to transmit and
acquire information from other people.\footnote{In this paper, I define
  ``social'' contexts as learning environments where another agent is
  present. This definition includes a broad range of social learning
  behaviors: e.g., observation, imitation, and learning from direct
  pedagogy.} One of the primary benefits of this cultural learning
process is that learners gain access to knowledge that has accumulated
over many generations: information that would be far too complex for any
individual to figure out on their own (Boyd, Richerson, \& Henrich,
2011). Moreover, social contexts can facilitate learning because more
knowledgable others can select input to best support children's learning
(Kline, 2015; Shafto et al., 2012), providing learning opportunities for
generalizable information (Csibra \& Gergely, 2009).

A large body of empirical work on children's learning shows the effect
of social contexts across a variety of learning domains. However, these
effects manifest via different pathways such as guiding attention,
increasing arousal, selecting better information, and changing the
strength of children's inferences. In this section, I review the
evidence for the role of each of these social learning processes, with
the goal of providing a high-level taxonomy of social learning effects.

\subsubsection{Social interactions enhance
attention}\label{social-interactions-enhance-attention}

From infancy, humans preferentially attend to social information. For
example, newborn infants prefer to look at face-like patterns compared
to other configurations (Johnson, Dziurawiec, Ellis, \& Morton, 1991)
and will even show a preference for faces that make direct eye contact
compared to faces with averted gaze (Farroni, Csibra, Simion, \&
Johnson, 2002). In the auditory domain, newborns prefer to listen to
speech over non-speech (Vouloumanos \& Werker, 2007), their mother's
voice over other voices (DeCasper, Fifer, Oates, \& Sheldon, 1987), and
infant-directed speech over adult-directed speech (Cooper \& Aslin,
1990; Fernald \& Kuhl, 1987; Pegg, Werker, \& McLeod, 1992). Moreover,
recent work by Yu \& Smith (2016) used head-mounted eye trackers to
record parent-child interactions and found that one-year-olds will
sustain visual attention to an object longer when their parents' had
previously looked at that object.

These attentional biases lead to different outcomes in social learning
contexts. For example, 4-month-olds' show better memory for faces if
that face gazed directly at them as compared to a face with averted gaze
(Farroni, Massaccesi, Menon, \& Johnson, 2007). 4-month-olds' memory for
objects is enhanced if an adult gazed at that object during learning
(Cleveland, Schug, \& Striano, 2007; Reid \& Striano, 2005). And
7-month-olds more easily segment words from infant-directed speech
compared to adult-directed speech (Thiessen, Hill, \& Saffran, 2005).

P. K. Kuhl (2007) refer to these effects as ``social gating'' phenomena
since the presence of another person activates or enhances children's
computational learning mechanisms. One particularly striking piece of
evidence for the social gating hypothesis comes from P. K. Kuhl, Tsao,
\& Liu (2003)`s study of infants' foreign-language phonetic learning. In
this experiment, 9-10 month-old English-learning infants listened to
Mandarin speakers either via live interactions or via audiovisual
recordings and their ability to discriminate Mandarin-specific phonemes
was assessed two months later. Only the infants who were exposed to
Mandarin within social interactions were able to succeed on the phonetic
discrimination task and infants in the audiovisual recording condition
showed no evidence of learning. P. K. Kuhl et al. (2003) also provided
evidence that infants in the social interaction condition showed higher
rates of visual attention to the speaker, suggesting that the social
contexts enhanced learning by increasing children's attention to the
input.

The common thread across these findings is that the presence of another
person is a particularly good way to increase attention. In this model,
social input becomes more salient and therefore more likely to come into
contact with children's general learning mechanisms. However, increases
in arousal, attention, and memory are only one way that social contexts
can influence learning. In fact, one of the defining features of early
learning environments is the presence of other people who know more than
the child. This creates opportunities for more knowledgable others to
select learning experiences that are particularly beneficial -- either
because the information is tuned to children's current cognitive
abilities or because the information is likely to be generalizable.

\subsubsection{\texorpdfstring{Social interactions provide ``good''
information}{Social interactions provide good information}}\label{social-interactions-provide-good-information}

The notion that children's input might be shaped to facilitate their
learning is a key tenet of several influential theories of cognitive
development (e.g., Zone of Proximal Development (Vygotsky, 1987), Guided
Participation (Rogoff et al., 1993), and Natural Pedagogy (Csibra \&
Gergely, 2009)). But how do social interactions provide particularly
useful information for children's learning?

A particularly compelling set of evidence comes from studies of how
caregivers alter the way they communicate with young children. That is,
adults do not speak to children in the same way as they speak to other
adults; instead, they exaggerate prosody, reduce speed, shorten
utterances, and elevate both pitch and affect (for a review, see
(Fernald \& Simon, 1984)). And subsequent empirical work has shown that
these features of ``infant-directed speech'' facilitate vowel learning
(Adriaans \& Swingley, 2017; De Boer \& Kuhl, 2003), word segmentation
(Fernald \& Mazzie, 1991; Thiessen et al., 2005), word recognition
(Singh, Nestor, Parikh, \& Yull, 2009), and word learning (Graf Estes \&
Hurley, 2013).

Work on infants' early vocal production also provides evidence for the
importance social feedback, highlighting the feature of
\emph{contingency}. For example, Goldstein \& Schwade (2008) measured
whether infants modified their babbling to produce more speechlike
sounds after interacting with caregivers who either provided contingent
or non-contingent responses to infants' babbling. They found that only
infants in the contingent feedback condition changed their vocalization
behavior to produce more adult-like language forms. Goldstein \& Schwade
(2008) hypothesized that the contingency effect was driven by infants'
receiving input that was particularly useful for solving this learning
problem since the feedback was close in time to infants' vocalizations,
making it easier for them to compare discrepancies between the two.

A third piece of evidence comes from research on children's early word
learning. Social-pragmatic theories of language acquisition have
emphasized the importance of social cues for reducing the (in principle)
unlimited amount of referential uncertainty present when children are
trying to acquire novel words (P. Bloom, 2002; Clark, 2009; Hollich et
al., 2000). Empirical work by Yu \& Smith (2012) shows that young
learners tend to retain words that are accompanied by clear referential
cues, which serve to make a single object dominant in the visual field
(see also (Yu \& Smith, 2013; Yu, Ballard, \& Aslin, 2005). Moreover,
correlational studies show positive links between early vocabulary
development and parents' tendency to refer to objects that children are
already attending to (i.e., ``follow-in'' labeling) (Tomasello \&
Farrar, 1986).

Thus far, I have reviewed evidence showing that social information can
benefit learning because it (a) enhances attention and it contains
features that make it easier to learn. Learning from other people also
changes learning by engaging distinct social reasoning processes that
change how learners interpret and learn from evidence.

\subsubsection{Social interactions change inferences and
generalization}\label{social-interactions-change-inferences-and-generalization}

Perhaps one of the defining features of human social learning is that
teachers and learners' actions are not random. Instead, people select
behaviors with respect to some goal (e.g., to communicate a concept),
and learners reason about \emph{why} someone chose to perform a
particular action. The key point is that this reciprocal process of
reasoning about others' goal-directed actions can change how people
interpret superficially similar behaviors, altering the learning
process.

In recent empirical and modeling work, Shafto et al. (2012) formalized
this social reasoning process within the framework of Bayesian models of
cognition. In these models, learning is a process of belief updating
that depends on two factors: what the learner believed before seeing the
data and what the learner thinks about the process that generated the
data. The key insight is that if the learner assumes that information is
generated with the intention to communicate/teach, then they can make
``stronger'' inferences.\footnote{Formally, these models change the form
  of the likelihood term in Bayes theorem in order to capture a person's
  theory of how data are generated.}

For example, Goodman, Baker, \& Tenenbaum (2009) presented adults with
causal learning scenarios with the following structure: either the
participant or someone else who knows the causal structure generates an
effect (e.g., growing flowers) by performing two actions at the same
time (e.g., pouring a yellow liquid and a blue liquid). The
participant's task was to identify the correct causal structure. Results
showed that when participants thought the other person was knowledgable,
they were more likely to infer that performing \emph{both} actions was
necessary. In contrast, when the participant performed the action on
their own (and did not know the causal structure), adults were less sure
that both actions were necessary. Shafto et al. (2012) interpreted these
results as learners going through a psychological reasoning process such
as ``if the other person was knowledgeable and wanted to generate the
effect, he would definitely perform both actions if that was the correct
causal structure.''

Similar psychological reasoning effects have been shown in the domains
of word learning (Frank \& Goodman, 2014; Xu \& Tenenbaum, 2007),
selective trust in testimony (Shafto, Eaves, Navarro, \& Perfors, 2012),
tool use (Sage \& Baldwin, 2011), and concept learning (Shafto et al.,
2014). Moreover, there is evidence that even very young learners are
sensitive to the presence of others' goal-directed behaviors. For
example, Yoon, Johnson, \& Csibra (2008) showed that 8-month-olds will
encode an object's identity if their attention was directed by a
communicative point, but they will encode an object's spatial location
if their attention was directed by non-communicative reach. And Senju \&
Csibra (2008) found that infants will follow another person's gaze only
if the gaze event was preceded by the person providing a relevant,
communicative cue (e.g., infant-directed speech or direct eye contact).

In addition to being easier to learn, information acquired in social
contexts is also more likely to generalize and be useful beyond the
current learning context. Csibra \& Gergely (2009) argue that this
assumption of \emph{generalizability} is a fundamental component of
``Natural Pedagogy'' -- a uniquely human communication system that
allows adults to efficiently pass along cultural knowledge to children.
Experimental work testing predictions from this account shows that
children are biased to think that information presented in communicative
contexts is generalizable (Butler \& Markman, 2012; Yoon et al., 2008),
and corpus analyses provide evidence that generic language (e.g.,
``birds fly'') is common in everyday adult-child conversations (Gelman,
Goetz, Sarnecka, \& Flukes, 2008).

For example, {[}natural pedagogy study here{]}

Across all of these studies, learners interpreted similar information in
very different ways depending on their assumptions about other people's
goals. These effects are different from the attentional and
informational explanations reviewed above in that the inferences based
on social information are part of the underlying computations that
support learning. This account fits well with evolutionary models that
emphasize the importance of pedagogy for the accumulation of human
cultural knowledge (Boyd et al., 2011; Kline, 2015) and theories of
cognitive development that emphasize the adult's role as providing
children with generalizable information (Csibra \& Gergely, 2009).

\section{Part II: Learning on your
own}\label{part-ii-learning-on-your-own}

A great deal of early learning occurs within social contexts where other
people shape the learning experience. But children are not just passive
recipients of information from the world; instead, they actively seek
information via their own actions and choices. This model of the child
as an ``active'' learner has been an influential idea in many classic
theories of cognitive development (e.g., Bruner (1961); Berlyne (1960);
and piaget\_cite). Moreover, recent theorizing has characterized
cognitive development as a process of active hypothesis testing and
theory revision similar to principles of the scientific method (Gopnik
et al., 1999; Schulz, 2012).

In addition to playing a prominent role in developmental theory, the
potential benefits of ``active''\footnote{The term ``active learning''
  has been used to describe a wide variety of behaviors such as question
  asking, increased physical activity, or active memory retrieval. In
  this paper, I focus on a specific subset of these behaviors: the
  \emph{decisions} that people make, or could make, during learning.
  This definition captures several ways that people can exert control
  over their learning experiences, including the selection, sequencing,
  and/or pacing of new information.} learning have been the focus of a
great deal of empirical work in education (Grabinger \& Dunlap, 1995;
Prince, 2004), machine learning (Ramirez-Loaiza, Sharma, Kumar, \&
Bilgic, 2017; Settles, 2012), and cognitive psychology (Castro et al.,
2009; Chi, 2009). The common finding across these studies is that active
learning contexts -- where people have control over some aspect of the
envrionment -- lead to better learning outcomes when compared to passive
contexts where people do not have control over the information that they
receive from the world.

But what makes active control such a powerful way to learn about the
world? In this section, I present evidence that active control changes
learning in both adults and children, leading to an advantage over
passive learning in some contexts. I also review two mechanisms through
which active control can affect learning outcomes.

\subsubsection{Active control enhances attention and
memory}\label{active-control-enhances-attention-and-memory}

There is a growing body of work directly comparing active and passive
learning across a variety of learning tasks. In a review of this
literature, D. B. Markant, Ruggeri, Gureckis, \& Xu (2016) suggest that
enhanced memory provides a unified account of these active learning
advantages. They also point out that active learning enhances memory in
different ways depending on the type of control. The first mechanism is
that active control allows people to coordinate incoming information
with their current cognitive state, including attention and motivation
to learn.

One particularly nice illustration of this approach comes from a study
by D. Markant, DuBrow, Davachi, \& Gureckis (2014). In their experiment,
participants memorized the identities and locations of objects that were
hidden in a grid (adapted from Voss et al. (2011)). D. Markant et al.
(2014) varied the \emph{level} of control across conditions and compared
the performance of active learners to a group of ``yoked'' participants
who saw training data that was generated by the active group. Across
conditions, participants could either control: (a) the next location in
the grid, (b) the next item to be revealed, (c) the duration of each
learning trial, and/or (d) the time between learning trials (i.e.,
inter-stimulus-interval or ISI). Results showed an active learning
advantage for all levels of control, including the lowest amount of
control in the ISI-only condition. D. Markant et al. (2014) interpreted
these results as providing evidence that active control allowed people
to, ``optimize their experience with respect to short-term fluctuations
in their own motivational or attentional state.''

Recent developmental work has extended the active learning advantage for
spatial memory to 6- to 8-year old children (Ruggeri, Markant, Gureckis,
\& Xu, 2016). And other work has found similar benefits of active
control in word learning (Partridge, McGovern, Yung, \& Kidd (2015); see
also Kachergis, Yu, \& Shiffrin (2013) for evidence in adults) and
understanding causal structures (Schulz, 2012). Moreover, young infants
benefit from active engagement with the environment. Begus, Gliga, \&
Southgate (2014) showed that 16-month-old infants learn better when
information was provided in response to their active pointing behavior.

Additional evidence that active control enhances attention comes from
work on children's engagement with educational technology (for a review,
see Hirsh-Pasek et al. (2015)). For example, Calvert, Strong, \&
Gallagher (2005) exposed preschool-aged chidren to two sessions of
reading a computer storybook with an adult, and manipulated whether the
adult or the child controlled the mouse, which advanced the story.
Children in the adult-control condition showed a decrease in attention
to the story in the second session; in contrast, children who were given
control over the experience maintained similar levels of attention.

\subsubsection{\texorpdfstring{Active control provides ``good''
information}{Active control provides good information}}\label{active-control-provides-good-information}

Another benefit of active learning is that it allows people to gather
information that is particularly useful for their own learning. That is,
learners have better access to their own prior knowledge, current
hypotheses, and ability level, which they can leverage to create more
helpful learning contexts (e.g., asking a question about something that
is particularly confusing). This benefit of active learning focuses on
learners' adaptive selection of \emph{content} and relies on a set of
metacognitive monitoring skills.

Recent work in cognitive science and machine learning has quantified the
advantage of active selection in adults. For example, Castro et al.
(2009) directly compared human active and passive category learning to
predictions from statistical learning theory under conditions of varying
difficulty. They found that human active learning was always superior to
passive learning, but did not reach the performance of the optimal model
and the advantage for active control decreased in the more difficult
(i.e., noisier) learning tasks. Using a similar approach, D. B. Markant
\& Gureckis (2014) investigated the effects of active vs.~passive
hypothesis testing on the rate of adults' category learning. They varied
the difficulty of the learning task by testing two different types of
category structures: a rule-based category, which varied along 1
dimension (easier to learn), and an information-integration category,
which varied along 2 dimensions (harder to learn). In the active
condition, the learner chose specific observations from the category to
test his or her beliefs, whereas in passive condition, the data were
generated randomly by the experiment. They also included an important
control condition where where passive learners were ``yoked'' to active
learners: that is, they passively encountered the same sequence of data
that was generated by an active learner. Participants in the active
condition learned the category structure faster and achieved a higher
overall accuracy rate compared to participants in the passive learning
condition and the yoked condition, but only for the simpler, rule-based
category.

Together, the Castro et al. (2009) and D. B. Markant \& Gureckis (2014)
findings illustrate several important points about the active learning
advantage. First, the quality of active exploration is fundamentally
linked to the the learner's understanding of the task: if the
representation is poor, then self-directed learning will be biased and
ineffective. The potential for bias in active learning suggests that
receiving passive training first might be especially important for less
constrained learning tasks where people are unlikely to generate
examples that help them learn the target concept. Second, the benefits
of active control are tied to the aspects of the individual learner --
prior knowledge, current hypotheses under consideration -- such that the
same sequence of data might not provide ``good'' information for another
learner. Third, the benefits of active learning diminished with
increased difficulty of the learning problem, likely because learners
could not generate helpful examples.

\subsubsection{What is missing from the active learning
account?}\label{what-is-missing-from-the-active-learning-account}

Active learning in social contexts. The presence of another agent can
change the cost/benefit structure of choices made for learning and
therefore we must include this information in our models of
self-directed learning, which often view the learner as moving back and
forth between active exploration and passive reception. This type of
active learning account does not leave room for social reasoning
processes (i.e., naive utility calculus, goal reasoning) to change the
utility of active learning behaviors.

\section{Part III: Integrating social and active
learning}\label{part-iii-integrating-social-and-active-learning}

\subsubsection{Social information
seeking}\label{social-information-seeking}

Active social learning - seek information from social targets. Models of
seeking information from social targets:

\begin{verbatim}
-  Baldwin & Moses (1998): The Ontogeny of Social Information gathering
-  Chouinard (2007): Children's questions as learning mechanism
-  Hyo's and Liz Bonawitz's work
\end{verbatim}

These studies of the benefits of active information selection connect
nicely to developmental work on children's question asking. Verbal
questions are a spontaneous behavior that occurs in everyday
interactions that could allow children to seek information that is
directly relevant to their current interests and misconceptions.
Moreover, asking a good question is complex: the child must know what
they don't know, how to ask about it, who to ask about it, and be able
to assimilate new information.

There is evidence that children from a young age use questions to gather
information from other people. For example, in a corpus analysis of four
children's parent-child conversations, Chouinard, Harris, \& Maratsos
(2007) found that children begin asking questions early in development
(18 months) and at an impressive rate, ranging from 70-198 questions per
hour of conversation. Chouinard et al. (2007) also coded the intent of
children's questions, finding that 71\% were for the purpose of
gathering information, as opposed to attention getting or
clarifications. Other corpus analyses provide converging evidence that
question asking is a common behavior in parent-child conversations
(Davis, 1932), that children are seeking knowledge with their questions
{[}Bova \& Arcidiacono (2013), and that children will persist in asking
questions if they do not recieve a satisfactory explanation (Frazier,
Gelman, \& Wellman, 2009).

Experimental work has investigated the quality of children's question
asking by measuring the quality of questions in constrained
problem-solving tasks. For example, Legare, Mills, Souza, Plummer, \&
Yasskin (2013) used a modified question asking game where 4- to
6-year-old children saw 16 cards with a drawing of an animal on them.
The animals varied along several dimensions, including type, size, and
pattern on the animal. The child's task was to ask the experimenter
yes-no questions in order to figure out which animal card the
experimenter had hidden in a special box. Legare et al. (2013) coded
whether children asked \emph{constraint-seeking} questions that narrowed
the set of possible cards by increasing knowledge of a particular
dimension or dimensions (e.g., ``Is it red?''), \emph{confirmatory}
questions that provided redundant information, or \emph{ineffective}
questions that did not provide any useful information (e.g., ``Does it
have a tail?''). Results showed that all age groups asked a higher
proprotion of the effective, constraint-seeking questions relative to
the other question types, and that the number of constraint-seeking
questions was correlated with children's accuracy in guessing the
identity of the card hidden in the special box. Legare et al. (2013)
interpret these results as evidence that children can use questions to
solve problems in a efficient manner. Converging evidence in support of
this interpretation comes from experimental work using this approach
finding that children prefer to direct questions to someone who is
knowledgable compared to someone who is inaccurate or ignorant (Mills,
Legare, Bills, \& Mejias, 2010; Mills, Legare, Grant, \& Landrum, 2011),

However, other work has focused on the social context in which question
asking occurs.

\subsubsection{Incorportating social reasoning in active
learning}\label{incorportating-social-reasoning-in-active-learning}

Active learning takes into account a utility structure that can include
both the costs of data acquisition and the rewards of choosing an
example (e.g., in terms of information acquisition/uncertainty reduction
relative to some longer term learning goal).

Focusing on \emph{choices} is useful since there is a rich literature
that has formalized decision-making process, which can be used to
describe behaviors made by both more knowledgeable others and by
learners. The interesting question is how costs/benefits of active
learning behaviors are altered by the social context and how reasoning
about learners as active might change the social context.

Process:

\begin{verbatim}
- analyze costs and benefits of behavior
- planning models that take into account long-term value
- decisions in the brain and in non-human primates 
\end{verbatim}

\section{Conclusion}\label{conclusion}

Models of self-directed learning should include information the
social-communicative context in which learning often occurs. Reasoning
about other people modulate the choices that learners make: whether it's
who to talk to, what to look at, or what questions to ask.

Models of social learning should take into account the choice behaviors
available to the learner. i.e., think about teaching as reasoning about
another person's active learning or setting up a social learning context
where the learner selects actions

\newpage

\section{References}\label{references}

\setlength{\parindent}{-0.4in} \setlength{\leftskip}{0.125in} \noindent

\hypertarget{refs}{}
\hypertarget{ref-adriaans2017prosodic}{}
Adriaans, F., \& Swingley, D. (2017). Prosodic exaggeration within
infant-directed speech: Consequences for vowel learnability. \emph{The
Journal of the Acoustical Society of America}, \emph{141}(5),
3070--3078.

\hypertarget{ref-begus2014infants}{}
Begus, K., Gliga, T., \& Southgate, V. (2014). Infants learn what they
want to learn: Responding to infant pointing leads to superior learning.

\hypertarget{ref-berlyne1960conflict}{}
Berlyne, D. E. (1960). Conflict, arousal, and curiosity.

\hypertarget{ref-bloom2002children}{}
Bloom, P. (2002). \emph{How children learn the meaning of words}. The
MIT Press.

\hypertarget{ref-bonawitz2016computational}{}
Bonawitz, E., \& Shafto, P. (2016). Computational models of development,
social influences. \emph{Current Opinion in Behavioral Sciences},
\emph{7}, 95--100.

\hypertarget{ref-bova2013investigating}{}
Bova, A., \& Arcidiacono, F. (2013). Investigating children's
why-questions: A study comparing argumentative and explanatory function.
\emph{Discourse Studies}, \emph{15}(6), 713--734.

\hypertarget{ref-boyd2011cultural}{}
Boyd, R., Richerson, P. J., \& Henrich, J. (2011). The cultural niche:
Why social learning is essential for human adaptation. \emph{Proceedings
of the National Academy of Sciences}, \emph{108}(Supplement 2),
10918--10925.

\hypertarget{ref-bruner1961act}{}
Bruner, J. S. (1961). The act of discovery. \emph{Harvard Educational
Review}.

\hypertarget{ref-butler2012preschoolers}{}
Butler, L. P., \& Markman, E. M. (2012). Preschoolers use intentional
and pedagogical cues to guide inductive inferences and exploration.
\emph{Child Development}, \emph{83}(4), 1416--1428.

\hypertarget{ref-calvert2005control}{}
Calvert, S. L., Strong, B. L., \& Gallagher, L. (2005). Control as an
engagement feature for young children's attention to and learning of
computer content. \emph{American Behavioral Scientist}, \emph{48}(5),
578--589.

\hypertarget{ref-castro2009human}{}
Castro, R. M., Kalish, C., Nowak, R., Qian, R., Rogers, T., \& Zhu, X.
(2009). Human active learning. In \emph{Advances in neural information
processing systems} (pp. 241--248).

\hypertarget{ref-chi2009active}{}
Chi, M. T. (2009). Active-constructive-interactive: A conceptual
framework for differentiating learning activities. \emph{Topics in
Cognitive Science}, \emph{1}(1), 73--105.

\hypertarget{ref-chouinard2007children}{}
Chouinard, M. M., Harris, P. L., \& Maratsos, M. P. (2007). Children's
questions: A mechanism for cognitive development. \emph{Monographs of
the Society for Research in Child Development}, i--129.

\hypertarget{ref-clark2009first}{}
Clark, E. V. (2009). \emph{First language acquisition}. Cambridge
University Press.

\hypertarget{ref-cleveland2007joint}{}
Cleveland, A., Schug, M., \& Striano, T. (2007). Joint attention and
object learning in 5-and 7-month-old infants. \emph{Infant and Child
Development}, \emph{16}(3), 295--306.

\hypertarget{ref-cooper1990preference}{}
Cooper, R. P., \& Aslin, R. N. (1990). Preference for infant-directed
speech in the first month after birth. \emph{Child Development},
\emph{61}(5), 1584--1595.

\hypertarget{ref-csibra2009natural}{}
Csibra, G., \& Gergely, G. (2009). Natural pedagogy. \emph{Trends in
Cognitive Sciences}, \emph{13}(4), 148--153.

\hypertarget{ref-davis1932form}{}
Davis, E. A. (1932). The form and function of children's questions.
\emph{Child Development}, \emph{3}(1), 57--74.

\hypertarget{ref-de2003investigating}{}
De Boer, B., \& Kuhl, P. K. (2003). Investigating the role of
infant-directed speech with a computer model. \emph{Acoustics Research
Letters Online}, \emph{4}(4), 129--134.

\hypertarget{ref-decasper1987human}{}
DeCasper, A. J., Fifer, W. P., Oates, J., \& Sheldon, S. (1987). Of
human bonding: Newborns prefer their mothers' voices. \emph{Cognitive
Development in Infancy}, 111--118.

\hypertarget{ref-eaves2016infant}{}
Eaves Jr, B. S., Feldman, N. H., Griffiths, T. L., \& Shafto, P. (2016).
Infant-directed speech is consistent with teaching. \emph{Psychological
Review}, \emph{123}(6), 758.

\hypertarget{ref-farroni2002eye}{}
Farroni, T., Csibra, G., Simion, F., \& Johnson, M. H. (2002). Eye
contact detection in humans from birth. \emph{Proceedings of the
National Academy of Sciences}, \emph{99}(14), 9602--9605.

\hypertarget{ref-farroni2007direct}{}
Farroni, T., Massaccesi, S., Menon, E., \& Johnson, M. H. (2007). Direct
gaze modulates face recognition in young infants. \emph{Cognition},
\emph{102}(3), 396--404.

\hypertarget{ref-fernald1987acoustic}{}
Fernald, A., \& Kuhl, P. (1987). Acoustic determinants of infant
preference for motherese speech. \emph{Infant Behavior and Development},
\emph{10}(3), 279--293.

\hypertarget{ref-fernald1991prosody}{}
Fernald, A., \& Mazzie, C. (1991). Prosody and focus in speech to
infants and adults. \emph{Developmental Psychology}, \emph{27}(2), 209.

\hypertarget{ref-fernald1984expanded}{}
Fernald, A., \& Simon, T. (1984). Expanded intonation contours in
mothers' speech to newborns. \emph{Developmental Psychology},
\emph{20}(1), 104.

\hypertarget{ref-frank2014inferring}{}
Frank, M. C., \& Goodman, N. D. (2014). Inferring word meanings by
assuming that speakers are informative. \emph{Cognitive Psychology},
\emph{75}, 80--96.

\hypertarget{ref-frank2009using}{}
Frank, M. C., Goodman, N. D., \& Tenenbaum, J. B. (2009). Using
speakers' referential intentions to model early cross-situational word
learning. \emph{Psychological Science}, \emph{20}(5), 578--585.

\hypertarget{ref-frazier2009preschoolers}{}
Frazier, B. N., Gelman, S. A., \& Wellman, H. M. (2009). Preschoolers'
search for explanatory information within adult--child conversation.
\emph{Child Development}, \emph{80}(6), 1592--1611.

\hypertarget{ref-gelman2008generic}{}
Gelman, S. A., Goetz, P. J., Sarnecka, B. W., \& Flukes, J. (2008).
Generic language in parent-child conversations. \emph{Language Learning
and Development}, \emph{4}(1), 1--31.

\hypertarget{ref-goldstein2008social}{}
Goldstein, M. H., \& Schwade, J. A. (2008). Social feedback to infants'
babbling facilitates rapid phonological learning. \emph{Psychological
Science}, \emph{19}(5), 515--523.

\hypertarget{ref-goodman2009cause}{}
Goodman, N. D., Baker, C. L., \& Tenenbaum, J. B. (2009). Cause and
intent: Social reasoning in causal learning. In \emph{Proceedings of the
31st annual conference of the cognitive science society} (pp.
2759--2764).

\hypertarget{ref-gopnik1999scientist}{}
Gopnik, A., Meltzoff, A. N., \& Kuhl, P. K. (1999). \emph{The scientist
in the crib: Minds, brains, and how children learn.} William Morrow \&
Co.

\hypertarget{ref-grabinger1995rich}{}
Grabinger, R. S., \& Dunlap, J. C. (1995). Rich environments for active
learning: A definition. \emph{Research in Learning Technology},
\emph{3}(2).

\hypertarget{ref-graf2013infant}{}
Graf Estes, K., \& Hurley, K. (2013). Infant-directed prosody helps
infants map sounds to meanings. \emph{Infancy}, \emph{18}(5), 797--824.

\hypertarget{ref-hirsh2015putting}{}
Hirsh-Pasek, K., Zosh, J. M., Golinkoff, R. M., Gray, J. H., Robb, M.
B., \& Kaufman, J. (2015). Putting education in ``educational'' apps:
Lessons from the science of learning. \emph{Psychological Science in the
Public Interest}, \emph{16}(1), 3--34.

\hypertarget{ref-hollich2000breaking}{}
Hollich, G. J., Hirsh-Pasek, K., Golinkoff, R. M., Brand, R. J., Brown,
E., Chung, H. L., \ldots{} Bloom, L. (2000). Breaking the language
barrier: An emergentist coalition model for the origins of word
learning. \emph{Monographs of the Society for Research in Child
Development}, i--135.

\hypertarget{ref-johnson1991newborns}{}
Johnson, M. H., Dziurawiec, S., Ellis, H., \& Morton, J. (1991).
Newborns' preferential tracking of face-like stimuli and its subsequent
decline. \emph{Cognition}, \emph{40}(1), 1--19.

\hypertarget{ref-kachergis2013actively}{}
Kachergis, G., Yu, C., \& Shiffrin, R. M. (2013). Actively learning
object names across ambiguous situations. \emph{Topics in Cognitive
Science}, \emph{5}(1), 200--213.

\hypertarget{ref-kline2015learn}{}
Kline, M. A. (2015). How to learn about teaching: An evolutionary
framework for the study of teaching behavior in humans and other
animals. \emph{Behavioral and Brain Sciences}, \emph{38}.

\hypertarget{ref-kuhl2007speech}{}
Kuhl, P. K. (2007). Is speech learning `gated'by the social brain?
\emph{Developmental Science}, \emph{10}(1), 110--120.

\hypertarget{ref-kuhl2003foreign}{}
Kuhl, P. K., Tsao, F.-M., \& Liu, H.-M. (2003). Foreign-language
experience in infancy: Effects of short-term exposure and social
interaction on phonetic learning. \emph{Proceedings of the National
Academy of Sciences}, \emph{100}(15), 9096--9101.

\hypertarget{ref-legare2013use}{}
Legare, C. H., Mills, C. M., Souza, A. L., Plummer, L. E., \& Yasskin,
R. (2013). The use of questions as problem-solving strategies during
early childhood. \emph{Journal of Experimental Child Psychology},
\emph{114}(1), 63--76.

\hypertarget{ref-markant2014better}{}
Markant, D. B., \& Gureckis, T. M. (2014). Is it better to select or to
receive? Learning via active and passive hypothesis testing.
\emph{Journal of Experimental Psychology: General}, \emph{143}(1), 94.

\hypertarget{ref-markant2016enhanced}{}
Markant, D. B., Ruggeri, A., Gureckis, T. M., \& Xu, F. (2016). Enhanced
memory as a common effect of active learning. \emph{Mind, Brain, and
Education}, \emph{10}(3), 142--152.

\hypertarget{ref-markant2014deconstructing}{}
Markant, D., DuBrow, S., Davachi, L., \& Gureckis, T. M. (2014).
Deconstructing the effect of self-directed study on episodic memory.
\emph{Memory \& Cognition}, \emph{42}(8), 1211--1224.

\hypertarget{ref-mills2010preschoolers}{}
Mills, C. M., Legare, C. H., Bills, M., \& Mejias, C. (2010).
Preschoolers use questions as a tool to acquire knowledge from different
sources. \emph{Journal of Cognition and Development}, \emph{11}(4),
533--560.

\hypertarget{ref-mills2011determining}{}
Mills, C. M., Legare, C. H., Grant, M. G., \& Landrum, A. R. (2011).
Determining who to question, what to ask, and how much information to
ask for: The development of inquiry in young children. \emph{Journal of
Experimental Child Psychology}, \emph{110}(4), 539--560.

\hypertarget{ref-partridge2015young}{}
Partridge, E., McGovern, M. G., Yung, A., \& Kidd, C. (2015). Young
children's self-directed information gathering on touchscreens. In
\emph{Proceedings of the 37th annual conference of the cognitive science
society, austin, tx. cognitive science society}.

\hypertarget{ref-pegg1992preference}{}
Pegg, J. E., Werker, J. F., \& McLeod, P. J. (1992). Preference for
infant-directed over adult-directed speech: Evidence from 7-week-old
infants. \emph{Infant Behavior and Development}, \emph{15}(3), 325--345.

\hypertarget{ref-prince2004does}{}
Prince, M. (2004). Does active learning work? A review of the research.
\emph{Journal of Engineering Education}, \emph{93}(3), 223--231.

\hypertarget{ref-ramirez2017active}{}
Ramirez-Loaiza, M. E., Sharma, M., Kumar, G., \& Bilgic, M. (2017).
Active learning: An empirical study of common baselines. \emph{Data
Mining and Knowledge Discovery}, \emph{31}(2), 287--313.

\hypertarget{ref-reid2005adult}{}
Reid, V. M., \& Striano, T. (2005). Adult gaze influences infant
attention and object processing: Implications for cognitive
neuroscience. \emph{European Journal of Neuroscience}, \emph{21}(6),
1763--1766.

\hypertarget{ref-rogoff1993guided}{}
Rogoff, B., Mistry, J., Göncü, A., Mosier, C., Chavajay, P., \& Heath,
S. B. (1993). Guided participation in cultural activity by toddlers and
caregivers. \emph{Monographs of the Society for Research in Child
Development}, i--179.

\hypertarget{ref-ruggeri2016active}{}
Ruggeri, A., Markant, D. B., Gureckis, T. M., \& Xu, F. (2016). Active
control of study leads to improved recognition memory in children. In
\emph{Proceedings of the 38th annual conference of the cognitive science
society. austin, tx: Cognitive science society}.

\hypertarget{ref-sage2011disentangling}{}
Sage, K. D., \& Baldwin, D. (2011). Disentangling the social and the
pedagogical in infants' learning about tool-use. \emph{Social
Development}, \emph{20}(4), 825--844.

\hypertarget{ref-schulz2012origins}{}
Schulz, L. (2012). The origins of inquiry: Inductive inference and
exploration in early childhood. \emph{Trends in Cognitive Sciences},
\emph{16}(7), 382--389.

\hypertarget{ref-senju2008gaze}{}
Senju, A., \& Csibra, G. (2008). Gaze following in human infants depends
on communicative signals. \emph{Current Biology}, \emph{18}(9),
668--671.

\hypertarget{ref-settles2012active}{}
Settles, B. (2012). Active learning. \emph{Synthesis Lectures on
Artificial Intelligence and Machine Learning}, \emph{6}(1), 1--114.

\hypertarget{ref-shafto2012epistemic}{}
Shafto, P., Eaves, B., Navarro, D. J., \& Perfors, A. (2012). Epistemic
trust: Modeling children's reasoning about others' knowledge and intent.
\emph{Developmental Science}, \emph{15}(3), 436--447.

\hypertarget{ref-shafto2012learning}{}
Shafto, P., Goodman, N. D., \& Frank, M. C. (2012). Learning from others
the consequences of psychological reasoning for human learning.
\emph{Perspectives on Psychological Science}, \emph{7}(4), 341--351.

\hypertarget{ref-shafto2014rational}{}
Shafto, P., Goodman, N. D., \& Griffiths, T. L. (2014). A rational
account of pedagogical reasoning: Teaching by, and learning from,
examples. \emph{Cognitive Psychology}, \emph{71}, 55--89.

\hypertarget{ref-singh2009influences}{}
Singh, L., Nestor, S., Parikh, C., \& Yull, A. (2009). Influences of
infant-directed speech on early word recognition. \emph{Infancy},
\emph{14}(6), 654--666.

\hypertarget{ref-thiessen2005infant}{}
Thiessen, E. D., Hill, E. A., \& Saffran, J. R. (2005). Infant-directed
speech facilitates word segmentation. \emph{Infancy}, \emph{7}(1),
53--71.

\hypertarget{ref-tomasello1986joint}{}
Tomasello, M., \& Farrar, M. J. (1986). Joint attention and early
language. \emph{Child Development}, 1454--1463.

\hypertarget{ref-voss2011spontaneous}{}
Voss, J. L., Warren, D. E., Gonsalves, B. D., Federmeier, K. D., Tranel,
D., \& Cohen, N. J. (2011). Spontaneous revisitation during visual
exploration as a link among strategic behavior, learning, and the
hippocampus. \emph{Proceedings of the National Academy of Sciences},
\emph{108}(31), E402--E409.

\hypertarget{ref-vouloumanos2007listening}{}
Vouloumanos, A., \& Werker, J. F. (2007). Listening to language at
birth: Evidence for a bias for speech in neonates. \emph{Developmental
Science}, \emph{10}(2), 159--164.

\hypertarget{ref-vygotsky1987zone}{}
Vygotsky, L. (1987). Zone of proximal development. \emph{Mind in
Society: The Development of Higher Psychological Processes},
\emph{5291}, 157.

\hypertarget{ref-xu2007word}{}
Xu, F., \& Tenenbaum, J. B. (2007). Word learning as bayesian inference.
\emph{Psychological Review}, \emph{114}(2), 245.

\hypertarget{ref-yoon2008communication}{}
Yoon, J. M., Johnson, M. H., \& Csibra, G. (2008). Communication-induced
memory biases in preverbal infants. \emph{Proceedings of the National
Academy of Sciences}, \emph{105}(36), 13690--13695.

\hypertarget{ref-yu2007unified}{}
Yu, C., \& Ballard, D. H. (2007). A unified model of early word
learning: Integrating statistical and social cues.
\emph{Neurocomputing}, \emph{70}(13), 2149--2165.

\hypertarget{ref-yu2012embodied}{}
Yu, C., \& Smith, L. B. (2012). Embodied attention and word learning by
toddlers. \emph{Cognition}.

\hypertarget{ref-yu2013joint}{}
Yu, C., \& Smith, L. B. (2013). Joint attention without gaze following:
Human infants and their parents coordinate visual attention to objects
through eye-hand coordination. \emph{PloS One}, \emph{8}(11), e79659.

\hypertarget{ref-yu2016social}{}
Yu, C., \& Smith, L. B. (2016). The social origins of sustained
attention in one-year-old human infants. \emph{Current Biology},
\emph{26}(9), 1235--1240.

\hypertarget{ref-yu2005role}{}
Yu, C., Ballard, D. H., \& Aslin, R. N. (2005). The role of embodied
intention in early lexical acquisition. \emph{Cognitive Science},
\emph{29}(6), 961--1005.

\bibliography{}

\end{document}
