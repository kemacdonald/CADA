% Template for APA submission with R Markdown

% Stuff changed from PLOS Template
\documentclass[a4paper,man,apacite,floatsintext]{apa6}
\usepackage{apacite}

% amsmath package, useful for mathematical formulas
\usepackage{amsmath}
% amssymb package, useful for mathematical symbols
\usepackage{amssymb}

% hyperref package, useful for hyperlinks
\usepackage{hyperref}

% graphicx package, useful for including eps and pdf graphics
% include graphics with the command \includegraphics
\usepackage{graphicx}

% Sweave(-like)
\usepackage{fancyvrb}
\DefineVerbatimEnvironment{Sinput}{Verbatim}{fontshape=sl}
\DefineVerbatimEnvironment{Soutput}{Verbatim}{}
\DefineVerbatimEnvironment{Scode}{Verbatim}{fontshape=sl}
\newenvironment{Schunk}{}{}
\DefineVerbatimEnvironment{Code}{Verbatim}{}
\DefineVerbatimEnvironment{CodeInput}{Verbatim}{fontshape=sl}
\DefineVerbatimEnvironment{CodeOutput}{Verbatim}{}
\newenvironment{CodeChunk}{}{}

% cite package, to clean up citations in the main text. Do not remove.
\usepackage{cite}

\usepackage{color}

% Use doublespacing - comment out for single spacing
%\usepackage{setspace}
%\doublespacing


% Text layout
\topmargin 0.0cm
\oddsidemargin 0.5cm
\evensidemargin 0.5cm
\textwidth 16cm
\textheight 21cm

% Bold the 'Figure #' in the caption and separate it with a period
% Captions will be left justified
\usepackage[labelfont=bf,labelsep=period,justification=raggedright]{caption}


% Remove brackets from numbering in List of References
\makeatletter
\renewcommand{\@biblabel}[1]{\quad#1.}
\makeatother


% Leave date blank
\date{}

%\pagestyle{myheadings}
%% ** EDIT HERE **


%% ** EDIT HERE **
%% PLEASE INCLUDE ALL MACROS BELOW

%% END MACROS SECTION


% ALL OF THE TITLE PAGE INFORMATION IS SPECIFIED IN THE YAML
\title{\textbf{Social contexts shape learners' choices during language learning}}
\shorttitle{Active learning in social contexts}

\author{Kyle MacDonald}

\affiliation{Department of Psychology, Stanford University}

\authornote{Fulfillment of Conceptual Analysis of Dissertation Area. Readers:
Michael C. Frank, Hyowon Gweon, and Anne Fernald}
\abstract{Human language acquisition presents a remarkable learning challenge.
Fortunately, children acquire language from knowledgable caregivers who
can shape the input to help them learn. Moroever, as children get older,
they become more adept at choosing behaviors to shape the environment to
better support their own learning. But how do these two important
learning forces interact? In this paper, I explore the hypothesis that
social languge learning contexts can be productively construed as
opportunities for constrained active learning. I review the literature
showing how active and social-pedagogcial contexts change learning
behaviors. Then, I review work that attempts to synthesize active and
social learning. Finally, I present a case study of decisions about
where to allocate visual attention during language comprehension as an
example of how including the presence of a social learning contexts can
change the underlying computations that drive behavior during language
learning.}
\keywords{language acquisition, active learning, social learning, theory}

\begin{document}
\maketitle

\section{Introduction}\label{introduction}

Human language learning is a remarkable feat. Consider that infants,
despite possessing limited information processing capacities, go on to
develop adult vocabularies ranging from 50,000 to 100,000 distinct words
(Bloom, 2002), allowing them to express internal desires, to coordinate
with others, and to transmit cultural knowledge to the next generation.
But, to become a skilled language user is by no means trivial and
children must solve a variety of complex problems such as word
segmentation (Saffran, Aslin, \& Newport, 1996), reducing referential
uncertainty (Quine, 1960), and making appropriate generalizations
(CITE). Moreover, they must do this all while processing a dynamic,
emphemoral stream of information that unfolds rapidly in time.

Fortunately, language learners do not have to solve these problems on
their own since much of their learning input comes from caregivers,
other knowledgeable adults, and older peers (Bloom, 2002; Clark, 2009).
Moreover, as children get older and become more adept at intervening on
their environments, they can take a more ``active'' role in shaping the
input to their current state (Baldwin \& Moses, 1996). And recent work
in machine learning (Settles, 2012) and cognitive science (Castro et
al., 2009; Markant \& Gureckis, 2014) suggests that providing the
learner additional control can lead to superior learning outcomes
compared to more passive contexts.

However, one of the fundamental open questions in developmental
psychology is to explain how children's developing active learning
skills interact with their social learning environments to support their
their impressive languauge learning. In this paper, I present the
hypothesis that the social context in which language learning occurs can
be productively construed as opportunities for constrained active
learning. The key insight is that language learning is a fundamentally
social endeavor where the presence of another agent shapes the
cost/benefit calculus for learners' choices. This analysis attempts to
explain a range of choice behaviors available to the learner during
language lerning from the micro (eye-movements to gather information to
understand langauge in real-time) to the macro (decisions about
conversational partners).

The plan for the paper is as follows. First, I review the literature on
the basic mechanisms that lead to better outcomes for active learning
compared to passive learning. Then, I present formal models of ow
learning changes in social contexts along with evidence that supports
the importance of social-pragmatic contexts for language learning.
Finally, I present a concpetual analysis of the active allocation of
visual attention during language comprehension as a case study of how
the active-social conceptual framework provides a way forward for
understanding the interplay between social and active learning.

\section{Part 1: Active learning}\label{part-1-active-learning}

\subsection{\texorpdfstring{What is ``active''
learning?}{What is active learning?}}\label{what-is-active-learning}

The potential benefits of active learning have been the focus of much
research in education (Grabinger \& Dunlap, 1995), machine learning
(Settles, 2012), and cognitive science (Castro et al., 2009). In a
review of this diverse literature, Gureckis \& Markant (2012) suggest
that active learning can be superior to passive learning because it
allows people to use their prior experience and current hypotheses to
select the most helpful examples (e.g., asking a question about
something that is particularly confusing). But is active learning always
better than passive learning?

\subsection{Why is active learning superior to passive
learning?}\label{why-is-active-learning-superior-to-passive-learning}

\subsection{What is missing from the active learning
account?}\label{what-is-missing-from-the-active-learning-account}

\section{Part 2: Social learning}\label{part-2-social-learning}

\subsection{What is social learning?}\label{what-is-social-learning}

Social learning is the accumulation of knowledge based on the sampling
decisions of other agents (e.g., via the framework in Shafto et al.,
2012). Requires reasoning about why the other agent made the choices
they did.

\subsection{Why is social learning so
powerful?}\label{why-is-social-learning-so-powerful}

\subsection{Different models of social
learning}\label{different-models-of-social-learning}

Examples: * Sobel and Kirkham * observational learning * pedagogical
inference * social as attention vs.~social as changing underlying
inferences because of reasoning about others minds

\subsection{What is missing from the social learning
account?}\label{what-is-missing-from-the-social-learning-account}

\section{Part 3: Active learning behaviors in social
contexts}\label{part-3-active-learning-behaviors-in-social-contexts}

Models of seeking information from social targets: * Baldwin \& Moses
(1998): The Ontogeny of Social Information gathering * Chouinard (2007):
Children's questions as learning mechanism

\section{Part 4: Case study of active information gathering via visual
fixations}\label{part-4-case-study-of-active-information-gathering-via-visual-fixations}

\section{Conclusions}\label{conclusions}

Models of self-directed learning cannot continue to ignore the
social-communicative context in which learning often occurs.
Reasoning/inferences about other people should modulate the choices that
learners make: whether it's who to talk to, what to attend to, or what
questions to ask.

\newpage

\section{References}\label{references}

\setlength{\parindent}{-0.4in} \setlength{\leftskip}{0.125in} \noindent

\hypertarget{refs}{}
\hypertarget{ref-baldwin1996ontogeny}{}
Baldwin, D. A., \& Moses, L. J. (1996). The ontogeny of social
information gathering. \emph{Child Development}, \emph{67}(5),
1915--1939.

\hypertarget{ref-bloom2002children}{}
Bloom, P. (2002). \emph{How children learn the meaning of words}. The
MIT Press.

\hypertarget{ref-castro2009human}{}
Castro, R. M., Kalish, C., Nowak, R., Qian, R., Rogers, T., \& Zhu, X.
(2009). Human active learning. In \emph{Advances in neural information
processing systems} (pp. 241--248).

\hypertarget{ref-clark2009first}{}
Clark, E. V. (2009). \emph{First language acquisition}. Cambridge
University Press.

\hypertarget{ref-grabinger1995rich}{}
Grabinger, R. S., \& Dunlap, J. C. (1995). Rich environments for active
learning: A definition. \emph{Research in Learning Technology},
\emph{3}(2).

\hypertarget{ref-gureckis2012self}{}
Gureckis, T. M., \& Markant, D. B. (2012). Self-directed learning a
cognitive and computational perspective. \emph{Perspectives on
Psychological Science}, \emph{7}(5), 464--481.

\hypertarget{ref-markant2014better}{}
Markant, D. B., \& Gureckis, T. M. (2014). Is it better to select or to
receive? Learning via active and passive hypothesis testing.
\emph{Journal of Experimental Psychology: General}, \emph{143}(1), 94.

\hypertarget{ref-quine19600}{}
Quine, W. V. (1960). 0. word and object. \emph{111e MIT Press}.

\hypertarget{ref-saffran1996statistical}{}
Saffran, J. R., Aslin, R. N., \& Newport, E. L. (1996). Statistical
learning by 8-month-old infants. \emph{Science}, \emph{274}(5294),
1926--1928.

\hypertarget{ref-settles2012active}{}
Settles, B. (2012). Active learning. \emph{Synthesis Lectures on
Artificial Intelligence and Machine Learning}, \emph{6}(1), 1--114.

\bibliography{}

\end{document}
