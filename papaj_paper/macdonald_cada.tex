\documentclass[english,man]{apa6}

\usepackage{amssymb,amsmath}
\usepackage{ifxetex,ifluatex}
\usepackage{fixltx2e} % provides \textsubscript
\ifnum 0\ifxetex 1\fi\ifluatex 1\fi=0 % if pdftex
  \usepackage[T1]{fontenc}
  \usepackage[utf8]{inputenc}
\else % if luatex or xelatex
  \ifxetex
    \usepackage{mathspec}
    \usepackage{xltxtra,xunicode}
  \else
    \usepackage{fontspec}
  \fi
  \defaultfontfeatures{Mapping=tex-text,Scale=MatchLowercase}
  \newcommand{\euro}{€}
\fi
% use upquote if available, for straight quotes in verbatim environments
\IfFileExists{upquote.sty}{\usepackage{upquote}}{}
% use microtype if available
\IfFileExists{microtype.sty}{\usepackage{microtype}}{}

% Table formatting
\usepackage{longtable, booktabs}
\usepackage{lscape}
% \usepackage[counterclockwise]{rotating}   % Landscape page setup for large tables
\usepackage{multirow}		% Table styling
\usepackage{tabularx}		% Control Column width
\usepackage[flushleft]{threeparttable}	% Allows for three part tables with a specified notes section
\usepackage{threeparttablex}            % Lets threeparttable work with longtable

% Create new environments so endfloat can handle them
% \newenvironment{ltable}
%   {\begin{landscape}\begin{center}\begin{threeparttable}}
%   {\end{threeparttable}\end{center}\end{landscape}}

\newenvironment{lltable}
  {\begin{landscape}\begin{center}\begin{ThreePartTable}}
  {\end{ThreePartTable}\end{center}\end{landscape}}

  \usepackage{ifthen} % Only add declarations when endfloat package is loaded
  \ifthenelse{\equal{\string man}{\string man}}{%
   \DeclareDelayedFloatFlavor{ThreePartTable}{table} % Make endfloat play with longtable
   % \DeclareDelayedFloatFlavor{ltable}{table} % Make endfloat play with lscape
   \DeclareDelayedFloatFlavor{lltable}{table} % Make endfloat play with lscape & longtable
  }{}%



% The following enables adjusting longtable caption width to table width
% Solution found at http://golatex.de/longtable-mit-caption-so-breit-wie-die-tabelle-t15767.html
\makeatletter
\newcommand\LastLTentrywidth{1em}
\newlength\longtablewidth
\setlength{\longtablewidth}{1in}
\newcommand\getlongtablewidth{%
 \begingroup
  \ifcsname LT@\roman{LT@tables}\endcsname
  \global\longtablewidth=0pt
  \renewcommand\LT@entry[2]{\global\advance\longtablewidth by ##2\relax\gdef\LastLTentrywidth{##2}}%
  \@nameuse{LT@\roman{LT@tables}}%
  \fi
\endgroup}


\ifxetex
  \usepackage[setpagesize=false, % page size defined by xetex
              unicode=false, % unicode breaks when used with xetex
              xetex]{hyperref}
\else
  \usepackage[unicode=true]{hyperref}
\fi
\hypersetup{breaklinks=true,
            pdfauthor={},
            pdftitle={How social contexts shape active learning},
            colorlinks=true,
            citecolor=blue,
            urlcolor=blue,
            linkcolor=black,
            pdfborder={0 0 0}}
\urlstyle{same}  % don't use monospace font for urls

\setlength{\parindent}{0pt}
%\setlength{\parskip}{0pt plus 0pt minus 0pt}

\setlength{\emergencystretch}{3em}  % prevent overfull lines

\ifxetex
  \usepackage{polyglossia}
  \setmainlanguage{}
\else
  \usepackage[english]{babel}
\fi

% Manuscript styling
\captionsetup{font=singlespacing,justification=justified}
\usepackage{csquotes}
\usepackage{upgreek}

 % Line numbering
  \usepackage{lineno}
  \linenumbers


\usepackage{tikz} % Variable definition to generate author note

% fix for \tightlist problem in pandoc 1.14
\providecommand{\tightlist}{%
  \setlength{\itemsep}{0pt}\setlength{\parskip}{0pt}}

% Essential manuscript parts
  \title{How social contexts shape active learning}

  \shorttitle{Active learning is social}


  \author{Kyle MacDonald\textsuperscript{1}}

  \def\affdep{{""}}%
  \def\affcity{{""}}%

  \affiliation{
    \vspace{0.5cm}
          \textsuperscript{1} Stanford University  }

  \authornote{
    \newcounter{author}
    Conceptual Analysis of Dissertation Area.
    
    Readers: Michael C. Frank, Hyowon Gweon, and Anne Fernald

                        }


  \abstract{Children's rapid conceptual development is one of the more remarkable
features of human cognition. How do they learn so much so quickly?
Social learning theories argue for the importance of learning from more
knowledgeable others. In contrast, active learning accounts focus on
children's knowledge acquisition via thei self-directed exploration. In
this paper, I argue that an important step towards a more complete
theory of early learning is to understand how active learning behaviors
unfold in social learning contexts. To integrate the two theoretical
accounts, I use ideas from theories of rational decision making that
emphasize the expected utility and cost of different actions in order to
explain choice behavior. The key insight is that the costs and benefits
of active learning behaviors (e.g., metacognitive monitoring,
spontaneous exploration, and question asking) are fundamentally shaped
by interactions with other people.}
  \keywords{active learning, social learning, decision making, theory \\

    \indent Word count: X
  }





\usepackage{amsthm}
\newtheorem{theorem}{Theorem}
\newtheorem{lemma}{Lemma}
\theoremstyle{definition}
\newtheorem{definition}{Definition}
\newtheorem{corollary}{Corollary}
\newtheorem{proposition}{Proposition}
\theoremstyle{definition}
\newtheorem{example}{Example}
\theoremstyle{definition}
\newtheorem{exercise}{Exercise}
\theoremstyle{remark}
\newtheorem*{remark}{Remark}
\newtheorem*{solution}{Solution}
\begin{document}

\maketitle

\setcounter{secnumdepth}{0}



\section{Introduction}\label{introduction}

Human learning is remarkable. Consider that children, despite
limitations on their general processing capabilities, are able to
acquire new concepts at a high rate, eventually reaching an adult
vocabulary ranging between 50,000 to 100,000 words (P. Bloom, 2002). And
they accomplish this while also developing motor skills, learning social
norms, and building causal knowledge. What sorts of processes can
account for children's prodigious learning abilities?

Social learning theories offer a solution by pointing out that children
do not solve these problems on their own. And although children learn a
great deal from observation, they are typically surrounded by parents,
other knowledgeable adults, or older peers -- all of whom likely know
more than they do. These social contexts can bootstrap learning via
several mechanisms. For example, work on early language acquisition
shows that social partners provide input that is tuned to children's
cognitive abilities (Eaves Jr, Feldman, Griffiths, \& Shafto, 2016;
Fernald \& Kuhl, 1987), that guides children's attention to important
features in the world (C. Yu \& Ballard, 2007), and that increases
levels of arousal and sustained attention, which lead to better learning
(P. K. Kuhl, 2007; C. Yu \& Smith, 2016).

Social contexts can also change the computations that support children'
learning from evidence. Recent work on both concept learning and causal
intervention suggests that the presence of another person leads the
learner to reason about \emph{why} people perform certain actions. The
key insight is that knowledge of the underlying process that generates
examples allows learners to make more appropriate inferences that speed
learning (E. Bonawitz \& Shafto, 2016; Frank, Goodman, \& Tenenbaum,
2009; Shafto, Goodman, \& Griffiths, 2014). For example, people will
draw different inferences from observing the same actions depending on
whether they think that the behavior was accidental or intentional.
Moreover, adults and children will make even stronger inferences if they
think an action was selected with the goal to help them learn (i.e.,
teaching) (Shafto, Goodman, \& Frank, 2012).

However, children are not merely passive recipients of information --
from people or from the world. Instead, children actively select
behaviors (e.g., asking questions or choosing where to allocate
attention) that change the content, pacing, and sequence of the
information that they receive. In fact, recent theorizing and empirical
work uses the metaphor of \enquote{child as intuitive scientist} and
characterizes early learning asa process of exploration and hypothesis
testing following principles of the scientific method (Gopnik, Meltzoff,
\& Kuhl, 1999; L. Schulz, 2012). Moreover, recent empirical work across
a variety of domains (education, machine learning, and cognitive
science) has begun to directly compare the benefits of self-directed
choice compared to passive contexts for speeding learning outcomes via
increases in learners' basic learning processes (attention and arousal)
or by providing learners with better information that is more tightly
linked to their current beliefs, goals, and interests (Castro et al.,
2009; D. B. Markant \& Gureckis, 2014; Settles, 2012).

Thus, both social and active learning accounts argue that cognitive
development is facilitated by the activation of distinct learning
processes and by the learner gaining access to better information from
the environment. However, children's learning often involves a mixing of
active behaviors (choices) with social contexts where other people
respond to children's actions. Thus, an efficient learner must integrate
information that they generate with information provided by other people
to decide what to do next. Thus, one of the fundamental challenges for
building a complete theory of human learning is to characterize the
effects of social contexts on self-directed learning behaviors.

In this paper, I take a step towards the goal of integrating pieces of
the active and social learning accounts. To do this, I will use the
framework of Optimal Experiment Design (Emery \& Nenarokomov, 1998),
which provides a useful formalization of human inquiry. One of the key
insights is that social learning contexts can help to \emph{scaffold}
the decision processes (i.e., choices) that confronts a self-directed
learner. I also propose that findings from social learning research
provide a way to explain a set of puzzles in the study of children's
uncertainty monitoring (markman; kim et al.), spontaneous information
seeking (begus; Deaf of Hearing social referencing), and verbal question
asking (katz et al. (2010). Before presenting the integrative account of
active and social learning, it is useful to review the evidence showing
that both social contexts (Part 1) and self-directed learning behaviors
(Part 2) fundamentally change how learning unfolds.

\section{Part I: Learning from other
people}\label{part-i-learning-from-other-people}

Social learning theories argue that children's rapid conceptual
development is facilitated by humans' unique capacity to transmit and
acquire information from other people.\footnote{In this paper, I define
  \enquote{social} contexts as learning environments where another agent
  is present. This definition includes a broad range of social learning
  behaviors: e.g., observation, imitation, and learning from direct
  pedagogy.} One of the primary benefits of cultural learning is that
children gain access to knowledge that has accumulated over many
generations; information that would be far too complex for any
individual person to figure out on their own (Boyd, Richerson, \&
Henrich, 2011). In addition to these cumulative effects, social contexts
facilitate learning because more knowledgeable others select the input
that could best support children's learning (Kline, 2015; Shafto et al.,
2012), providing learning opportunities for generalizable information
(Csibra \& Gergely, 2009).

There is now a large body of empirical work on children's learning that
show the effect of social contexts across a variety of domains. These
learning effects manifest via different pathways such as guiding
attention, increasing arousal, providing better information, and
changing the strength of children's inferences. In this section, I
briefly review the evidence for the role of each of these social
learning processes, with the goal of providing a high-level taxonomy of
social learning effects. Outlining these social learning effects will
set the stage for the discussion of how they shape self-directed
learning behaviors in Part III.

\subsubsection{Social interactions enhance attention and
memory}\label{social-interactions-enhance-attention-and-memory}

From infancy humans preferentially attend to social information. For
example, newborn infants will choose to look at face-like patterns
compared to other abstract configurations (Johnson, Dziurawiec, Ellis,
\& Morton, 1991) and will even show a preference for faces that make
direct eye contact compared to faces with averted gaze (Farroni, Csibra,
Simion, \& Johnson, 2002). In the auditory domain, newborns prefer to
listen to speech over non-speech (Vouloumanos \& Werker, 2007), their
mother's voice over other voices (DeCasper, Fifer, Oates, \& Sheldon,
1987), and infant-directed speech over adult-directed speech (Cooper \&
Aslin, 1990; Fernald \& Kuhl, 1987; Pegg, Werker, \& McLeod, 1992). And
recent work by C. Yu and Smith (2016) using head-mounted eye trackers to
record parent-child interactions found that one-year-olds will sustain
visual attention to an object longer when their parents' had previously
looked at that object.

These early attentional biases can lead to differential outcomes when
learning occurs with another person present. For example, 4-month-olds'
show better memory for faces if that face gazed directly at them as
compared to memory for a face with averted gaze (Farroni, Massaccesi,
Menon, \& Johnson, 2007) and for objects if an adult gazed at that
object during learning (Cleveland, Schug, \& Striano, 2007; Reid \&
Striano, 2005). Moreover, 7-month-olds perform better at word
segmentation if the words are presented in infant-directed speech
compared to adult-directed speech (Thiessen, Hill, \& Saffran, 2005).

P. K. Kuhl (2007) refer to these effects as \enquote{social gating}
phenomena since the presence of another person activates or enhances
children's underlying computational learning mechanisms such as
attention. One particularly striking piece of evidence for the social
gating hypothesis comes from P. K. Kuhl, Tsao, and Liu (2003)\enquote{s
study of infants} foreign-language phonetic learning. In this
experiment, 9-10 month-old English-learning infants listened to Mandarin
speakers either via live interactions or via audiovisual recordings and
their ability to discriminate Mandarin-specific phonemes was assessed
two months later. Only the infants who were exposed to Mandarin within
social interactions were able to succeed on the phonetic discrimination
task and infants in the audiovisual recording condition showed no
evidence of learning. P. K. Kuhl et al. (2003) also provided evidence
that infants in the social interaction condition showed higher rates of
visual attention to the speaker, suggesting that the social contexts
enhanced learning by increasing children's attention to the input.

The common thread across these findings is that the presence of another
person is a particularly good way to increase attention. In this model,
social input becomes more salient and therefore more likely to come into
contact with general learning mechanisms. However, increases in arousal,
attention, and memory are only one way that social contexts can
influence learning. In fact, one of the defining features of early
learning environments is the presence of other people who know more than
the child, creating opportunities for more knowledgeable others to
select learning experiences that are particularly beneficial -- either
because the information is tuned to children's current cognitive
abilities or because the information is likely to be generalizable.

\subsubsection{\texorpdfstring{Social interactions provide
\enquote{good}
information}{Social interactions provide good information}}\label{social-interactions-provide-good-information}

The notion that children's input might be shaped to facilitate their
learning is a key tenet of several influential theories of cognitive
development (e.g., Zone of Proximal Development (Vygotsky, 1987), Guided
Participation (Rogoff et al., 1993), and Natural Pedagogy (Csibra \&
Gergely, 2009)). But how do social interactions provide particularly
useful information for children's learning?

A particularly compelling set of evidence comes from studies of how
caregivers alter the way they communicate with young children. That is,
adults do not speak to children in the same way as they speak to other
adults; instead, they exaggerate prosody, reduce speed, shorten
utterances, and elevate both pitch and affect (for a review, see
(Fernald \& Simon, 1984)). And subsequent empirical work has shown that
these features of \enquote{infant-directed speech} facilitate vowel
learning (Adriaans \& Swingley, 2017; De Boer \& Kuhl, 2003), word
segmentation (Fernald \& Mazzie, 1991; Thiessen et al., 2005), word
recognition (Singh, Nestor, Parikh, \& Yull, 2009), and word learning
(Graf Estes \& Hurley, 2013).

Work on infants' early vocal production also provides evidence for the
importance social feedback, highlighting the feature of
\emph{contingency}. For example, Goldstein and Schwade (2008) measured
whether infants modified their babbling to produce more speech-like
sounds after interacting with caregivers who either provided contingent
or non-contingent responses to infants' babbling. They found that only
infants in the contingent feedback condition changed their vocalization
behavior to produce more adult-like language forms. Goldstein and
Schwade (2008) hypothesized that the contingency effect was driven by
infants' receiving input that was particularly useful for solving this
learning problem since the feedback was close in time to infants'
vocalizations, making it easier for them to compare discrepancies
between the two.

A third piece of evidence comes from research on children's early word
learning. Social-pragmatic theories of language acquisition have
emphasized the importance of social cues for reducing the (in principle)
unlimited amount of referential uncertainty present when children are
trying to acquire novel words (P. Bloom, 2002; Clark, 2009; Hollich et
al., 2000). Empirical work by C. Yu and Smith (2012a) shows that young
learners tend to retain words that are accompanied by clear referential
cues, which serve to make a single object dominant in the visual field
(see also (C. Yu \& Smith, 2013; C. Yu, Ballard, \& Aslin, 2005).
Moreover, correlational studies show positive links between early
vocabulary development and parents' tendency to refer to objects that
children are already attending to (i.e., \enquote{follow-in} labeling)
(Tomasello \& Farrar, 1986).

Thus far, I have reviewed evidence showing that social information can
benefit learning because it enhances attention and it contains features
that make it easier to learn. Learning from other people also changes
learning by engaging distinct social reasoning processes that change how
learners interpret and learn from evidence.

\subsubsection{Social interactions influence inferences and
generalization}\label{social-interactions-influence-inferences-and-generalization}

Perhaps one of the defining features of human social learning is that
teachers and learners' actions are not random. Instead, people select
behaviors with respect to some goal (e.g., to communicate a concept),
and learners reason about \emph{why} someone chose to perform a
particular action. The key point is that this reciprocal process of
reasoning about others' goal-directed actions can change how people
interpret superficially similar behaviors, altering the learning
process.

In recent empirical and modeling work, Shafto et al. (2012) formalized
this social reasoning process within the framework of Bayesian models of
cognition. In these models, learning is a process of belief updating
that depends on two factors: what the learner believed before seeing the
data and what the learner thinks about the process that generated the
data. The key insight is that if the learner assumes that information is
generated with the intention to communicate/teach, then they can make
\enquote{stronger} inferences.\footnote{Formally, these models change
  the form of the likelihood term in Bayes theorem in order to capture a
  person's theory of how data are generated.}

For example, Goodman, Baker, and Tenenbaum (2009) presented adults with
causal learning scenarios with the following structure: either the
participant or someone else who knows the causal structure generates an
effect (e.g., growing flowers) by performing two actions at the same
time (e.g., pouring a yellow liquid and a blue liquid). The
participant's task was to identify the correct causal structure. Results
showed that when participants thought the other person was
knowledgeable, they were more likely to infer that performing
\emph{both} actions was necessary. In contrast, when the participant
performed the action on their own (and did not know the causal
structure), adults were less sure that both actions were necessary.
Shafto et al. (2012) interpreted these results as learners going through
a psychological reasoning process such as \enquote{if the other person
was knowledgeable and wanted to generate the effect, he would definitely
perform both actions if that was the correct causal structure.}

Similar psychological reasoning effects have been shown in the domains
of word learning (Frank \& Goodman, 2014; Xu \& Tenenbaum, 2007),
selective trust in testimony (Shafto, Eaves, Navarro, \& Perfors, 2012),
tool use (Sage \& Baldwin, 2011), and concept learning (Shafto et al.,
2014). Moreover, there is evidence that even young learners are
sensitive to the presence of others' goal-directed behaviors. For
example, Yoon, Johnson, and Csibra (2008) showed that 8-month-olds will
encode an object's identity if their attention was directed by a
communicative point, but they will encode an object's spatial location
if their attention was directed by non-communicative reach. And Senju
and Csibra (2008) found that infants will follow another person's gaze
only if the gaze event was preceded by the person providing a relevant,
communicative cue (e.g., infant-directed speech or direct eye contact).

In addition to being easier to learn, information acquired in social
contexts is also more likely to generalize and be useful beyond the
current learning context. Csibra and Gergely (2009) argue that this
assumption of \emph{generalizability} is a fundamental component of
\enquote{Natural Pedagogy} -- a uniquely human communication system that
allows adults to efficiently pass along cultural knowledge to children.
Experimental work testing predictions from this account shows that
children are biased to think that information presented in communicative
contexts is generalizable (Butler \& Markman, 2012; Yoon et al., 2008),
and corpus analyses provide evidence that generic language (e.g.,
\enquote{birds fly}) is common in everyday adult-child conversations
(Gelman, Goetz, Sarnecka, \& Flukes, 2008).

Across all of these studies, learners interpreted similar information in
different ways depending on their assumptions about other people's
goals. These effects are different from the attentional and
informational explanations reviewed above in that the inferences based
on social information are part of the underlying computations that
support learning. This account fits well with evolutionary models that
emphasize the importance of pedagogy for the accumulation of human
cultural knowledge (Boyd et al., 2011; Kline, 2015) and theories of
cognitive development that emphasize the adult's role as providing
children with generalizable information (Csibra \& Gergely, 2009).

\begin{table}[h]
\begin{center}
\begin{threeparttable}
\caption{\label{tab:learning_overview}How active and social learning affect basic learning processes with relevant citations.}
\begin{tabular}{llll}
\toprule
Type of learning & \multicolumn{1}{c}{Change in learning process} & \multicolumn{1}{c}{Proposed mechanism} & \multicolumn{1}{c}{Example papers}\\
\midrule
Social & Enhanced attention and memory & The presence of social partners makes information more salient and more likely to be learned & NA\\
Social & Better information & Social partners tune input to learner's current cognitive state & NA\\
Social & More restrictive Inferences and generalization & Psychological reasoning leads to stronger inferences. Information presented in social contexts is more likely to generalize & NA\\
Active & Enhanced memory & Active behaviors engage a range of basic memory mechanisms such as encoding of distinctive sensorimotor associations, coordination of selective attention and encoding, and enhanced metacognitive monitoring & NA\\
Active & Better information & Learner have access to prior knowledge, current hypotheses, and ability level, which they can leverage to create more helpful learning contexts & NA\\
Active & Less restrictive inferences and generalization & In these cases, self-directed learners might at first default to a weak sampling assumption. Weak sampling implies less restrictive generalization because the sampling process itself conveys no information about the to-be-learned concept. & NA\\
\bottomrule
\end{tabular}
\end{threeparttable}
\end{center}
\end{table}

\section{Part II: Learning on your
own}\label{part-ii-learning-on-your-own}

Another key ingredient for children's rapid conceptual development is
their ability to learn on their own. That is, children are not just
passive recipients of information; instead, they actively seek knowledge
via their own actions. This model of the child as an \enquote{active}
learner has been an influential idea in many classic theories of
cognitive development (e.g., Bruner (1961); Berlyne (1960)). And recent
theorizing has characterized cognitive development as a process of
active hypothesis testing and theory revision following principles
similar to the scientific method (Gopnik et al., 1999; L. Schulz, 2012).

In addition to playing a prominent role in developmental theory, the
potential benefits of \enquote{active}\footnote{The term \enquote{active
  learning} has been used to describe a wide variety of behaviors such
  as question asking, increased physical activity, or active memory
  retrieval. In this paper, I focus on a specific subset of these
  behaviors: the \emph{decisions} that people make, or could make,
  during learning. This definition captures several ways that people can
  exert control over their learning experiences, including the
  selection, sequencing, and/or pacing of new information.} learning
have been the focus of a great deal of empirical work in education
(Grabinger \& Dunlap, 1995; Prince, 2004), machine learning
(Ramirez-Loaiza, Sharma, Kumar, \& Bilgic, 2017; Settles, 2012), and
cognitive psychology (Castro et al., 2009; Chi, 2009). The common
finding across these studies is that active learning contexts -- where
people have control over some aspect of the learning environment -- lead
to better outcomes when compared to passive contexts where people do not
have control over the information that they receive.

But what makes active control a useful way to learn about the world? In
this section, I present evidence for two mechanisms -- enhanced
attention/memory and higher quality information -- through which active
control can improve learning outcomes. I then review work that
formalizes human inquiry as a process of \enquote{optimal experiment
design} (OED) to ask when and how human self-directed learning deviates
from optimal information gathering principles. I conclude Part II with a
discussion of what makes optimal active learning difficult and why this
is an intresting point of contact with research on children's social
learning.

\subsubsection{Active control enhances attention and
memory}\label{active-control-enhances-attention-and-memory}

A growing body of work has explored the effect of active control on
basic processes underlying learning and memory. In these tasks, outcomes
for active and passive learning experiences are directly compared across
a variety of tasks, such as episodic memory, casual learning, and
concept learning. D. B. Markant, Ruggeri, Gureckis, and Xu (2016) review
this diverse literature and suggest that the active learning advantage
found across these domains is caused by an increase in attention and
memory with the precise pathway determined by the type of control in the
study. For example, one effect of active control is that is allows
people to coordinate the timing of incoming information with their
current cognitive state, including attention and readiness to learn.

One nice illustration of this effect comes from a study by D. Markant,
DuBrow, Davachi, and Gureckis (2014). In this task, participants
memorized the identities and locations of objects that were hidden in a
grid (adapted from Voss et al. (2011)). D. Markant et al. (2014) varied
the \emph{level} of control across conditions and compared the
performance of active learners to a group of \enquote{yoked}
participants who saw training data that was generated by the active
group. Across conditions, participants could either control: (a) the
next location in the grid, (b) the next item to be revealed, (c) the
duration of each learning trial, and/or (d) the time between learning
trials (i.e., inter-stimulus-interval or ISI). Results showed an active
learning advantage for all levels of control, including the lowest
amount of control in the ISI-only condition. D. Markant et al. (2014)
interpreted these results as providing evidence that active control
allowed people to, \enquote{optimize their experience with respect to
short-term fluctuations in their own motivational or attentional state.}

Developmental studies have extended this work on adults' spatial memory
to 6- to 8-year old children, showing similar advantages for conditions
of active control (Ruggeri, Markant, Gureckis, \& Xu, 2016). Other work
has found similar benefits of active control in word learning
(Partridge, McGovern, Yung, and Kidd (2015); see also Kachergis, Yu, and
Shiffrin (2013) for evidence in adults) and understanding causal
structures (L. Schulz, 2012). Sobel and Kushnir (2006) showed that
learners who designed their own interventions on a causal system learned
better than yoked participants who either passively observed the same
sequence of actions or re-created the same choices made by others.
Moreover, even young infants seem to benefit from active engagement with
the learning environment. For example, Begus, Gliga, and Southgate
(2014) showed that 16-month-old infants show evidence of stronger memory
for information that was provided about an object they had previously
pointed to as opposed to information about an object they had previously
ignored.

Additional evidence that active control enhances attention and memory
comes from research on children's engagement with educational technology
(for a review, see Hirsh-Pasek et al. (2015)). For example, Calvert,
Strong, and Gallagher (2005) exposed preschool-aged children to two
sessions of reading a computer storybook with an adult, and manipulated
whether the adult or the child controlled the mouse and could advance
the story. Children in the adult-control condition showed a decrease in
attention to the storybook materials in the second session; in contrast,
children who were given control over the experience maintained similar
levels of attention across both sessions. Other research shows that when
adults interact with an avatar that is controlled by a real person
rather than a computer, people experience higher levels of arousal,
learn more, and pay more attention (Okita, Bailenson, \& Schwartz,
2008). And work by Roseberry et al. (2014) showed children learned
equally well from interactions with a person in a video chat (e.g.,
Skype) when social contingency was established, but they did not learn
from watching a digital interaction between the adult and another child.

These results parallel the literature on attention/memory effects in
social learning reviewed in Part 1. That is, both active and social
processes can modulate attention and memory to facilitate in-the-moment
learning. However, as in social learning, the effects of active control
operate through multiple mechanisms, going beyond changes in lower-level
cognitive processes to changing the quality of \emph{information} that
learners get from the world.

\subsubsection{\texorpdfstring{Active control provides \enquote{good}
information}{Active control provides good information}}\label{active-control-provides-good-information}

Active learning allows people to gather information that is particularly
\enquote{useful} for their own learning. This benefit relies on the fact
that learners have better access to their own prior knowledge, current
hypotheses, and ability level, which they can leverage to create more
helpful learning contexts (e.g., asking a question about something that
is particularly confusing). Research on this component of active
learning focuses on how learners select actions to create experiences
that are more useful compared to entirely passive contexts where the
learner has less control.

For example, Castro et al. (2009) directly compared adult active and
passive category learning to predictions from statistical learning
theory under conditions of varying difficulty. They found that human
active learning was always superior to passive learning, but did not
reach the performance of the optimal model and the advantage for active
control decreased in the more difficult (i.e., noisier) learning tasks.
Using a similar model-based approach, D. B. Markant and Gureckis (2014)
investigated the effects of active vs.~passive hypothesis testing on the
rate of adults' category learning. They varied the difficulty of the
learning task by testing two different types of category structures: a
rule-based category, which varied along 1 dimension (easier to learn),
and an information-integration category, which varied along 2 dimensions
(harder to learn). In the active condition, the learner chose specific
observations from the category to test his or her beliefs, whereas in
the passive condition, the data were generated randomly by the
experiment. Participants in the active condition learned the category
structure faster and achieved a higher overall accuracy rate compared to
participants in the passive learning condition, but only for the
simpler, rule-based category.

Together, the Castro et al. (2009) and D. B. Markant and Gureckis (2014)
results illustrate several important points about active learning.
First, the quality of active exploration was fundamentally linked to the
learner's understanding of the task: if the representation was poor,
then self-directed learning was biased and ineffective. Second, the
benefits of active control were tied to the aspects of the individual
learner -- i.e., their prior knowledge and the current hypotheses under
consideration -- such that the same sequence of data did not provide
\enquote{good} information for another learner. And third, the benefits
of active learning diminished with increased task difficulty, perhaps
because learners struggled to generate \enquote{helpful} examples.

A recent body of developmental work on children's pointing has begun to
explore how children's actions can change the flow of information to
support their later learning. For example, Wu and Gros-Louis (2015)
found that adults generate a higher number of object labels for objects
that their 12-month-olds pointed to, suggesting that the infants'
pointing elicited information that was more useful for early word
learning (for converging evidence, see Kishimoto, Shizawa, Yasuda,
Hinobayashi, and Minami (2007); Goldin-Meadow, Goodrich, Sauer, and
Iverson (2007); and Olson and Masur (2011)). In addition, experimental
by Begus and Southgate (2012) showed that infants point more in the
presence of a knowledgeable person compared to an incompetent person,
which suggests that children's behavior is driven by a desire to learn
something new as opposed to just share attention with another person.
Finally, infants early pointing behaviors have been directly linked to
their later language learning (Rowe \& Goldin-Meadow, 2009), providing
additional evidence that pointing elicits language that is particulary
helpful for building early vocabulary.

A little later in development, after children develop the requisite
productive language skills, they begin to generate verbal questions. And
there is reason to believe that children are using questions to gather
\enquote{good} information from other people. For example, in a corpus
analysis of four children's parent-child conversations, Chouinard,
Harris, and Maratsos (2007) found that children begin asking questions
early in development (18 months) and at an impressive rate, ranging from
70-198 questions per hour of conversation. Chouinard et al. (2007) also
coded the intent of children's questions, finding that 71\% were for the
purpose of gathering information, as opposed to attention getting or
clarifications. Other corpus analyses provide converging evidence that
question asking is a common behavior in parent-child conversations
(Davis, 1932), that children are seeking knowledge with their questions
(Bova \& Arcidiacono, 2013), and that children will persist in asking
questions if they do not receive a satisfactory explanation (Frazier,
Gelman, \& Wellman, 2009).

Perhaps the best evidence that children are capable of effective
self-directed learning comes from research on children's active
behaviors in causal learning tasks. In these studies, children are often
presented with a novel toy that has some unknown causal structure, and
they are then given the opportunity to design interventions to figure
out how the toy works. For example, Cook, Goodman, and Schulz (2011)
showed preschoolers a device that played music when beads were placed on
top of it. To test whether children would choose interventions that
generated useful information about the causal structure work, they
manipulated the usefulness of different actions children could choose to
test the devide. That is, half of the children were trained to think
that all different types of beads could make the machine work, while the
the other half of children learned that only certain types of beads
could make it go. Next children were given the opportunity to choose
their own beads to try to make the machine play music, and could chose
either: (1) a set of two beads that were stuck together or (2) a set of
two beads that could be separated. Results showed that children who
learned that only \emph{some} of the beads worked were twice as likely
to choose the separable beads to test the device, suggesting that they
were sensitive to the fact that there was information to be gained by
choosing this behavior (i.e., they could pull the beads apart and test
each one independently).

Other work in the domain of causal inference shows that preschoolers can
integrate prior beliefs and evidence to alter how they explore a causal
system to learn something like balance-relations (E. B. Bonawitz,
Schijndel, Friel, \& Schulz, 2012); that when preschoolers see
confounded evidence for how something works, they spend more time
exploring that object (Schulz \& Bonawitz, 2007); that even 8-month-old
infants will selectively explore objects that violate their prior
expectations (Stahl \& Feigenson, 2015); and that children became more
efficient in producing causal interventions, as measured by their
informativeness, as they get older (McCormack, Bramley, Frosch, Patrick,
\& Lagnado, 2016).

Another early behavior that appears sensitive to the value of seeking
certain kinds of information is visual and auditory attention. This
research starts from two assumptions: one that children possess limited
cognitive resources with which to process information from the world,
and (2) that efficient learning is faciliated by attending to
information that is particularly likely to be learned, i.e., useful. For
example, Kidd, Piantadosi, and Aslin (2012) took measured how long 7-
and 8-month-olds visually attended to a monitor that displayed a
sequence of images of familiar objects (e.g., a toy truck). Within each
sequence, infants saw trials that varied along a continuum from low to
high complexity, with the complexity of specific trial defined using a
model-based approach that took into account the prior objects that
infants saw in the sequence and quantified how surprising the current
object was. Infants spent the most time looking on trials of
intermediate complexity, choosing to look away sooner when the object
was either highly predictable or highly surprising based on the prior
sequence of objects. Kidd, Piantadosi, and Aslin (2014) extended these
results to the auditory domain, showing a similar pattern of increased
attention to intermediate complexity for sequences of nonsocial sounds
such as a door closing or a train whistle. One interpretation of these
results is that infants' in-the-moment visual and auditory attention
leverages their prior experiences to focus on specific information that
is likely to be learned

In addition to seeking learnable information, infants also show evidence
of avoiding information that is unlearnable. For example, Gerken,
Balcomb, and Minton (2011) tested whether 17-month-olds would pay more
attention to a stream of input that consisted of a learnable structure
(e.g., Russian feminine words take the endings oj and u, and masculine
words take the endings ya and yem) as opposed to a random stream of
input without any information to extract (e.g., word endings were not
diagnostic of category structure). Results showed that infants were
quicker to disengage attention from the unlearnable information stream,
suggesting that they were doing some form of tracking their own learning
progress, using this estimate to decide to stop gathering information
when learning progress was sufficiently low.

Taken together, the findings reviewed in this section highlight two
points. First, from an early age children are capable of engaging in
behaviors that appear to serve the goal of seeking useful information to
support learning. Second, these findings illustrate the complexity of
studying self-directed learning. That is, many behaviors can be cast as
examples of self-directed learning, and it is not the case that active
learning functions similarly across individuals, contexts, and learning
domains. Moreover, there are a multitude of factors that could modulate
the effectiveness of active learning behaviors, creating a large set of
possibilities for researchers to explore.

One way to get a handle on this complexity is to develop a formal model
of human active learning that abstracts away some detail in order to
gain access to general principles that shape active learning behaviors.
Recently, researchers in both developmental and cognitive psychology
have taken advantage of the development of formal models of scientific
inquiry by statisticians (Optimal Experiment Design or OED models) in
order to select the best experiment from a set of possible experiments
that a researcher could conduct when trying to learn about a new
phenomenon. With this formallization in hand, researchers are able to
make qualitative and quantiative comparisons to people's information
seeking behavior, targeting specific components of the OED models with
experiments and asking when/why people deviate from the optimal active
learning behavior.

\subsubsection{Optimal Experiment Design: A formal account of active
learning}\label{optimal-experiment-design-a-formal-account-of-active-learning}

Optimal Experiment Design (OED) (Emery \& Nenarokomov, 1998; Lindley,
1956; Nelson, 2005) is a statisical framework that attempts to quantify
the \enquote{usefulness} of each experiment within the set of possible
experiments that an experimenter could conduct to improve their
understanding of a phenomenon. The key insight was described by Lindley
(1956) as a transition from viewing the practice of statistics as making
binary decisions about what to do next to the practice of gathering
information in order to sharpen one's understanding of the
\enquote{state of nature} (p.~987), More concretely, he proposed that
experiments should be designed to maximize a measure of expected
information gain (taken from Information Theory and discussed in more
detail below) and that researchers should continue the experiment until
some pre-determined threshold of information is reached.

The benefit of using the OED approach is that it allows scientists to
make design choices that maximize the effectiveness of their
experiments, thereby reducing inefficiency and the costs related to
additional experimentation. For example, Nelson, McKenzie, Cottrell, and
Sejnowski (2010) used OED principles to differentiate competing theories
of information seeking during adults' category learning. That is, they
wrote down an OED model of their category learning task (classifying a
set of images into one of two categories), the possible design choices
(e.g., what combination of features to show participants), and the
relevant behavioral hypotheses (i.e., different theories of category
learning). And using this model, they were able to figure out the
feature combinations for their stimuli on whichthe competing theories
made different predictions, allowing them to

A growing body of psychological research has used the OED framework as a
metaphor for human active learning. The idea is that when people make
decisions about how to act on the world, they are engaging in a similar
process of evaluating the \enquote{goodness} of these different actions
(i.e., experiments) relative to some learning goal, and in turn, select
behaviors that maximize the potential for gaining information about the
world. One of the major successes of the OED model is that it can be
used to account for a wide range of information seeking behaviors,
including verbal question asking (Ruggeri \& Lombrozo, 2015), planning
interventions in causal learning tasks (Cook et al., 2011), and
decisions about visual fixations during scene understanding (Najemnik \&
Geisler, 2005).

Coenen, Nelson, and Gureckis (2017) provide a thorough review of the OED
framework and its links to research on the psychology of human
information seeking. In their review, they lay out the four critical
parts of an OED model: 1) a set of hypotheses, 2) a set of questions
(i.e., actions) to learn about the hypotheses, 3) a way to model the
types of answers that each question could elicit, and 4) a way to score
each of the possible answers with respect to some usefulness metric. In
addition to these components, they highlight the importance of learners'
inquiry goal (e.g., \enquote{What's that object called?}) for engaging
in OED-like reasoning. The key point is that without a clear goal, then
it becomes difficult to instantiate the hypotheses, questions, and
answers that a learner should consider when deciding how to act. Next, I
provide the mathematical details of the OED approach as described in
Coenen et al. (2017). The goal of laying out the formal model in detail
is to provide a structure for Part III where I discuss how social
learning contexts can intervene on the different components of the OED
model.

Together, the pieces of an OED model allow researchers to quantify the
\emph{expected utility} of a particular information seeking behavior
such as a question \(EU(Q)\)) amongst a set of questions that a person
could ask \(Q_1, Q_2,..., Q_n = \{Q\}\). The expected utility is a
function of two factors: (1) the probability of obtaining a specific
answer to a question \(P(a)\) weighted by (2) the usefulness of that
answer for achieving the learner's goal \(U(a)\). Taken together, we can
define the expected utility for a specific question \(Q\) as the sum of
all utilities for all the possible answers to that question.

\[EU(Q) = \sum_{a\in q}{P(a)U(a)}\]

There are a variety of ways to define the usefulness function to score
each answer. An exhaustive review is beyond the scope of this paper, but
for a detailed analysis of different approaches to modelling the
usefulness of actions with respect to information seeking, see Nelson
(2005). One common approach is to use \emph{information gain} defined as
the change in the learner's overall uncertainty before and after
receiving an answer. One way to instanstiate this calculation is to take
the change from prior to posterior entropy (i.e., uncertainty) after
getting a particular answer.

\[P(h|a) = ent(H) - ent(H|a)\]

Where the prior entropy \(ent(H)\) can be defined as Shannon entropy,
which provides a measure of the overall amount of uncertainty in the
learner's beliefs about the candidate hypotheses.

\[ent(H) = -\sum_{h\in H}{P(h)logP(h)}\] The posterior entropy
\(ent(H|a)\) is the entropy conditioned on a specific answer.

\[ ent(H|a) = -\sum_{h\in H}{P(h|a)logP(h|a)} \] And the \(P(h|a)\) can
be calculated using Bayes rule.

\[ P(h|a) = \frac{P(h)P(a|h)}{P(a)}  \]

If all pieces of and OED model are defined (hypotheses, questions,
answers, and the usefulness function), then selecting the optimal
question is straightforward. All the learner must do is perform the
expected utility computation for each question in the set of possible
questions and pick the one that maximizes utility, or is most useful for
their learning goal. In practice, the learner must consider each
possible answer, score the answer using the usefulness function, and
weight the score using the probability of getting that answer.

There are several benefits of the OED formalization for understanding
human active learning. First, it makes researchers define the different
components of an active learning problem, thus making their assumptions
about the phenomenon more explicit. Second, if researchers can develop
an OED model, then they can ask whether people's behavior matches or
deviates from the optimal behavior predicted by the model. Finally,
casting information seeking as rational \emph{choice} links psychology
with several rich literatures (economics, statistics, computer science)
that have attempted to formalize the decision-making process as a
process of utility analysis that can include both the costs of
information acquisition and the benefits of choosing a particular
behavior.

One nice demonstration of this approach comes from Nelson (2005) model
of eye movements during novel concept learning. The model combines
Bayesian probabilistic learning, which represents the learner's current
knowledge as a probability distribution over a concept, with an OED
model of the usefulness of a particular eye movement (modeled as a type
of question-asking behavior) for gathering additional information about
the target concept from the visual world. Together, these model
components allowed Nelson (2005) to predict changes in the pattern of
eye movements at different time points in the learning task.
Specifically, they found that early in learning, when the concepts were
unfamiliar, the model predicted a wider, less efficient distribution of
fixations to all candidate features that could be used to categorize the
stimulus. However, after the model learned the target concepts, eye
movement patterns shifted, becoming more efficient and focusing on a
single stimulus dimension.

Another promising aspect of the OED models is that recent developmental
work has provided evidence that even young children appear to select
behaviors that efficiently maximize learning goals. Experimental work
has investigated the quality of children's question asking by measuring
the quality of questions in constrained problem-solving tasks. For
example, Legare, Mills, Souza, Plummer, and Yasskin (2013) used a
modified question asking game where 4- to 6-year-old children saw 16
cards with a drawing of an animal on them. The animals varied along
several dimensions, including type, size, and pattern on the animal. The
child's task was to ask the experimenter yes-no questions in order to
figure out which animal card the experimenter had hidden in a special
box. Legare et al. (2013) coded whether children asked
constraint-seeking questions that narrowed the set of possible cards by
increasing knowledge of a particular dimension or dimensions (e.g.,
\enquote{Is it red?}), confirmatory questions that provided redundant
information, or ineffective questions that did not provide any useful
information (e.g., \enquote{Does it have a tail?}). Results showed that
all age groups asked a higher proportion of the effective,
constraint-seeking questions relative to the other question types, and
that the number of constraint-seeking questions was correlated with
children's accuracy in guessing the identity of the card hidden in the
special box. Legare et al. (2013) interpret these results as evidence
that children can use questions to solve problems in a efficient manner.
Converging evidence in support of this interpretation comes from
experimental work using this approach finding that children prefer to
direct questions to someone who is knowledgeable compared to someone who
is inaccurate or ignorant (Mills, Legare, Bills, \& Mejias, 2010; Mills,
Legare, Grant, \& Landrum, 2011),

Although the OED approach has provided a formal account of seemingly
unconstrained information seeking, there are several ways in which it
falls short as an explanation of human self-directed learning. Coenen et
al. (2017) argue that in practice OED models make several critical
assumptions about the learner and the problem, including the
hypotheses/questions/answers under consideration, that people are
actually engaging in some kind of expected utility computation in order
to maximize the goal of knowledge acquisition, and that the learner has
sufficient cognitive capacities to carry out the computations.

In the next section, I argue that limitations of the OED approach can be
productively reconstrued as opportunities for understanding how learning
from other people can scaffold self-directed learning. I focus on
integrating social learning with five key components of the OED model:
learners' inquiry goals, hypotheses, questions, answers, and stopping
rules. The key insight is that learning from more knowledgable others
provides the building blocks that are required for children to engage in
effective self-directed learning.

\section{Part III: Active learning within social
contexts}\label{part-iii-active-learning-within-social-contexts}

Why should we attempt to integrate social and active learning accounts?
First, children do not re-invent knowledge of the world, and while they
can learn a tremendous amount from their own behaviors, much of their
generalization and abstraction is shaped by input from other people.
Moreover, social learning can sometimes be the only way to learn
something and sometimes it can be a faster or more efficient route for
learning. Finally, children are often surrounded by parents, other
knowledgeable adults, and older peers -- all of whom may know more about
the world than they do, creating contexts where the opportunity for
social learning is ubiquitous.

In addition, there is a body of empirical work showing that active
learning can be biased and ineffective in systematic ways. For example,
work by Klahr and Nigam (2004) showed that elementary school-aged
children were less effective at discovering the principles of
well-controlled experiments from their own self-directed learning, but
were capaple of learning these principles from from direct instruction.
D. B. Markant and Gureckis (2014) showed that active exploration
provided no benefit over passive input in an abstract category learning
task when there was a mismatch between the target concept and adults'
prior hypotheses going into the learning task. And McCormack et al.
(2016) found that 6-7 year-olds showed no learning benefits when allowed
to actively intervene on a causal system compared to observing another
person performing the interventions, which the authors suggest might be
due to the relative decrease in cognitive demands in the observational
learning condition.

In a comprehensive review of the self-directed learning literature,
Gureckis and Markant (2012) point out that the quality of active
exploration is linked to aspects of the learner's understanding of the
task: if the representation is poor, then self-directed learning will be
biased and ineffective. Coenen et al. (2017) go a step further and
outline the challenges of demonstrating efficient active learning
behaviors. Might social learning accounts have something to say in
addressing the mixed results and challenges in the self-directed
learning literature?

This section proposes one potential solution to the challenge of
characterizing children's learning as efficient information seeking
guided by OED principles. I take the OED model outlined in Coenen et al.
(2017) as a starting point for defining efficient inquiry behavior, and
use it to integrate social and active learning accounts. The benefit of
this formal framework of self-directed learning is that it makes the
different components of active learning explicit and highlights aspects
that might be particularly challenging for young learners with limited
cognitive resources.

I propose that we can reconstrue the \enquote{limitations} of the OED
account as a model of human learning as opportunities for understanding
how social contexts (i.e., interactions with more knowledable others)
can support information seeking. In each section, I highlight the
developmental challenge for each component of the OED model and then
discuss how features of the social learning context could influence play
a role. I also highlight prior work that highlight the potntial for
social contexts to shape self-directed learning and point to
interesting, open questions that are promising areas for future work.

\begin{table}[h]
\begin{center}
\begin{threeparttable}
\caption{\label{tab:oed_table}How features of social learning contexts map onto different components of the OED account.}
\begin{tabular}{lllll}
\toprule
OED model component & \multicolumn{1}{c}{Description} & \multicolumn{1}{c}{Developmental challenge} & \multicolumn{1}{c}{Effect of social context} & \multicolumn{1}{c}{Relevant literatures}\\
\midrule
Inquiry goals & targets of information seeking behaviors. "What should I learn?" & Where do inquiry goals come from? How do children select which inquiry goal to pursue? & Initialize and/or constrain the inquiry goals under consideration. Introduce social goals (i.e., reputation management) & Zone of Proximal Development; Guided Play; Guided Participation; Performance vs. learning goals\\
Hypotheses & beliefs about the world. "this toy is a dax" & Where do hypotheses come from? How to select likely hypotheses? & Reduce the space of possible hypotheses & NA\\
Questions & actions that gather information about hypotheses. "Is that a dax?" & How to generate good questions? & model useful questions and provide question templates & NA\\
Answers & outcomes that occur in response to questions. "yes, that's a dax." & How to find good answers? & provide answers that contain generalizable information & NA\\
Stopping rules & decisions to stop collecting information & How to know when to stop collecting information? & constrain exploration. Encourage persistence & NA\\
\bottomrule
\end{tabular}
\end{threeparttable}
\end{center}
\end{table}

\subsubsection{Inquiry goals}\label{inquiry-goals}

An inquiry goal refers to the underlying motivation for people's
information seeking behaviors. Often this is simply defined as a search
for the correct hypothesis amongst a set of candidate hypotheses.
Intuitively, an inquiry goal is what drives people to learn. Some
examples of plausible inquiry goals that span across a variety of
learning tasks and developmental research literatures include:

\begin{itemize}
\tightlist
\item
  Is this person a reliable source of information? (selective learning)
\item
  What is this speaker referring to? (referential uncertainty)
\item
  What types of objects are called \enquote{daxes}? (category learning)
\item
  How does this toy work? (causal learning)
\item
  Where should I look next? (allocation of visual attention)
\end{itemize}

Without an explicit inquiry goal, it becomes difficult for an active
learner to compare the quality of different behaviors since the learner
cannot evaluate how an action will lead to learning progress. Coenen et
al. (2017) illustrate this point by highlighting how researchers often
go to great lengths to communicate the specific inquiry goal of an
experimental task, saying:

\begin{quote}
\enquote{The importance of such goals is made clear by the fact that in
experiments designed to evaluate OED principles, participants are
usually instructed on the goal of a task and are often incentivized by
some monetary reward tied to achieving participants that goal.
Similarly, in developmental studies, children are often explicitly asked
to answer certain questions, solve a particular problem, or choose
between a set of actions.} (p.~32-33)
\end{quote}

Thus, characterizing children's goals during learning becomes critical
for evaluating the quality of self-directed learning behaviors. However,
this is no easy task since children could be considering a wide range of
goals during any moment and there is no guarantee that learning progress
should be one of the goals. In fact, one line of theorizing about OED as
a model of human inquiry argues that we should only expect to see
efficient information seeking behaviors in contexts where there is a
clear task and learning goal. For example, when a parent gives their
child a new toy with several buttons on it and the child's goal is to
figure out how to make it work. In this case, we could ask whether the
child approaches the learning task in an efficient way, selecting
actions that are most likely to provide useful information about the
toy's causal structure.

In the previous example, we see the potential for children's
interactions with more knowledable others to play a role in triggering
inquiry behavior. That is, adults and older peers have the capacity to
construct contexts with clear learning goals in order to support
children's own information seeking. This connection draws on influential
ideas in cognitive development that frame social learning as a form of
scaffolding where children are placed in contexts that present something
new to be learned but importantly conatin learning goals that are
achieveable given children's current capabilties (e.g., Zone of Proximal
Development (Vygotsky, 1987), Rogoff's theory of Guided Participation
(Rogoff et al., 1993), and more recently Guided Play (Weisberg,
Hirsh-Pasek, \& Golinkoff, 2013)).

For example, Weisberg et al. (2013) define guided play as an
intermediate learning context that falls between totally unstructured
free play and constrained direct instruction. The boundaries between
these contexts are difficult to define, but the critical dimension is
the level of control that the more knowledgable participant exerts over
the activity. In free play, the child decides what to do next; whereas
in direct instruction, the more knowledgable person explicitly tells the
learner what to do, asks questions, or demonstrates new concepts.

In their review, they present the following example to illustrate the
key difference between guided play and direct ins

\begin{quote}
For example, a teacher with the goal of teaching new vocabulary words
could take a direct instruction approach, by telling children the
meanings of the new words they encounter in a storybook or by showing
examples: \enquote{This is a helmet. A helmet goes on your head to stop
your head from getting hurt if you fall off your bike.} Or, she could
take a guided-play approach, introducing the new words in the context of
a child's play episode while encouraging children to think broadly about
the word's meaning: \enquote{She's got a helmet on while riding her
bike. What do you think would happen if she fell off her bike and wasn't
wearing her helmet?} (p.~106)
\end{quote}

While these contexts appear quite similar, the key difference is whether
the child initiated the activity. Weisberg et al. (2013) hypothesize
that guided play context provides the right combination of structure
with the opportuntity for children's to exercise self-efficacy over
learning, and leads to better learning outcomes.

One less-emphasized feature of the Guided Play proposal is the
importance of an adult initializing a clear learning goal for the
activity. In the previous example, if we removed the adult and only
provided the child with a storybook to explore, it is unclear what goals
the child would pursue when deciding what actions to take. However, the
very presence of an adult who has knowledge of the names of objects in
the book and has the goal to teach those names to the child changes the
potential for the child to engage in informations-seeking behaviors.
More formally, we can connect this example to the OED framework of
self-directed learning, and it becomes clear that one potentially
important role of more knowledable others is to present children with a
clear learning goal, which in turn sets the stage for children to reason
about what actions to take next (e.g., what question to ask or what
object to point at) and how those actions might best support the current
learning goal (i.e., learning the names of objects in the storybook).

Another interesting connection between social contexts and children's
inquiry goals comes from a body of work exploring how children's input
shapes their implicit theories of intelligence and in turn influences
the goals they choose to pursuse in novel learning contexts (Dweck \&
Leggett, 1988). Specifically, implicit theories of intelligence refer to
children's internal working models of the world and provide general
frameworks for processing information and generating predictions about
behavior. Dweck and Leggett (1988) propose a causal model where implicit
theories of intelligence cause different goal orientations, which
interact with perceptions of present ability to generate different
behaviors. For example, if children hold a belief that intelligence is
malleable (an incremental theory), they will want to increase competence
(select a learning goal) and therefore be more likely to select tasks
that support learning (mastery-oriented).

Empirical work has shown that children can be oriented toward learning
goals by an experimenter. For example, Elliott and Dweck (1988) directly
manipulated elementary school-aged children's goals by presenting them
with a choice between one of two tasks described in the following ways:

\begin{itemize}
\tightlist
\item
  Performance task. In this box we have problems of different levels.
  Some are hard, some are easier. If you pick this box, although you
  won't learn new things, it will really show me what kids can do.
\item
  Learning task. If you pick the task in this box, you'll probably learn
  a lot of new things. But you'll probably make a bunch of mistakes, get
  a little confused, maybe feel a little dumb at times --- but
  eventually you'll learn some useful things.
\end{itemize}

Elliott and Dweck (1988) found that when children were oriented towards
the learning goal, they tended to choose the more difficult
\enquote{learning} task even though they were likely to make mistakes
and risk looking incompetent. In another study, Dweck and Leggett (1988)
showed that children who already held performance goals viewed effort on
a task as an index of ability, whereas children with learning goals view
effort as a means for improvement. Morever, both lab-based experiments
and observational work provide evidence that the language adults choose
to use when praising chidren can shape how likely children are to hold
implicit theories that emphasize learning over and above performance
goals (Cimpian, Arce, Markman, \& Dweck, 2007; Gunderson et al., 2013).

Taken together, the research on implicit theories suggests another
pathway through which social contexts can trigger inquiry goals. We can
recast the Elliott and Dweck (1988) finding -- that children oriented
toward learning goals select harder tasks -- in terms of the OED
framwork. That is, the goal manipulation is an instance of the social
context initializing an inquiry goal, which in turn influences
children's decision making, leading children to select behaviors that
result in a higher chance of learning new concepts, an explicit
prediction that falls out of the OED model of human inquiry.

One important gap in the literature on inquiry goals is a good estimate
of how often children participate in contexts with clear learning goals
in their daily lives, as opposed to contexts where learning goals are
absent or contexts where children are clearly pursuing other
non-learning related goals (e.g., some example here). Moreover, there is
a need for more research on the kinds of events that lead children to
generate inquiry goals. However, Rogoff et al. (1993)'s work on
\emph{Guided participation} provides an interesting counter-example
where they coded the rate of \enquote{caregiver orienting} behaviors in
parent-child interactions with their 12- to 24-month-old infants across
four different cultural communities (a Mayan Indian town in Guatemala, a
middle-class urban group in the United States, a tribal village in
India, and a middle-class urban neighborhood in Turkey) that varied
along the dimensions of how separated children were from adult
activities and whether formal schooling was emphasized. Specifically,
caregiver orienting was defined as doing the following behavior during
parent-child interaction around a set of novel toys,

\begin{quote}
Caregiver orients child involved introducing new information or
structure to the child (at any point in the episode) regarding the
overall goals or a key part of the event or what was expected in the
situation. Orienting framed a major goal, not just specific little
directives for particular actions.
\end{quote}

Rogoff et al. (1993) found that parents in all four communities produced
high rates of structuring and orienting behaviors (with the lowest rate
of structuring being 81\% of play episodes). Thus, when placed in a
structured activity, adults make sure children are aware of the goal
(e.g., learning the function of the novel toy). However, the communities
differed in how often children were directly involved with adult
activities in day-to-day life, with the children raised in rural
villages often having early acces to adult economic and social
activities. An interesting open question is whether older peers and
adults need to be directly engaging with the child in order to trigger
inquiry goals and efficient self-directed learning behaviors. Perhaps
increased access to observing lots of adult goal-directed behaviors can
faciliate children to generate leanring goals, for example as they see
activies being completed that they do not understand.

One important direction for future research to map the space of
children's goals during everyday learning contexts. It would be
interesting to know the proportion of children's daily activities that
involve contexts where there is a clear learning goal either being
demonstrated by the child or by older peers and adults. It would also be
useful to know how the distribution of these tasks changes as a fnction
of development, especially as children enter school and across different
cultural contexts where children have differential access to structured
(e.g., lessons and sports) vs.~unstructured activies (e.g., free play).

\subsubsection{Hypotheses}\label{hypotheses}

After establishing that there is something to be learned, the next key
component of inquiry is deciding what hypotheses should be considered
and tested. Intuitively, a hypothesis is a candidate explanation about
how the world works. For example, if a child is in a concrete word
learning context -- i.e., she hears a new word (\enquote{dax}) and is
surrounded by a three unfamiliar objects (A, B, and C) -- then she might
entertain at least\footnote{This hypothesis space only considers
  one-to-one mappings.} the following hypotheses about the meaning of
dax: dax = A, dax = B, or dax = C.

The set of hypotheses under consideration is critical for measuring
effective self-directed learning. The usefulness function outlined in
Part II works by comparing the learner's uncertainty over hypotheses
before and after the she performs some action on the world. Without
knowing what is in the hypothesis space, it becomes challenging to
figure out the best action for reducing uncertainty. Put another way,
the OED framework does not easily deal with situations where learners
might have to consider a large space of hypotheses, might actually hold
the wrong hypotheses, or perform actions without considering any
hypotheses at all. This scenarios seem plausible for young learners and
thus present a challenge to using OED principles as a model of early
active leanring.

However, one important functions of social learning contexts is to
provide a clear set of possible explanations for the true state of the
world. That is, adults and older peers, who might have access to the
correct hypothesis, can restrict the hypothesis space in order to help
guide children information seeking behaviors. The effect of social
contexts on hypotheses parallels the effect on goals reviewed in the
previous section: that the behaviors of other people have the capacity
to initialize and constrain.

One relevant case study of the capacity for social contexts to constrain
hypotheses comes from work on children's early word learning. The
challenge for the young word learner is that even the simplest of words,
concrete nouns, are often used in complex contexts with multiple
possible referents, which in turn have many conceptually natural
properties that a speaker could talk about. This creates the potential
for an (in principle) unlimited amount of hypotheses that children could
consider for the meaning a novel word. Remarkably, word learning
proceeds despite this massive uncertainty, with estimates of adult
vocabularies ranging between 50,000 to 100,000 distinct words (P. Bloom,
2002).

It does not seem plausible for children to entertain all hypotheses
about possible word-object links. But how might children constrain the
hypotheses that they consider? One proposed solution is for word
learners to only consider a single word-object hypothesis at a time
(Medina, Snedeker, Trueswell, \& Gleitman, 2011; Trueswell, Medina,
Hafri, \& Gleitman, 2013). That is, the child could make an initial
guess about the meaning of a new word, and then only consider that guess
until she receives sufficient evidence that her initial hypothesis was
incorrect. If she seees sufficient counter-evidence, then she will
switch to a new hypothesis that better matches the statistics in the
input.\footnote{This \enquote{propose-but-verify} account parallels work
  by E. Bonawitz, Denison, Gopnik, and Griffiths (2014) in the domain of
  causal learning, which suggests that a \enquote{Win-Stay, Lose-Sample}
  algorithm (inspired by efficient sampling procedures in computer
  science) provides a better explanation of children's hypothesis
  testing behaviors compared to an algorithm that enumerates the entire
  hypothesis space.} Another influential account of early word learning,
inspired by basic associative learning principles, argues that children
store more than a single hypothesis, suggesting that the hypothesis
space is gradually reduced via the aggregation of word-object labels
across multiple labeling events (Siskind, 1996; C. Yu \& Smith, 2012b).
Support for this experimental work has shown that both adults and young
infants can use word-object co-occurrence statistics to learn word
meaning from individually ambiguous naming events (Smith \& Yu, 2008).
Moreover, adults show evidence of being able to recall multiple
word-object links from an initial naming event (Yurovsky \& Frank,
2015).

The key difference between these proposals is how much information
learners store in their hypothesis space. Understading the nature of the
hypotheses that learners consider is critical for evaluating children's
ability to seek information with respect to those hypotheses. Some of
our own work provides direct evidence that the social context of
language learning can modulate the content of the learner's hypothesis
space (MacDonald, Yurovsky, \& Frank, 2017). Inspired by ideas from
Social-pragmatic theories of language acquisition that emphasize the
importance of social cues for word learning (P. Bloom, 2002; Clark,
2009; Hollich et al., 2000), we showed adults a series of word learning
contexts that varied in ambiguity depending on whether there was a
useful social cue to reference (i.e., a gaze cue). We then measured
learners' memory for alternative word-object links at different levels
of attention and memory demands. Results showed that learners flexibly
responded to the amount of ambiguity in the input, and as uncertainty
increased, learners tended to store more word-object links. Morever, we
found that learners stored representations with different levels of
fidelity as a function of the reliability of the social cue and despite
having the same amount of time to visually explore the objects during
the initial labeling event.

These results suggest that the content of learners' hypothesis spaces
changed as a function of the quality of the social learning context.
Further suppport for this idea comes from experimental work showing that
even children as young as 16 months prefer to map novel words to objects
that are the target of a speaker's gaze and not their own (D. A.
Baldwin, 1993), and analyses of naturalistic parent-child labeling
events shows that young learners tended to retain labels that were
accompanied by clear referential cues, which served to make a single
object dominant in the visual field (C. Yu \& Smith, 2012a). One
important direction for future research is to measure the full causal
pathway from variation in social learning contexts to the nature of
children's hypothesis spaces and their information seeking behaviors.
For example, it would be interesting to know whether learners'
subsequent information seeking behaviors would be affected by social
context manipulations like the ones used in our task.

\subsubsection{Questions}\label{questions}

Questions in the OED framework refer to the experiments that a scientist
can conduct in order to gather information with respect to their
hypotheses about the world. When we consider \enquote{questions} in
human informaion seeking, it's important to note that
\enquote{questions} can map onto a range of information seeking
behaviors, such as verbal questions, pushing a button to figure out how
a toy works, and decisions about where to look. In fact, the capacity to
provide general principles to explain such a broad range of behaviors is
one of the strengths of the OED account as a model of human active
learning.

The challenge for the young learner is to discover what behaviors are
available and of those behaviors which might be particularly good for
gathering information to support learning. In this section, I illustrate
how the social learning context provides the input to this learning
process via demonstrations of the range of actions that learners could
take and by adults' modeling effective information seeking behaviors.

It seems obvious that children would look to older peers or adults to
learn what actions are possible and useful. However, a large body of
empirical work suggests that even young infants will not imitate every
action that they see. Instead, children show evidence of
\enquote{rational imitation} and look for cues about other people's
goal-directed behaviors and use this information to determine what
behaviors are worth imitating . For example, Gergely, Bekkering, and
Király (2002) measured how often 14-month-old infants imitated an
adult's inefficient action -- turning on a light with her head (less
efficient) instead of her hands (more efficient) -- as a function of
whether there was a relevant explanation for selecting the less
efficient action (whether the adult's hands were occupied). They found a
large difference in imitation rates across conditions (69\% in the
hands-free vs.~21\% in the hands-occupied), suggesting that children
recognized the reason for the inefficient action and chose to ignore the
means and focus on the goal of turning on the light in the most
efficient way possible.

The high rates of imitation in the hands-free condition highlight
another important compoenent of learning from others' actions: that
children tend to overimitate behaviors even when these actions are not
directly relevant to the task. For example, Call, Carpenter, and
Tomasello (2005) compared imitation behaviors of 2-year-old children
after they watched someone demonstrate how to open a tube using only the
necessary actions or using the actions and a style component that was
unrelated to opening the tube (e.g., removing the tube's cap with an
exaggerated twisting motion). Children imitated the causally irrelevant
action at a high rate (93\% of children), providing evidence that they
were focused on reproducing each of the experimenter's actions and not
just reproducing the outcome of opening the tube.

Other empirical work provides insight into the importance of considering
the social factors that influence whether children choose to imitate.
Carpenter, Akhtar, and Tomasello (1998) showed that 14- and
18-month-olds were less likely to imitate adults' action if the action
was accompanied by a verbal cue that flagged the action as a mistake
(e.g., \enquote{Whoops!}). Buchsbaum, Gopnik, Griffiths, and Shafto
(2011) provide evidence that the children are more likely to overimitate
when the adult is described as a \enquote{knowledgeable} teacher as
oppossed to \enquote{naive.} And Carpenter, Call, and Tomasello (2002)
showed that giving children explicit information about another person's
goals prior to a causal demonstration leads to an increase in imitation
and learning of the correct casual structure.

Taken together, the work on children's learning via imitiation and their
tendency to overimitate suggests that inferences about others'
intentions plays a critical role in the actions that children will use
in their own behaviors. For the purpose of this paper, this work
provides a way forward for understanding the origin of the information
seeking behaviors that might be available to children, i.e., what are
the available actions that a learner could take to gather information.

One domain where progress has been made in understanding how social
contexts directly shape children's information seeking capacities is
verbal question asking. Consider that in order to ask a useful question
in natural language, children must possess the requisite language
skills, which are learned from their language input. Both experimental
work and corpus analyses provide evidence that children's
question-asking becomes more varied and effective over the first years
of life (e.g., see Chouinard et al. (2007) and Legare et al. (2013)
reviewed in Part II). Moreover, children improve in the timing of their
turn-taking during question-answer exchanges, reducing the length of
gaps between turns (Casillas, 2014). Interestingly, ({\textbf{???}})
also found that adults appeared to be sensitive to children's developing
question-answering skills by asking more difficult questions (i.e.,
questions that required more complex answers) as children to older
children and by modifying questions that appeared to confuse children
(e.g., \enquote{Who is this? What's he called? Who is he? What is his
name?}). It is interesting to consider how children might internalize
these modifications as part of their own question asking behaviors.

The majority of the work on children's question asking has focused on
aspects of the child's behavior, exploring how the type, content, and
effectiveness of questions changes as children develop. However, several
studies have measured aspects of caregivers question asking. For
example, B. Yu Yue (2017) coded parent-child interactions from the
CHILDES database to measure the amount of \enquote{pedagogical}
questions in children's input. They differentiate \enquote{information
seeking} questions from \enquote{pedagogical} questions, by coding
whether the adult already knew the answer (e.g., \enquote{What's that
called?}\enquote{;}What does this button do?" vs. \enquote{What did you
do at school?}), and interpreted the goal of the pedagogical questions
as helping the child learn. Results showed that approximately 30\% of
parents' questions were pedagogical, 60\% were information seeking, and
10\% were rhetorical (i.e., not intended to be answered verbally).
Parents also directed a smaller proportion of pedagogical questions to
older children.

More experimental work linking adults' question-asking practices to
children's behaviors is needed. This is especially interesing since
observational studies have found that that parents' use of wh-questions
predicts children's later vocabulary and verbal reasoning outcomes
(Rowe, Leech, \& Cabrera, 2017) and children of parents who were trained
to ask \enquote{good} questions during bookreading episodes at home also
asked better questions during bookreading sessions at school (Birbili \&
Karagiorgou, 2009). One explanation for these associations is that
wh-questions challenge children to produce more complex verbal responses
that in turn builds verbal abilities. However, another interesting
possibilty is that the frequency and type of questions that parents ask
serve as models that could shape children's information seeking
abilities by providing templates for useful behaviors.

Generating a set of questions represents the first step in efficient
information seeking. Next, children have to evaluate the relative
\enquote{goodness} (i.e., utility) of different behaviors. But how do
children learn the features of a good question? One solution is to
observe other people's question asking, recognize which questions are
useful, and imitate those behaviors.

In fact, there is evidence from work with adults that shows a large
difference between people's question-generating (harder) and
question-evaluation (easier) skills. For example, Rothe, Lake, and
Gureckis (2015) asked a group of adults to play a modified version of
the game \enquote{Battleship} where they had to find the location of
three ships that considted of 2-4 tiles and could oriented in either the
vertical or horizontal direction on a 6x6 grid. Participants gathering
information sequentially by uncovering one tile at a time. At different
points in the task, the game would stop and participants were given the
opportunity to ask any question using natural language. Rothe et al.
(2015) used a formal model of the expected information gain of each
question (i.e., the expected reduction in uncertainty after getting the
answer) to evaluate the quality of adults' free-form questions. Results
showed that people rarely produced high information value questions.
However, in a follow-up experiment Rothe et al. (2015) had a different
group of adults play the same game, but this time they provided the list
of questions generated by participantsin the free-form version, and in
this contexts, adults were quite good at selecting high information
value questions.

Developmental work provides evidence of the same generation-recognition
asymmetry. First, experimental work has shown that children younger than
the age of three have difficulty generating appropriate verbal questions
compared to their older peers in \enquote{Twenty Questions} task
designed to measure question-asking skill (Mills et al., 2010, 2011).
However, when Mills, Danovitch, Grant, and Elashi (2012) tested 3- to
5-year-old's capacity to learn from observing third-party
question-answer exchanges. They found that even the youngest children
were capable of using information elicited by others' yes/no questions
to identify the contents of a box. Interestingly, children
differentiated the usefulness of others' question-answer exchanges,
paying more attention to them as compared to exchanges of irrelevant
information.

Together, these results suggest that even at an age where generating
questions \enquote{from scratch} might be difficult, children can
observe and learn from questions that occur in their social environment.
Mills et al. (2011) also explored this phenomenon by directly
manipulating whether children were exposed to a training phase where
adults modeled effective questions prior to playing the question asking
game. They found that even though children were not successful at
constructing good questions on their own, they were able to ask
effective questions at much higher rate following explicit modeling.

Work with elementary-school-aged children in the domain of scientific
inquiry also shows that generating a good question is a challenging
aspect of inquiry skills. One particular relevant example come from Kuhn
and Pease (2008)'s 3-year intervention study that compared children who
were directly trained on inquiry skills (e.g., understanding the
objectives of inquiry and identifying questions) to a group of slightly
older students who had not participated in the training. Children in the
training group showed progress in the skills, but children in the
comparison group failed to develop these skills in the absence of the
particular kinds of input. Summarizing one of the key results,

\begin{quote}
Consistent with the findings of Kuhn and Dean (2005), identifying a
question appears to play a key role in making the rest of the inquiry
cycle productive. In the probabilistic version of Ocean Voyage in year
2, for example, students floundered until they were helped to formulate
a specific research question. Like other components of the inquiry
process, this skill is not one a student learns once and has mastered.
\end{quote}

A final example of how social contexts change the set of possible
questions is the very fact of having other people around adds a social
\enquote{target} for information seeking behaviors. This is in contrast
to the effects of social context discussed until this point that serve
as inputs that shape subsequent information seeking behaviors.
Intuitively, if a child is trying to learn how a toy works, they could
try actions to test the system directly (i.e., seek information directly
from the world). But if another perosn is present, then they can choose
to ask questions or seek help via noverbal means (i.e., seek information
directly from another person). Thus, the very presence of another person
modifies the choices that are available to the learner.

Recent empirical work has begun to explore the factors that affect
children's decisions about whether to seek information from the world or
from other people. For example, ({\textbf{???}}) had 4- to 6-year-old
children decide how toto learn about a novel social category:
\enquote{moozles.} The concept was either visible (e.g., the color of
hair) or invisible (e.g., an internal preference). Children were given a
choice of looking directly at the moozle or asking a moozle expert.
Six-year-olds chose to look for a visible property and to ask for an
invisible property above chance in both experiments; whereas
four-year-olds behavior was a bit noisier but showed a preference for
the rational looking behavior with additional cueing in a follow-up
experiment.

Other relevant examples comes from work on children's help-seeking
behavior. ({\textbf{???}}) had children build toys that required
multiple steps, and on each each step children were given the
opportunity to ask for help from the experimenter. Each step varied in
difficulty and children naturally varied in their toy building skill.
Children asked for help when the step was harder, and less competent
children asked for help more often, suggesting that preschoolers sought
help in a systematic way -- when they needed it and not when they
didn't. Moreover, work by ({\textbf{???}}) found that 16-month-old
infants are selective in help-seeking, turning to a social target to
seek information or acting directly on the world depending on which
information source ws more likely to help them reach their current goal.
In this case, the infants' goal was to make a malfunctioning toy produce
music and the critical manipulation was whether children saw evidence
that explained the likely cause of failure being the toy versus their
capacity for making the toy play music. When the toy was likely to be
broken, they reached for a new object (queried the world), but in
contrast, when the evidence suggested that the child was the issue, then
they sought help from a nearby adult.

Some of our own work has explored how the presence of another person
might change the dynamics of children's decisions about where to look
during familiar language comprehension.

In sum, the set of questions that children consider provide the tools in
their information seeking toolkit. However, we need more research to
understand how children generate possible questions. One possible
explanation explored in this section is that children might use their
powerful imitiative learning skills to model the question-asking
behaviors demonstrated by more knowlegable others present in their.
Moreover, social contexts fundamentally change the set of behaviors that
are available to the developing learner by providing a social target for
information seeking behaviors.

\subsubsection{Answers}\label{answers}

Answers in the OED model refer to

\begin{itemize}
\item
  I think the pedagogical inference work fits in here. We get more
  information out of answers that we know were selected with our
  learning as the goal.
\item
  Selective learning literature also fits in here since reasoning about
  the expertise of another person might change the amount of belief
  updating that occurs from a certain answer. When we are selective with
  whom we ask for information, we are using their past behavior at
  generating answers to our queries and using that to guide our future
  question asking. Thus, the role of the people in the self-directed
  learning context matters for the calculus. Put another way, not all
  information sources are created equal.
\end{itemize}

Note that there's an interesting distinction between active learning
behaviors that directly affect social targets (or are directed towards
social targets) such as questions and social referencing compared to
active learning behaviors that might be changed by the presence of a
social partner or being in a communicative context such as gaze patterns
to explore the visual world or motor behaviors to explore a new toy.

\subsubsection{Stopping rules}\label{stopping-rules}

\begin{itemize}
\item
  Case study 1: luke and ellen's work on persistence in pedagogcial
  contexts
\item
  Case study 2: hyo and liz's work no exploration in pedagogical
  contexts.
\item
  \enquote{That said, new research on the development of questioning
  indicates that preschoolers can some- times determine when they have
  gathered sufficient information to address their questions (e.g.,
  Frazier, Gelman, \& Wellman, 2009; Kemler Nelson, Egan, \& Holt,
  2004).} (mills)
\end{itemize}

\section{Conclusions and a way
forward}\label{conclusions-and-a-way-forward}

Models of self-directed learning should include information the
social-communicative context in which learning often occurs. Reasoning
about other people modulate the choices that learners make: whether it's
who to talk to, what to look at, or what questions to ask. Moreover,
models of social learning should take into account the choice behaviors
available to the learner. i.e., think about teaching as reasoning about
another person's active learning or setting up a social learning context
where the learner selects actions

\newpage

\section{References}\label{references}

\setlength{\parindent}{-0.5in} \setlength{\leftskip}{0.5in}

\hypertarget{refs}{}
\hypertarget{ref-adriaans2017prosodic}{}
Adriaans, F., \& Swingley, D. (2017). Prosodic exaggeration within
infant-directed speech: Consequences for vowel learnability. \emph{The
Journal of the Acoustical Society of America}, \emph{141}(5),
3070--3078.

\hypertarget{ref-baldwin1993infants}{}
Baldwin, D. A. (1993). Infants' ability to consult the speaker for clues
to word reference. \emph{Journal of Child Language}, \emph{20}(02),
395--418.

\hypertarget{ref-begus2012infant}{}
Begus, K., \& Southgate, V. (2012). Infant pointing serves an
interrogative function. \emph{Developmental Science}, \emph{15}(5),
611--617.

\hypertarget{ref-begus2014infants}{}
Begus, K., Gliga, T., \& Southgate, V. (2014). Infants learn what they
want to learn: Responding to infant pointing leads to superior learning.

\hypertarget{ref-berlyne1960conflict}{}
Berlyne, D. E. (1960). Conflict, arousal, and curiosity.

\hypertarget{ref-birbili2009helping}{}
Birbili, M., \& Karagiorgou, I. (2009). Helping children and their
parents ask better questions: An intervention study. \emph{Journal of
Research in Childhood Education}, \emph{24}(1), 18--31.

\hypertarget{ref-bloom2002children}{}
Bloom, P. (2002). \emph{How children learn the meaning of words}. The
MIT Press.

\hypertarget{ref-bonawitz2012children}{}
Bonawitz, E. B., Schijndel, T. J. van, Friel, D., \& Schulz, L. (2012).
Children balance theories and evidence in exploration, explanation, and
learning. \emph{Cognitive Psychology}, \emph{64}(4), 215--234.

\hypertarget{ref-bonawitz2016computational}{}
Bonawitz, E., \& Shafto, P. (2016). Computational models of development,
social influences. \emph{Current Opinion in Behavioral Sciences},
\emph{7}, 95--100.

\hypertarget{ref-bonawitz2014win}{}
Bonawitz, E., Denison, S., Gopnik, A., \& Griffiths, T. L. (2014).
Win-stay, lose-sample: A simple sequential algorithm for approximating
bayesian inference. \emph{Cognitive Psychology}, \emph{74}, 35--65.

\hypertarget{ref-bova2013investigating}{}
Bova, A., \& Arcidiacono, F. (2013). Investigating children's
why-questions: A study comparing argumentative and explanatory function.
\emph{Discourse Studies}, \emph{15}(6), 713--734.

\hypertarget{ref-boyd2011cultural}{}
Boyd, R., Richerson, P. J., \& Henrich, J. (2011). The cultural niche:
Why social learning is essential for human adaptation. \emph{Proceedings
of the National Academy of Sciences}, \emph{108}(Supplement 2),
10918--10925.

\hypertarget{ref-bruner1961act}{}
Bruner, J. S. (1961). The act of discovery. \emph{Harvard Educational
Review}.

\hypertarget{ref-buchsbaum2011children}{}
Buchsbaum, D., Gopnik, A., Griffiths, T. L., \& Shafto, P. (2011).
Children's imitation of causal action sequences is influenced by
statistical and pedagogical evidence. \emph{Cognition}, \emph{120}(3),
331--340.

\hypertarget{ref-butler2012preschoolers}{}
Butler, L. P., \& Markman, E. M. (2012). Preschoolers use intentional
and pedagogical cues to guide inductive inferences and exploration.
\emph{Child Development}, \emph{83}(4), 1416--1428.

\hypertarget{ref-call2005copying}{}
Call, J., Carpenter, M., \& Tomasello, M. (2005). Copying results and
copying actions in the process of social learning: Chimpanzees (pan
troglodytes) and human children (homo sapiens). \emph{Animal Cognition},
\emph{8}(3), 151--163.

\hypertarget{ref-calvert2005control}{}
Calvert, S. L., Strong, B. L., \& Gallagher, L. (2005). Control as an
engagement feature for young children's attention to and learning of
computer content. \emph{American Behavioral Scientist}, \emph{48}(5),
578--589.

\hypertarget{ref-carpenter1998fourteen}{}
Carpenter, M., Akhtar, N., \& Tomasello, M. (1998). Fourteen-through
18-month-old infants differentially imitate intentional and accidental
actions. \emph{Infant Behavior and Development}, \emph{21}(2), 315--330.

\hypertarget{ref-carpenter2002understanding}{}
Carpenter, M., Call, J., \& Tomasello, M. (2002). Understanding ``prior
intentions'' enables two--year--olds to imitatively learn a complex
task. \emph{Child Development}, \emph{73}(5), 1431--1441.

\hypertarget{ref-casillas2014turn}{}
Casillas, M. (2014). Turn-taking. \emph{Pragmatic Development in First
Language Acquisition}, 53--70.

\hypertarget{ref-castro2009human}{}
Castro, R. M., Kalish, C., Nowak, R., Qian, R., Rogers, T., \& Zhu, X.
(2009). Human active learning. In \emph{Advances in neural information
processing systems} (pp. 241--248).

\hypertarget{ref-chi2009active}{}
Chi, M. T. (2009). Active-constructive-interactive: A conceptual
framework for differentiating learning activities. \emph{Topics in
Cognitive Science}, \emph{1}(1), 73--105.

\hypertarget{ref-chouinard2007children}{}
Chouinard, M. M., Harris, P. L., \& Maratsos, M. P. (2007). Children's
questions: A mechanism for cognitive development. \emph{Monographs of
the Society for Research in Child Development}, i--129.

\hypertarget{ref-cimpian2007subtle}{}
Cimpian, A., Arce, H.-M. C., Markman, E. M., \& Dweck, C. S. (2007).
Subtle linguistic cues affect children's motivation. \emph{Psychological
Science}, \emph{18}(4), 314--316.

\hypertarget{ref-clark2009first}{}
Clark, E. V. (2009). \emph{First language acquisition}. Cambridge
University Press.

\hypertarget{ref-cleveland2007joint}{}
Cleveland, A., Schug, M., \& Striano, T. (2007). Joint attention and
object learning in 5-and 7-month-old infants. \emph{Infant and Child
Development}, \emph{16}(3), 295--306.

\hypertarget{ref-coenen2017asking}{}
Coenen, A., Nelson, J. D., \& Gureckis, T. (2017). Asking the right
questions about human inquiry.

\hypertarget{ref-cook2011science}{}
Cook, C., Goodman, N. D., \& Schulz, L. E. (2011). Where science starts:
Spontaneous experiments in preschoolers' exploratory play.
\emph{Cognition}, \emph{120}(3), 341--349.

\hypertarget{ref-cooper1990preference}{}
Cooper, R. P., \& Aslin, R. N. (1990). Preference for infant-directed
speech in the first month after birth. \emph{Child Development},
\emph{61}(5), 1584--1595.

\hypertarget{ref-csibra2009natural}{}
Csibra, G., \& Gergely, G. (2009). Natural pedagogy. \emph{Trends in
Cognitive Sciences}, \emph{13}(4), 148--153.

\hypertarget{ref-davis1932form}{}
Davis, E. A. (1932). The form and function of children's questions.
\emph{Child Development}, \emph{3}(1), 57--74.

\hypertarget{ref-de2003investigating}{}
De Boer, B., \& Kuhl, P. K. (2003). Investigating the role of
infant-directed speech with a computer model. \emph{Acoustics Research
Letters Online}, \emph{4}(4), 129--134.

\hypertarget{ref-decasper1987human}{}
DeCasper, A. J., Fifer, W. P., Oates, J., \& Sheldon, S. (1987). Of
human bonding: Newborns prefer their mothers' voices. \emph{Cognitive
Development in Infancy}, 111--118.

\hypertarget{ref-dweck1988social}{}
Dweck, C. S., \& Leggett, E. L. (1988). A social-cognitive approach to
motivation and personality. \emph{Psychological Review}, \emph{95}(2),
256.

\hypertarget{ref-eaves2016infant}{}
Eaves Jr, B. S., Feldman, N. H., Griffiths, T. L., \& Shafto, P. (2016).
Infant-directed speech is consistent with teaching. \emph{Psychological
Review}, \emph{123}(6), 758.

\hypertarget{ref-elliott1988goals}{}
Elliott, E. S., \& Dweck, C. S. (1988). Goals: An approach to motivation
and achievement. \emph{Journal of Personality and Social Psychology},
\emph{54}(1), 5.

\hypertarget{ref-emery1998optimal}{}
Emery, A., \& Nenarokomov, A. V. (1998). Optimal experiment design.
\emph{Measurement Science and Technology}, \emph{9}(6), 864.

\hypertarget{ref-farroni2002eye}{}
Farroni, T., Csibra, G., Simion, F., \& Johnson, M. H. (2002). Eye
contact detection in humans from birth. \emph{Proceedings of the
National Academy of Sciences}, \emph{99}(14), 9602--9605.

\hypertarget{ref-farroni2007direct}{}
Farroni, T., Massaccesi, S., Menon, E., \& Johnson, M. H. (2007). Direct
gaze modulates face recognition in young infants. \emph{Cognition},
\emph{102}(3), 396--404.

\hypertarget{ref-fernald1987acoustic}{}
Fernald, A., \& Kuhl, P. (1987). Acoustic determinants of infant
preference for motherese speech. \emph{Infant Behavior and Development},
\emph{10}(3), 279--293.

\hypertarget{ref-fernald1991prosody}{}
Fernald, A., \& Mazzie, C. (1991). Prosody and focus in speech to
infants and adults. \emph{Developmental Psychology}, \emph{27}(2), 209.

\hypertarget{ref-fernald1984expanded}{}
Fernald, A., \& Simon, T. (1984). Expanded intonation contours in
mothers' speech to newborns. \emph{Developmental Psychology},
\emph{20}(1), 104.

\hypertarget{ref-frank2014inferring}{}
Frank, M. C., \& Goodman, N. D. (2014). Inferring word meanings by
assuming that speakers are informative. \emph{Cognitive Psychology},
\emph{75}, 80--96.

\hypertarget{ref-frank2009using}{}
Frank, M. C., Goodman, N. D., \& Tenenbaum, J. B. (2009). Using
speakers' referential intentions to model early cross-situational word
learning. \emph{Psychological Science}, \emph{20}(5), 578--585.

\hypertarget{ref-frazier2009preschoolers}{}
Frazier, B. N., Gelman, S. A., \& Wellman, H. M. (2009). Preschoolers'
search for explanatory information within adult--child conversation.
\emph{Child Development}, \emph{80}(6), 1592--1611.

\hypertarget{ref-gelman2008generic}{}
Gelman, S. A., Goetz, P. J., Sarnecka, B. W., \& Flukes, J. (2008).
Generic language in parent-child conversations. \emph{Language Learning
and Development}, \emph{4}(1), 1--31.

\hypertarget{ref-gergely2002developmental}{}
Gergely, G., Bekkering, H., \& Király, I. (2002). Developmental
psychology: Rational imitation in preverbal infants. \emph{Nature},
\emph{415}(6873), 755--755.

\hypertarget{ref-gerken2011infants}{}
Gerken, L., Balcomb, F. K., \& Minton, J. L. (2011). Infants avoid
`labouring in vain'by attending more to learnable than unlearnable
linguistic patterns. \emph{Developmental Science}, \emph{14}(5),
972--979.

\hypertarget{ref-goldin2007young}{}
Goldin-Meadow, S., Goodrich, W., Sauer, E., \& Iverson, J. (2007). Young
children use their hands to tell their mothers what to say.
\emph{Developmental Science}, \emph{10}(6), 778--785.

\hypertarget{ref-goldstein2008social}{}
Goldstein, M. H., \& Schwade, J. A. (2008). Social feedback to infants'
babbling facilitates rapid phonological learning. \emph{Psychological
Science}, \emph{19}(5), 515--523.

\hypertarget{ref-goodman2009cause}{}
Goodman, N. D., Baker, C. L., \& Tenenbaum, J. B. (2009). Cause and
intent: Social reasoning in causal learning. In \emph{Proceedings of the
31st annual conference of the cognitive science society} (pp.
2759--2764).

\hypertarget{ref-gopnik1999scientist}{}
Gopnik, A., Meltzoff, A. N., \& Kuhl, P. K. (1999). \emph{The scientist
in the crib: Minds, brains, and how children learn.} William Morrow \&
Co.

\hypertarget{ref-grabinger1995rich}{}
Grabinger, R. S., \& Dunlap, J. C. (1995). Rich environments for active
learning: A definition. \emph{Research in Learning Technology},
\emph{3}(2).

\hypertarget{ref-graf2013infant}{}
Graf Estes, K., \& Hurley, K. (2013). Infant-directed prosody helps
infants map sounds to meanings. \emph{Infancy}, \emph{18}(5), 797--824.

\hypertarget{ref-gunderson2013parent}{}
Gunderson, E. A., Gripshover, S. J., Romero, C., Dweck, C. S.,
Goldin-Meadow, S., \& Levine, S. C. (2013). Parent praise to 1-to
3-year-olds predicts children's motivational frameworks 5 years later.
\emph{Child Development}, \emph{84}(5), 1526--1541.

\hypertarget{ref-gureckis2012self}{}
Gureckis, T. M., \& Markant, D. B. (2012). Self-directed learning a
cognitive and computational perspective. \emph{Perspectives on
Psychological Science}, \emph{7}(5), 464--481.

\hypertarget{ref-hirsh2015putting}{}
Hirsh-Pasek, K., Zosh, J. M., Golinkoff, R. M., Gray, J. H., Robb, M.
B., \& Kaufman, J. (2015). Putting education in ``educational'' apps:
Lessons from the science of learning. \emph{Psychological Science in the
Public Interest}, \emph{16}(1), 3--34.

\hypertarget{ref-hollich2000breaking}{}
Hollich, G. J., Hirsh-Pasek, K., Golinkoff, R. M., Brand, R. J., Brown,
E., Chung, H. L., \ldots{} Bloom, L. (2000). Breaking the language
barrier: An emergentist coalition model for the origins of word
learning. \emph{Monographs of the Society for Research in Child
Development}, i--135.

\hypertarget{ref-johnson1991newborns}{}
Johnson, M. H., Dziurawiec, S., Ellis, H., \& Morton, J. (1991).
Newborns' preferential tracking of face-like stimuli and its subsequent
decline. \emph{Cognition}, \emph{40}(1), 1--19.

\hypertarget{ref-kachergis2013actively}{}
Kachergis, G., Yu, C., \& Shiffrin, R. M. (2013). Actively learning
object names across ambiguous situations. \emph{Topics in Cognitive
Science}, \emph{5}(1), 200--213.

\hypertarget{ref-kidd2012goldilocks}{}
Kidd, C., Piantadosi, S. T., \& Aslin, R. N. (2012). The goldilocks
effect: Human infants allocate attention to visual sequences that are
neither too simple nor too complex. \emph{PloS One}, \emph{7}(5),
e36399.

\hypertarget{ref-kidd2014goldilocks}{}
Kidd, C., Piantadosi, S. T., \& Aslin, R. N. (2014). The goldilocks
effect in infant auditory attention. \emph{Child Development},
\emph{85}(5), 1795--1804.

\hypertarget{ref-kishimoto2007pointing}{}
Kishimoto, T., Shizawa, Y., Yasuda, J., Hinobayashi, T., \& Minami, T.
(2007). Do pointing gestures by infants provoke comments from adults?
\emph{Infant Behavior and Development}, \emph{30}(4), 562--567.

\hypertarget{ref-klahr2004equivalence}{}
Klahr, D., \& Nigam, M. (2004). The equivalence of learning paths in
early science instruction effects of direct instruction and discovery
learning. \emph{Psychological Science}, \emph{15}(10), 661--667.

\hypertarget{ref-kline2015learn}{}
Kline, M. A. (2015). How to learn about teaching: An evolutionary
framework for the study of teaching behavior in humans and other
animals. \emph{Behavioral and Brain Sciences}, \emph{38}.

\hypertarget{ref-kuhl2007speech}{}
Kuhl, P. K. (2007). Is speech learning `gated'by the social brain?
\emph{Developmental Science}, \emph{10}(1), 110--120.

\hypertarget{ref-kuhl2003foreign}{}
Kuhl, P. K., Tsao, F.-M., \& Liu, H.-M. (2003). Foreign-language
experience in infancy: Effects of short-term exposure and social
interaction on phonetic learning. \emph{Proceedings of the National
Academy of Sciences}, \emph{100}(15), 9096--9101.

\hypertarget{ref-kuhn2008needs}{}
Kuhn, D., \& Pease, M. (2008). What needs to develop in the development
of inquiry skills? \emph{Cognition and Instruction}, \emph{26}(4),
512--559.

\hypertarget{ref-legare2013use}{}
Legare, C. H., Mills, C. M., Souza, A. L., Plummer, L. E., \& Yasskin,
R. (2013). The use of questions as problem-solving strategies during
early childhood. \emph{Journal of Experimental Child Psychology},
\emph{114}(1), 63--76.

\hypertarget{ref-lindley1956measure}{}
Lindley, D. V. (1956). On a measure of the information provided by an
experiment. \emph{The Annals of Mathematical Statistics}, 986--1005.

\hypertarget{ref-macdonald2017social}{}
MacDonald, K., Yurovsky, D., \& Frank, M. C. (2017). Social cues
modulate the representations underlying cross-situational learning.
\emph{Cognitive Psychology}, \emph{94}, 67--84.

\hypertarget{ref-markant2014better}{}
Markant, D. B., \& Gureckis, T. M. (2014). Is it better to select or to
receive? Learning via active and passive hypothesis testing.
\emph{Journal of Experimental Psychology: General}, \emph{143}(1), 94.

\hypertarget{ref-markant2016enhanced}{}
Markant, D. B., Ruggeri, A., Gureckis, T. M., \& Xu, F. (2016). Enhanced
memory as a common effect of active learning. \emph{Mind, Brain, and
Education}, \emph{10}(3), 142--152.

\hypertarget{ref-markant2014deconstructing}{}
Markant, D., DuBrow, S., Davachi, L., \& Gureckis, T. M. (2014).
Deconstructing the effect of self-directed study on episodic memory.
\emph{Memory \& Cognition}, \emph{42}(8), 1211--1224.

\hypertarget{ref-mccormack2016children}{}
McCormack, T., Bramley, N., Frosch, C., Patrick, F., \& Lagnado, D.
(2016). Children's use of interventions to learn causal structure.
\emph{Journal of Experimental Child Psychology}, \emph{141}, 1--22.

\hypertarget{ref-medina2011words}{}
Medina, T. N., Snedeker, J., Trueswell, J. C., \& Gleitman, L. R.
(2011). How words can and cannot be learned by observation.
\emph{Proceedings of the National Academy of Sciences}, \emph{108}(22),
9014--9019.

\hypertarget{ref-mills2012little}{}
Mills, C. M., Danovitch, J. H., Grant, M. G., \& Elashi, F. B. (2012).
Little pitchers use their big ears: Preschoolers solve problems by
listening to others ask questions. \emph{Child Development},
\emph{83}(2), 568--580.

\hypertarget{ref-mills2010preschoolers}{}
Mills, C. M., Legare, C. H., Bills, M., \& Mejias, C. (2010).
Preschoolers use questions as a tool to acquire knowledge from different
sources. \emph{Journal of Cognition and Development}, \emph{11}(4),
533--560.

\hypertarget{ref-mills2011determining}{}
Mills, C. M., Legare, C. H., Grant, M. G., \& Landrum, A. R. (2011).
Determining who to question, what to ask, and how much information to
ask for: The development of inquiry in young children. \emph{Journal of
Experimental Child Psychology}, \emph{110}(4), 539--560.

\hypertarget{ref-najemnik2005optimal}{}
Najemnik, J., \& Geisler, W. S. (2005). Optimal eye movement strategies
in visual search. \emph{Nature}, \emph{434}(7031), 387.

\hypertarget{ref-nelson2005finding}{}
Nelson, J. D. (2005). Finding useful questions: On bayesian
diagnosticity, probability, impact, and information gain.
\emph{Psychological Review}, \emph{112}(4).

\hypertarget{ref-nelson2010experience}{}
Nelson, J. D., McKenzie, C. R., Cottrell, G. W., \& Sejnowski, T. J.
(2010). Experience matters: Information acquisition optimizes
probability gain. \emph{Psychological Science}, \emph{21}(7), 960--969.

\hypertarget{ref-olson2011infants}{}
Olson, J., \& Masur, E. F. (2011). Infants' gestures influence mothers'
provision of object, action and internal state labels. \emph{Journal of
Child Language}, \emph{38}(5), 1028--1054.

\hypertarget{ref-partridge2015young}{}
Partridge, E., McGovern, M. G., Yung, A., \& Kidd, C. (2015). Young
children's self-directed information gathering on touchscreens. In
\emph{Proceedings of the 37th annual conference of the cognitive science
society, austin, tx. cognitive science society}.

\hypertarget{ref-pegg1992preference}{}
Pegg, J. E., Werker, J. F., \& McLeod, P. J. (1992). Preference for
infant-directed over adult-directed speech: Evidence from 7-week-old
infants. \emph{Infant Behavior and Development}, \emph{15}(3), 325--345.

\hypertarget{ref-prince2004does}{}
Prince, M. (2004). Does active learning work? A review of the research.
\emph{Journal of Engineering Education}, \emph{93}(3), 223--231.

\hypertarget{ref-ramirez2017active}{}
Ramirez-Loaiza, M. E., Sharma, M., Kumar, G., \& Bilgic, M. (2017).
Active learning: An empirical study of common baselines. \emph{Data
Mining and Knowledge Discovery}, \emph{31}(2), 287--313.

\hypertarget{ref-reid2005adult}{}
Reid, V. M., \& Striano, T. (2005). Adult gaze influences infant
attention and object processing: Implications for cognitive
neuroscience. \emph{European Journal of Neuroscience}, \emph{21}(6),
1763--1766.

\hypertarget{ref-rogoff1993guided}{}
Rogoff, B., Mistry, J., Göncü, A., Mosier, C., Chavajay, P., \& Heath,
S. B. (1993). Guided participation in cultural activity by toddlers and
caregivers. \emph{Monographs of the Society for Research in Child
Development}, i--179.

\hypertarget{ref-rothe2015asking}{}
Rothe, A., Lake, B. M., \& Gureckis, T. M. (2015). Asking useful
questions: Active learning with rich queries. In \emph{CogSci}.

\hypertarget{ref-rowe2009early}{}
Rowe, M. L., \& Goldin-Meadow, S. (2009). Early gesture selectively
predicts later language learning. \emph{Developmental Science},
\emph{12}(1), 182--187.

\hypertarget{ref-rowe2017going}{}
Rowe, M. L., Leech, K. A., \& Cabrera, N. (2017). Going beyond input
quantity: Wh-questions matter for toddlers' language and cognitive
development. \emph{Cognitive Science}, \emph{41}(S1), 162--179.

\hypertarget{ref-ruggeri2015children}{}
Ruggeri, A., \& Lombrozo, T. (2015). Children adapt their questions to
achieve efficient search. \emph{Cognition}, \emph{143}, 203--216.

\hypertarget{ref-ruggeri2016active}{}
Ruggeri, A., Markant, D. B., Gureckis, T. M., \& Xu, F. (2016). Active
control of study leads to improved recognition memory in children. In
\emph{Proceedings of the 38th annual conference of the cognitive science
society. austin, tx: Cognitive science society}.

\hypertarget{ref-sage2011disentangling}{}
Sage, K. D., \& Baldwin, D. (2011). Disentangling the social and the
pedagogical in infants' learning about tool-use. \emph{Social
Development}, \emph{20}(4), 825--844.

\hypertarget{ref-schulz2012origins}{}
Schulz, L. (2012). The origins of inquiry: Inductive inference and
exploration in early childhood. \emph{Trends in Cognitive Sciences},
\emph{16}(7), 382--389.

\hypertarget{ref-schulz2007serious}{}
Schulz, L. E., \& Bonawitz, E. B. (2007). Serious fun: Preschoolers
engage in more exploratory play when evidence is confounded.
\emph{Developmental Psychology}, \emph{43}(4), 1045.

\hypertarget{ref-senju2008gaze}{}
Senju, A., \& Csibra, G. (2008). Gaze following in human infants depends
on communicative signals. \emph{Current Biology}, \emph{18}(9),
668--671.

\hypertarget{ref-settles2012active}{}
Settles, B. (2012). Active learning. \emph{Synthesis Lectures on
Artificial Intelligence and Machine Learning}, \emph{6}(1), 1--114.

\hypertarget{ref-shafto2012epistemic}{}
Shafto, P., Eaves, B., Navarro, D. J., \& Perfors, A. (2012). Epistemic
trust: Modeling children's reasoning about others' knowledge and intent.
\emph{Developmental Science}, \emph{15}(3), 436--447.

\hypertarget{ref-shafto2012learning}{}
Shafto, P., Goodman, N. D., \& Frank, M. C. (2012). Learning from others
the consequences of psychological reasoning for human learning.
\emph{Perspectives on Psychological Science}, \emph{7}(4), 341--351.

\hypertarget{ref-shafto2014rational}{}
Shafto, P., Goodman, N. D., \& Griffiths, T. L. (2014). A rational
account of pedagogical reasoning: Teaching by, and learning from,
examples. \emph{Cognitive Psychology}, \emph{71}, 55--89.

\hypertarget{ref-singh2009influences}{}
Singh, L., Nestor, S., Parikh, C., \& Yull, A. (2009). Influences of
infant-directed speech on early word recognition. \emph{Infancy},
\emph{14}(6), 654--666.

\hypertarget{ref-siskind1996computational}{}
Siskind, J. M. (1996). A computational study of cross-situational
techniques for learning word-to-meaning mappings. \emph{Cognition},
\emph{61}(1), 39--91.

\hypertarget{ref-smith2008infants}{}
Smith, L. B., \& Yu, C. (2008). Infants rapidly learn word-referent
mappings via cross-situational statistics. \emph{Cognition},
\emph{106}(3), 1558--1568.

\hypertarget{ref-stahl2015observing}{}
Stahl, A. E., \& Feigenson, L. (2015). Observing the unexpected enhances
infants' learning and exploration. \emph{Science}, \emph{348}(6230),
91--94.

\hypertarget{ref-thiessen2005infant}{}
Thiessen, E. D., Hill, E. A., \& Saffran, J. R. (2005). Infant-directed
speech facilitates word segmentation. \emph{Infancy}, \emph{7}(1),
53--71.

\hypertarget{ref-tomasello1986joint}{}
Tomasello, M., \& Farrar, M. J. (1986). Joint attention and early
language. \emph{Child Development}, 1454--1463.

\hypertarget{ref-trueswell2013propose}{}
Trueswell, J. C., Medina, T. N., Hafri, A., \& Gleitman, L. (2013).
Propose but verify: Fast mapping meets cross-situational word learning.
\emph{Cognitive Psychology}, \emph{66}(1), 126--156.

\hypertarget{ref-voss2011spontaneous}{}
Voss, J. L., Warren, D. E., Gonsalves, B. D., Federmeier, K. D., Tranel,
D., \& Cohen, N. J. (2011). Spontaneous revisitation during visual
exploration as a link among strategic behavior, learning, and the
hippocampus. \emph{Proceedings of the National Academy of Sciences},
\emph{108}(31), E402--E409.

\hypertarget{ref-vouloumanos2007listening}{}
Vouloumanos, A., \& Werker, J. F. (2007). Listening to language at
birth: Evidence for a bias for speech in neonates. \emph{Developmental
Science}, \emph{10}(2), 159--164.

\hypertarget{ref-vygotsky1987zone}{}
Vygotsky, L. (1987). Zone of proximal development. \emph{Mind in
Society: The Development of Higher Psychological Processes},
\emph{5291}, 157.

\hypertarget{ref-weisberg2013guided}{}
Weisberg, D. S., Hirsh-Pasek, K., \& Golinkoff, R. M. (2013). Guided
play: Where curricular goals meet a playful pedagogy. \emph{Mind, Brain,
and Education}, \emph{7}(2), 104--112.

\hypertarget{ref-wu2015caregivers}{}
Wu, Z., \& Gros-Louis, J. (2015). Caregivers provide more labeling
responses to infants' pointing than to infants' object-directed
vocalizations. \emph{Journal of Child Language}, \emph{42}(3), 538--561.

\hypertarget{ref-xu2007word}{}
Xu, F., \& Tenenbaum, J. B. (2007). Word learning as bayesian inference.
\emph{Psychological Review}, \emph{114}(2), 245.

\hypertarget{ref-yoon2008communication}{}
Yoon, J. M., Johnson, M. H., \& Csibra, G. (2008). Communication-induced
memory biases in preverbal infants. \emph{Proceedings of the National
Academy of Sciences}, \emph{105}(36), 13690--13695.

\hypertarget{ref-yu2017peagogical}{}
Yu, B., Yue. (2017). Pedagogical questions in parent-child
conversations. \emph{Child Development}.

\hypertarget{ref-yu2007unified}{}
Yu, C., \& Ballard, D. H. (2007). A unified model of early word
learning: Integrating statistical and social cues.
\emph{Neurocomputing}, \emph{70}(13), 2149--2165.

\hypertarget{ref-yu2012embodied}{}
Yu, C., \& Smith, L. B. (2012a). Embodied attention and word learning by
toddlers. \emph{Cognition}.

\hypertarget{ref-yu2012modeling}{}
Yu, C., \& Smith, L. B. (2012b). Modeling cross-situational
word--referent learning: Prior questions. \emph{Psychological Review},
\emph{119}(1), 21.

\hypertarget{ref-yu2013joint}{}
Yu, C., \& Smith, L. B. (2013). Joint attention without gaze following:
Human infants and their parents coordinate visual attention to objects
through eye-hand coordination. \emph{PloS One}, \emph{8}(11), e79659.

\hypertarget{ref-yu2016social}{}
Yu, C., \& Smith, L. B. (2016). The social origins of sustained
attention in one-year-old human infants. \emph{Current Biology},
\emph{26}(9), 1235--1240.

\hypertarget{ref-yu2005role}{}
Yu, C., Ballard, D. H., \& Aslin, R. N. (2005). The role of embodied
intention in early lexical acquisition. \emph{Cognitive Science},
\emph{29}(6), 961--1005.

\hypertarget{ref-yurovsky2014algorithmic}{}
Yurovsky, D., \& Frank, M. C. (2015). An integrative account of
constraints on cross-situational learning. \emph{Cognition}.






\end{document}
